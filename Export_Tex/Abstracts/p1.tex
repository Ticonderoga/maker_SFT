

    %=====================================
    %   WARNING
    %   FICHIER AUTOMATISE
    %   NE PAS MODIFIER
    %=====================================

    \newpage

%%%%%%%%%%%%%%%%%%%%%%%%%%%%%%%%%%%%%%%%%%%%
%% Papier 1
%%%%%%%%%%%%%%%%%%%%%%%%%%%%%%%%%%%%%%%%%%%%

% Indexations
\index{VilleminThomas@Villemin, Thomas}
\index{FargesOlivier@Farges, Olivier}
\index{ParentGilles@Parent, Gilles}
\index{ClaverieRemy@Claverie, Rémy}
\index{BouyerJulien@Bouyer, Julien}
%
% Titre
\begin{flushleft}
\phantomsection\addtocounter{section}{1}
\addcontentsline{toc}{section}{Modélisation et intégration temporelle d'un problème thermique couplé par la méthode de Monte Carlo}
{\Large \textbf{Modélisation et intégration temporelle d'un problème thermique couplé par la méthode de Monte Carlo}}\label{ref:1}
\end{flushleft}
%
% Auteurs
Thomas Villemin$^{1,\star}$, Olivier Farges$^{2}$, Gilles Parent$^{2}$, Rémy Claverie$^{3}$, Julien Bouyer$^{4}$\\[2mm]
$^{\star}$ \Letter : \url{thomas.villemin@univ-lorraine.fr}\\[2mm]
{\footnotesize $^{1}$ Université de Lorraine, CNRS, LEMTA, F-54000 Nancy, France ; Cerema Est, Équipe de Recherche TEAM, 71 rue de la grande haie – 54510 Tomblaine, France}\\
{\footnotesize $^{2}$ Université de Lorraine, CNRS, LEMTA, F-54000 Nancy, France}\\
{\footnotesize $^{3}$ Cerema, Équipe Recherche TEAM, 71 rue de la grande haie, 54510 Tomblaine, France}\\
{\footnotesize $^{4}$ Cerema Est, Équipe de Recherche TEAM, 71 rue de la grande haie – 54510 Tomblaine, France}\\
[4mm]
%
% Mots clés
\noindent \textbf{Mots clés : } Méthode de Monte Carlo, Couplage, Conduction, Rayonnement, Convection\\[4mm]
%
% Résumé
\noindent \textbf{Résumé : } 

{\normalsize
La fonction des panneaux photovoltaïques est de convertir l'énergie solaire reçue sous forme de rayonnement en énergie électrique. Cependant, une part importante de l'énergie incidente est aussi transformée en chaleur conduisant à une élévation rapide de la température du panneau, supérieure à 70$^{\circ}$C en surface, ce qui a pour conséquence une baisse significative du rendement. Afin d'évaluer correctement ce rendement, une modélisation thermique est nécessaire car il dépend directement de la température des cellules photovoltaïques, grandeur soumise à la variation temporelle des paramètres climatiques dans l'environnement du panneau. De nombreux modèles ont été développés en se basant sur des hypothèses ne prenant pas en compte – ou très peu – les conditions environnementales locales (e.g. variables climatiques, masques solaires) ou l'ensemble des phénomènes physiques (e.g. bilan d'énergie de surface).







Ce travail porte spécifiquement sur la modélisation thermique d'un milieu solide opaque en environnement réaliste soumis à des conditions climatiques effectives. L'aspect semi-transparent d'un panneau solaire n'est pas pris en considération dans notre étude. Le modèle développé avec la méthode de Monte Carlo couple tous les types de transferts thermiques – rayonnement, convection, conduction et flux solaire – appliqués à une géométrie simple en intégrant temporellement les paramètres climatiques. Par ailleurs, la méthode de Monte Carlo ayant déjà fait ses preuves sur sa capacité à gérer les géométries complexes, le passage à une géométrie réelle (e.g. multicouche, plusieurs matériaux) n'ajoutera pas de complexité au modèle.







La géométrie choisie est une plaque de dimensions 1 m × 1 m × 0,035 m disposée parallèlement au sol. Les simulations des températures de la plaque sont effectuées en considérant plusieurs matériaux. Les paramètres climatiques sont le rayonnement global ($\unit{W\cdot m^{-2}}$), les températures d'air et de surface de sol ($^{\circ}$C) ainsi que la vitesse du vent ($\unit{m \cdot s^{-1}}$). Les données pro- viennent de la plateforme expérimentale du Cerema à Nancy.







Les résultats obtenus mettent en évidence l'influence du choix des matériaux et des para- mètres climatiques sur les températures simulées et valident les hypothèses initiales.

 \vfill doi : \url{https://doi.org/10.25855/SFT2021-001}

}
 
