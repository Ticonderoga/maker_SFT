

    %=====================================
    %   WARNING
    %   FICHIER AUTOMATISE
    %   NE PAS MODIFIER
    %=====================================

    \newpage

\backgroundsetup{contents={Work In Progress},scale=7}
\BgThispage
%%%%%%%%%%%%%%%%%%%%%%%%%%%%%%%%%%%%%%%%%%%%
%% Papier 62
%%%%%%%%%%%%%%%%%%%%%%%%%%%%%%%%%%%%%%%%%%%%

% Indexations
\index{ChevalierStephane@Chevalier, Stéphane}
\index{SommierAlain@Sommier, Alain}
\index{GrozMarie-Marthe@Groz, Marie-Marthe}
\index{AbissetEmmanuelle@Abisset, Emmanuelle}
\index{BatsaleJean-Christophe@Batsale, Jean-Christophe}
\index{PradereChristophe@Pradère, Christophe}
%
% Titre
\begin{flushleft}
\phantomsection\addtocounter{section}{1}
\addcontentsline{toc}{section}{Mesure du flux thermique à l'interface buse/polymère dans le procédé de fabrication additive par dépôt de fil fondu}
{\Large \textbf{Mesure du flux thermique à l'interface buse/polymère dans le procédé de fabrication additive par dépôt de fil fondu}}\label{ref:62}
\end{flushleft}
%
% Auteurs
Stéphane Chevalier$^{1,\star}$, Alain Sommier$^{1}$, Marie-Marthe Groz$^{1}$, Emmanuelle Abisset$^{1}$, Jean-Christophe Batsale$^{1}$, Christophe Pradère$^{1}$\\[2mm]
$^{\star}$ \Letter : \url{stephane.chevalier@u-bordeaux.fr}\\[2mm]
{\footnotesize $^{1}$ I2M}\\
[4mm]
%
% Mots clés
\noindent \textbf{Mots clés : } Fabrication additive, méthode inverse, thermographie infrarouge,\\[4mm]
%
% Résumé
\noindent \textbf{Résumé : } 

{\normalsize
Le procédé de dépôt de fil de fondu permet la fabrication de pièces en polymère par adition de matière. Ce procédé devient largement répandu dans l'industrie pour la fabrication de pièces plastiques non contraintes mécaniquement. Afin de le fiabiliser et de rendre robuste la production en série ces pièces plastiques, la modélisation est souvent employée pour prédire les propriétés thermomécaniques et/ou prévenir l'apparition d'éventuels défauts lors de la fabrication.







Dans ce contexte, la connaissance de la condition de flux à l'interface buse/polymère lors du dépôt de fil fondu est essentielle pour prédire correctement le transfert de chaleur dans la pièce en cours de fabrication, puis ses propriétés mécaniques finales. Quelques travaux ont déjà été réalisés en ce sens, mais jusqu'à présent les auteurs s'intéressaient à la température d'interface plutôt qu'à la mesure du flux thermique. Ils ont néanmoins montré que la connaissance et le contrôle de cette condition à la limite a une grande influence sur la qualité finale de la pièce.







Pour répondre à cette problématique, une imprimante 3D Delta a été instrumentée durant la fabrication d'un mur en acide polylactique (PLA) de plusieurs millimètres de haut et de 500~$\unit{\mu m}$ d'épaisseur. Une caméra thermique (FLIR SC7600) est utilisée pour la mesure transitoire du champ de température avec une résolution spatiale de 72 $\unit{\mu m/px}$. Ces mesures sont analysées avec un modèle linéaire Eulerien du transfert thermique unidirectionnelle dans un premier temps. Les grandes vitesses de déplacements de buse (de l'ordre de 30 mm/s) permettent de faire l'hypothèse que les gradients conductifs longitudinaux sont négligeables devant le transport de chaleur dû aux déplacements de la buse. L'inversion par les moindres carrés de ce modèle à partir les champs thermiques mesurés permet de remonter à l'estimation du flux en amplitude et en forme. Les résultats obtenus sont cohérents avec la puissance thermique moyenne apportée par la buse lors de la fusion du polymère. Ils seront présentés en détail dans cette communication, et l'influence de quelques paramètres procédés (vitesse de buse, refroidissement) sur l'amplitude et la forme du flux sera étudiée.

 \vfill Work In Progress

}
 
