

    %=====================================
    %   WARNING
    %   FICHIER AUTOMATISE
    %   NE PAS MODIFIER
    %=====================================

    \newpage

%%%%%%%%%%%%%%%%%%%%%%%%%%%%%%%%%%%%%%%%%%%%
%% Papier 10
%%%%%%%%%%%%%%%%%%%%%%%%%%%%%%%%%%%%%%%%%%%%

% Indexations
\index{ChtiouiFeryal@Chtioui, Feryal}
\index{BozonnetEmmanuel@Bozonnet, Emmanuel}
\index{SalagnacPatrick@Salagnac, Patrick}
\index{MachardAnais@Machard, Anaïs}
%
% Titre
\begin{flushleft}
\phantomsection\addtocounter{section}{1}
\addcontentsline{toc}{section}{Comparaison de deux techniques de rafraîchissement passif en toiture sous différentes conditions climatiques}
{\Large \textbf{Comparaison de deux techniques de rafraîchissement passif en toiture sous différentes conditions climatiques}}\label{ref:10}
\end{flushleft}
%
% Auteurs
Feryal Chtioui$^{1,\star}$, Emmanuel Bozonnet$^{1}$, Patrick Salagnac$^{1}$, Anaïs Machard$^{1}$\\[2mm]
$^{\star}$ \Letter : \url{feryal.chtioui1@univ-lr.fr}\\[2mm]
{\footnotesize $^{1}$ Université de la Rochelle, LaSIE}\\
[4mm]
%
% Mots clés
\noindent \textbf{Mots clés : } Rafraîchissement passif, Bassin en toiture-terrasse, Bâtiment commercial, Indicateurs de performance, Simulation, Modèle thermique, Changement climatique\\[4mm]
%
% Résumé
\noindent \textbf{Résumé : } 

{\normalsize
Suite aux changements climatiques, les bâtiments doivent s'adapter à de nouvelles contraintes. Jusqu'à présent, lorsqu'un bâtiment était dimensionné du point de vue thermique, on s'intéressait essentiellement à la problématique d'hiver (chauffage du bâtiment). Depuis quelques années, du fait de la hausse des températures et de périodes caniculaires en été, les bâtiments doivent répondre à de nouveaux enjeux qui sont d'éviter l'inconfort d'été. Pour faire face à la forte augmentation de la consommation en climatisation, il est nécessaire de développer des systèmes de rafraîchissement dits « passifs ». Ce contexte nous amène à étudier, en période estivale, une technique de rafraîchissement passif par rétention d'eau en toiture-terrasse de bâtiments commerciaux, d'en évaluer son potentiel et de la comparer à une autre solution de rafraîchissement passif par revêtement froid de type « cool-roof ».



Après une description des différents mécanismes physiques de transfert au sein d'une toiture terrasse et du modèle numérique développé, une étude paramétrique est réalisée pour la technique de rétention d'eau en prenant en compte l'épaisseur de la lame d'eau, les propriétés radiatives de la toiture, différents climats… Cette technique est basée sur les phénomènes d'évaporation, d'échanges radiatifs nocturnes et d'inertie thermique de la masse d'eau. En effet, le modèle théorique décrit le comportement thermique de la technique de rafraîchissement passif par rétention d'eau en toiture-terrasse avec l'ambiance intérieure du bâtiment et l'air extérieur en tenant compte des propriétés de l'enveloppe du bâtiment et des conditions climatiques ainsi que les paramètres de conception de dispositif.



Dans une seconde partie, la toiture « évaporative » est comparée à une toiture avec revêtement « cool » à l'aide d'indicateurs de performance clés (KPIs) pour différents climats actuels et futures (méditerranéen, océanique) et pour une période caniculaire afin d'évaluer les performances de deux solutions passives ainsi que l'impact des paramètres de conception.



Après une description des différents mécanismes de transfert au sein d'une toiture terrasse et du modèle numérique développé, une étude paramétrique est réalisée pour la technique de rétention d'eau prenant en compte l'épaisseur de la lame d'eau, les propriétés radiatives de la toiture, différents climats… Dans une seconde partie, la toiture « évaporative » est comparée à une toiture avec revêtement « cool » à l'aide d'indicateurs de performance pour différents climats actuels et futures (méditerranéen, océanique) et pour une période caniculaire.

 \vfill doi : \url{https://doi.org/10.25855/SFT2021-010}

}
 
