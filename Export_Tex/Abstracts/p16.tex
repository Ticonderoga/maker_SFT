

    %=====================================
    %   WARNING
    %   FICHIER AUTOMATISE
    %   NE PAS MODIFIER
    %=====================================

    \newpage

\backgroundsetup{contents={Work In Progress},scale=7}
\BgThispage
%%%%%%%%%%%%%%%%%%%%%%%%%%%%%%%%%%%%%%%%%%%%
%% Papier 16
%%%%%%%%%%%%%%%%%%%%%%%%%%%%%%%%%%%%%%%%%%%%

% Indexations
\index{DelaleuxFabien@Delaleux, Fabien}
\index{ZanattaLuca@Zanatta, Luca}
\index{ArnaudPierre@Arnaud, Pierre}
\index{SaintLaurentClaire@Saint Laurent, Claire}
\index{DurastantiJean-Felix@Durastanti, Jean-Félix}
%
% Titre
\begin{flushleft}
\phantomsection\addtocounter{section}{1}
\addcontentsline{toc}{section}{Production d'électricité par valorisation énergétique des effluents de station d'épuration}
{\Large \textbf{Production d'électricité par valorisation énergétique des effluents de station d'épuration}}\label{ref:16}
\end{flushleft}
%
% Auteurs
Fabien Delaleux$^{1,\star}$, Luca Zanatta$^{2}$, Pierre Arnaud$^{2}$, Claire Saint Laurent$^{2}$, Jean-Félix Durastanti$^{1}$\\[2mm]
$^{\star}$ \Letter : \url{fabien.delaleux@u-pec.fr}\\[2mm]
{\footnotesize $^{1}$ CERTES, Université Paris Est Créteil}\\
{\footnotesize $^{2}$ SIARCE}\\
[4mm]
%
% Mots clés
\noindent \textbf{Mots clés : } valorisation énergétique, solaire, STEP\\[4mm]
%
% Résumé
\noindent \textbf{Résumé : } 

{\normalsize
Ce travail est le fruit d'une collaboration entre le laboratoire CERTES et le SIARCE (syndicat des eaux ayant en charge les réseaux d'assainissement et des stations d'épuration en Essonne). Il s'inscrit dans le cadre d'un projet de Schéma Directeur Syndical des Energies Renouvelables et Ressources Réutilisables, qui permet de fixer les objectifs et les moyens de mise en œuvre de projets relatifs aux énergies renouvelables et à la récupération d'énergie.



L'objectif est de réaliser une analyse du potentiel de récupération énergétique des effluents en sortie d'une station d'épuration située à Vert-le-Grand (91) et de proposer des solutions de mise en œuvre. Cette STEP traite les eaux de 3970 équivalent habitants correspondant à une charge hydraulique de l'ordre de 1400~$\unit{m^3}$ par jour ce qui en fait une station de taille moyenne. Sa consommation énergétique moyenne de 0,79~$\unit{kWh.m^{-3}}$ traité la classe parmi la moyenne haute en termes de consommation, à procédé de traitement équivalent.



L'étude énergétique des effluents en sortie de la STEP de Vert-le-Grand montre que le débit (inférieur à 40~$\unit{m^3.h^{-1}}$ 90\% du temps) et la température (entre 10 et 20$^{\circ}$C en fonction de la saison) sont faibles et difficilement exploitables sous forme de chaleur pour alimenter un réseau urbain, d'autant plus que sa situation géographique, entourée de terres agricoles, ne le permet pas.



L'idée originale de ce projet est de considérer cette ressource énergétique non pas comme une source chaude d'un procédé de production mais de l'envisager comme une source froide  thermiquement très stable tout au long de l'année. L'option envisagée, objet de cette pré-étude, est donc la production d'électricité par un cycle thermodynamique de type ORC en utilisant un système à concentration solaire comme source chaude (rendu possible par la surface au sol disponible dans l'enceinte même de la STEP) et d'utiliser les effluents de sortie comme source froide au condenseur de la boucle thermodynamique. Le principal avantage de cette solution est qu'elle utilise deux sources énergétiques gratuites et abondantes. Pour un ensoleillement moyen de 135 kWh/m² en période hivernale et 215 kWh/m²  en période estivale, une première approche montre qu'avec environ 900 m² de surface disponible, la production d'électricité annuelle moyenne s'élèverait à plus 300 MWh, couvrant ainsi une grande part (70 à 100\% suivant les hypothèses) des besoins de la STEP.



La suite du travail est de compléter cette pré-étude par une modélisation plus fine des performances et de la production attendue, puis de dimensionner et d'installer un démonstrateur sur site qui intégrera des solutions d'optimisation comme un système de stockage thermique. L'objectif est de faire la preuve du concept à l'aide de ce prototype avant d'envisager un déploiement à plus grande échelle.

 \vfill Work In Progress

}
 
