

    %=====================================
    %   WARNING
    %   FICHIER AUTOMATISE
    %   NE PAS MODIFIER
    %=====================================

    \newpage

%%%%%%%%%%%%%%%%%%%%%%%%%%%%%%%%%%%%%%%%%%%%
%% Papier 42
%%%%%%%%%%%%%%%%%%%%%%%%%%%%%%%%%%%%%%%%%%%%

% Indexations
\index{BillardEtienne@Billard, Etienne}
\index{LeCocqThomas@Le Cocq, Thomas}
\index{PicgirardFabien@Picgirard, Fabien}
\index{HoarauGuillaume@Hoarau, Guillaume}
\index{MartinJean-Francois@Martin, Jean-François}
\index{Castaing-LasvignottesJean@Castaing-Lasvignottes, Jean}
\index{MarcOlivier@Marc, Olivier}
%
% Titre
\begin{flushleft}
\phantomsection\addtocounter{section}{1}
\addcontentsline{toc}{section}{Étude des performances de climatiseurs individuels en climat tropical}
{\Large \textbf{Étude des performances de climatiseurs individuels en climat tropical}}\label{ref:42}
\end{flushleft}
%
% Auteurs
Etienne Billard$^{1}$, Thomas Le Cocq$^{1}$, Fabien Picgirard$^{2}$, Guillaume Hoarau$^{3}$, Jean-François Martin$^{1}$, Jean Castaing-Lasvignottes$^{1}$, Olivier Marc$^{1,\star}$\\[2mm]
$^{\star}$ \Letter : \url{olivier.marc@univ-reunion.fr}\\[2mm]
{\footnotesize $^{1}$ PIMENT}\\
{\footnotesize $^{2}$ ADEME}\\
{\footnotesize $^{3}$ EDF}\\
[4mm]
%
% Mots clés
\noindent \textbf{Mots clés : } Climatiseurs individuels, banc de test, charge partielle, Energy Efficiency Ratio\\[4mm]
%
% Résumé
\noindent \textbf{Résumé : } 

{\normalsize
Situé dans l'Océan Indien près de Madagascar, l'île de La Réunion présente un contexte énergétique particulier dans la mesure où elle est une Zone Non Interconnectée (ZNI) et l'équilibre production - consommation se fait donc à l'échelle de l'île. Le mix énergétique est très carboné (680 g$\unit{CO_2}$/kWh) car il est constitué à 64\% d'énergies fossiles (fioul, charbon) et à 36\% d'énergies renouvelables. Le secteur du bâtiment représente environ 80\% de la consommation électrique de l'île dont la moitié pour la climatisation, qui est un des postes les plus énergivores de l'île. Face à ce constat, un projet visant à réduire la consommation électrique des systèmes de climatisation individuels sur l'île est mené par EDF et l'ADEME en partenariat avec le laboratoire PIMENT de l'Université de La Réunion. L'objectif est de déterminer les modèles les plus vertueux pour les valoriser sur le marché réunionnais.



Afin de mener à bien ce projet, les tests sont réalisés sur un banc d'essai constitué de deux enceintes mitoyennes contrôlées en température et en humidité. La première simule les besoins thermiques du bâtiment, l'autre les conditions climatiques extérieures réunionnaises et le système étudié est installé à l'interface de ces dernières. En premier lieu, ce papier présente le banc de test : l'instrumentation, les appareils de mesures et de contrôles, ainsi que les résultats expérimentaux pour un modèle de climatiseur 7000 BTU/h de classe énergétique A+++. Afin d'en étudier les performances, nous avons retenu principalement deux indicateurs : la consommation électrique et le coefficient EER (Energy Efficiency Ratio). L'étude évalue, explique et quantifie l'influence de 4 conditions de fonctionnement sur ces indicateurs : le pourcentage de la charge nominale de la climatisation (100\%, 74\%, 47\%, 21\%), la température extérieure (35$^{\circ}$C, 30$^{\circ}$C, 25$^{\circ}$C, 20$^{\circ}$C), la température intérieure (de 21$^{\circ}$C à 28$^{\circ}$C avec un pas de 1$^{\circ}$C) et l'hygrométrie intérieure (avec ou sans apport de chaleur latente). Un autre indicateur pertinent est le SEER (Seasonal Energy Efficiency Ratio) qui permet d'associer une classe énergétique au climatiseur. D'après la norme NF EN 14825, il est calculé grâce aux tests à charges partielles en pondérant les EER obtenus expérimentalement avec le nombre d'heures d'utilisation de la climatisation pour une température extérieure associée. L'analyse du climat Réunionnais montre des différences notables entre les heures d'utilisation données par la norme et celles rencontrées localement. A l'image du calcul du SCOP pour les pompes à chaleur qui tient compte de l'hétérogénéité des climats européens, nous choisissons de distinguer l'usage d'un climatiseur en résidentiel ou en tertiaire. Pour qualifier la performance de ces derniers et afin de se rapprocher des conditions réelles de fonctionnement, nous interagissons avec le climatiseur uniquement en imposant la température de consigne via la télécommande. Les résultats obtenus montrent que l'EER passe par un maximum pour un taux de charge compris entre 50\% et 100\% et que le SEER local s'éloigne de celui calculé selon la norme. Enfin, à température de consigne identique, on note une modification des performances obtenues lorsque le climatiseur fonctionne avec une forte humidité intérieure.

 \vfill doi : \url{https://doi.org/10.25855/SFT2021-042}

}
 
