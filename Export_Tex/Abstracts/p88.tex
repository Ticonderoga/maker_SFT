

    %=====================================
    %   WARNING
    %   FICHIER AUTOMATISE
    %   NE PAS MODIFIER
    %=====================================

    \newpage

\backgroundsetup{contents={Work In Progress},scale=7}
\BgThispage
%%%%%%%%%%%%%%%%%%%%%%%%%%%%%%%%%%%%%%%%%%%%
%% Papier 88
%%%%%%%%%%%%%%%%%%%%%%%%%%%%%%%%%%%%%%%%%%%%

% Indexations
\index{SaidouniWafaHafsa@Saidouni, Wafa Hafsa}
\index{HarelFabien@Harel, Fabien}
\index{BegotSylvie@Bégot, Sylvie}
\index{LepillerValerie@Lepiller, Valérie}
%
% Titre
\begin{flushleft}
\phantomsection\addtocounter{section}{1}
\addcontentsline{toc}{section}{Etude expérimentale du démarrage à froid de pile à combustible de type PEM}
{\Large \textbf{Etude expérimentale du démarrage à froid de pile à combustible de type PEM}}\label{ref:88}
\end{flushleft}
%
% Auteurs
Wafa Hafsa Saidouni$^{1,\star}$, Fabien Harel$^{2}$, Sylvie Bégot$^{1}$, Valérie Lepiller$^{1}$\\[2mm]
$^{\star}$ \Letter : \url{wafa_hafsa.saidouni@edu.univ-fcomte.fr}\\[2mm]
{\footnotesize $^{1}$ FEMTO-ST Institute, Univ. Bourgogne Franche-Comté, CNRS}\\
{\footnotesize $^{2}$ Université Gustave Eiffel}\\
[4mm]
%
% Mots clés
\noindent \textbf{Mots clés : } Pile à combustible, écoulements alternés\\[4mm]
%
% Résumé
\noindent \textbf{Résumé : } 

{\normalsize
La pile à combustible type PEM (Proton Exchange Membrane), également appelée pile à combustible à polymère solide, a été développée en 1960 par General Electric aux Etats-Unis afin qu’elle puisse être utilisée par la NASA pour les premiers véhicules spatiaux. Le cœur de la pile est formé par l’assemblage de deux électrodes et d’une membrane échangeuse de protons. L’alimentation en gaz et le circuit caloporteur sont assurés par des plaques dites plaques bipolaires. Le démarrage d’un véhicule équipé d’une pile à combustible doit pouvoir être assuré en température négative. Or, la pile à combustible de type PEM produit de l’eau. La problématique du démarrage à froid est due à la formation d’eau dans la pile, en effet en condition de gel, de la glace se forme et empêche le démarrage de la pile.

Le but de cette étude est d’utiliser la chaleur produite lors des réactions chimiques afin de pouvoir gérer la production de chaleur nécessaire qui permet d’assurer une montée en température suffisante du système afin d’éviter la formation du gel et sans faire appel à système de chauffage extérieur. La solution proposée pour cette problématique est l’utilisation d’un écoulement alterné. Un écoulement alterné est un écoulement instationnaire périodique alterné à vitesse moyenne nulle. Nous avons conçu un banc d’essai comprenant un système d’écoulement alterné composé d’une pompe et de deux vannes trois voies, l’instrumentation est assurée par un débitmètre Coriolis, un capteur de pression différentiel et des capteurs de température de type thermocouple. Le débit est réglable par une vanne manuelle.  Nous simulons la production de chaleur de la cellule par un film chauffant électrique de 100 W, sa puissance électrique est mesurée par un capteur de courant et un capteur de tension. La température du film chauffant est mesurée par une sonde PT100 et les températures des plaques bipolaires sont mesurées par des thermocouples. Le contrôle commande permet l’acquisition des mesures et le réglage de la fréquence de l’écoulement. La première campagne de mesures nous a permis d’obtenir l’évolution de la température de la pile à combustible lors du démarrage pour différentes fréquences d’oscillation, nous avons également obtenu l’évolution des pertes de charges en fonction des fréquences. Les prochaines campagnes de mesures nous permettrons de déterminer des corrélations expérimentales du coefficient de frottement et du coefficient d’échange en fonction de différentes fréquences. A l’aide de ces corrélations nous serons en mesure de proposer un système optimisé pour le démarrage d’un véhicule équipé d’une pile à combustible en température négative.

 \vfill Work In Progress

}
 
