

    %=====================================
    %   WARNING
    %   FICHIER AUTOMATISE
    %   NE PAS MODIFIER
    %=====================================

    \newpage

\backgroundsetup{contents={Work In Progress},scale=7}
\BgThispage
%%%%%%%%%%%%%%%%%%%%%%%%%%%%%%%%%%%%%%%%%%%%
%% Papier 45
%%%%%%%%%%%%%%%%%%%%%%%%%%%%%%%%%%%%%%%%%%%%

% Indexations
\index{XieBaoshan@Xie, Baoshan}
\index{LuoLingai@Luo, Lingai}
\index{BaudinNicolas@Baudin, Nicolas}
\index{SotoJerome@Soto, Jérôme}
\index{FanYilin@Fan, Yilin}
%
% Titre
\begin{flushleft}
\phantomsection\addtocounter{section}{1}
\addcontentsline{toc}{section}{Wall impact on a packed-bed thermal energy storage system efficiency}
{\Large \textbf{Wall impact on a packed-bed thermal energy storage system efficiency}}\label{ref:45}
\end{flushleft}
%
% Auteurs
Baoshan Xie$^{1}$, Lingai Luo$^{1}$, Nicolas Baudin$^{1,\star}$, Jérôme Soto$^{2}$, Yilin Fan$^{1}$\\[2mm]
$^{\star}$ \Letter : \url{nicolas.baudin@univ-nantes.fr}\\[2mm]
{\footnotesize $^{1}$  Laboratoire de Thermique et Energie de Nantes (LTéN), UMR CNRS 6607, Polytech' Nantes - Université de Nantes}\\
{\footnotesize $^{2}$ Laboratoire de Thermique et Energie de Nantes (LTéN), et Laboratoire     Énergétique Mécanique et Matériaux (ICAM)}\\
[4mm]
%
% Mots clés
\noindent \textbf{Mots clés : } Thermal energy storage, packed bed, thermocline, heat loss, wall impact\\[4mm]
%
% Résumé
\noindent \textbf{Résumé : } 

{\normalsize
Packed-bed energy storage single tank is a low-cost alternative to the conventional two-tank system for the concentrated solar power plant. In the single tank, both cold and hot heat transfer fluids are stored together and a thermal stratification is formed between two fluids by buoyancy force due to the different densities. This stratification region, called the thermocline, as an indicator of the thermal performance of the tank system, can be influenced by factors, such as, packing conditions, operating conditions, and tank geometry. Our target is to optimize influence parameters to increase both the heat storage capacity and exergy efficiency. A numerical model had first been built to study a packed-bed storage tank charging and discharging, validated against experiments. The next step will be to couple this model to optimisation algorithms to reach the target.







One of the novelty of the study is the investigation of the wall effect on the storage tank performances. Two configurations were compared: a small lab-scale storage tank, with a polycarbonate wall and nitrile rubber insulation, and an industrial-scale storage tank, with a steel wall and mineral wool insulation. A Schumann's model with temperature resolution in the wall and insulation was first used. It was found that although the influence of the wall on the industrial-scale tank was less important, the performances of the heat storage were dependant on the wall and insulation properties in both configurations. The transient heat storage in the wall can be up to 15\% of the heat provided to the tank and thus an additional capacitive term corresponding to the wall is to be added in the conventional Schuman's model. The wall heat losses are time dependant due to the transient conduction in the wall but using a steady state thermal resistance keeps the model error on exergy efficiency below 5\%. Finally, it was found that the wall axial conduction has a negligible effect on the thermocline thickness increase in the range of the operational parameters. These results are also validated against experimental data for the lab-scale setup.







Next step will be to verify experimentally the heat storage enhancement of a multi-layered packed-bed tank, filled with sensible material and phase change material, designed from the simulations beforehand.

 \vfill Work In Progress

}
 
