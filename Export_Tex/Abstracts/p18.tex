

    %=====================================
    %   WARNING
    %   FICHIER AUTOMATISE
    %   NE PAS MODIFIER
    %=====================================

    \newpage

%%%%%%%%%%%%%%%%%%%%%%%%%%%%%%%%%%%%%%%%%%%%
%% Papier 18
%%%%%%%%%%%%%%%%%%%%%%%%%%%%%%%%%%%%%%%%%%%%

% Indexations
\index{OlivesRegis@Olivès, Régis}
\index{TouzoAubin@Touzo, Aubin}
%
% Titre
\begin{flushleft}
\phantomsection\addtocounter{section}{1}
\addcontentsline{toc}{section}{Modélisation analytique d'un stockage thermocline pour la récupération de chaleur fatale ou pour centrale solaire à concentration}
{\Large \textbf{Modélisation analytique d'un stockage thermocline pour la récupération de chaleur fatale ou pour centrale solaire à concentration}}\label{ref:18}
\end{flushleft}
%
% Auteurs
Régis Olivès$^{1,\star}$, Aubin Touzo$^{2}$\\[2mm]
$^{\star}$ \Letter : \url{olives@univ-perp.fr}\\[2mm]
{\footnotesize $^{1}$ PROMES-CNRS - UPVD}\\
{\footnotesize $^{2}$ PROMES-CNRS}\\
[4mm]
%
% Mots clés
\noindent \textbf{Mots clés : } stockage thermique, thermocline, centrale solaire à concentration, chaleur fatale,\\[4mm]
%
% Résumé
\noindent \textbf{Résumé : } 

{\normalsize
Dans le cadre de la transition énergétique, le stockage thermique est un élément essentiel tant pour les centrales solaires à concentration que pour la récupération de chaleurs fatales. Le stockage thermique peut être intégré au réseau d'énergie afin de contribuer à un management de l'énergie plus efficient. Il peut permettre d'assurer une fourniture d'énergie contrôlée à partir de sources intermittentes telles que l'énergie solaire ou les chaleurs à haute température issues de procédés industriels. Parmi les technologies envisagées, le stockage de type thermocline avec un fluide de transfert qui traverse un lit de particules constitue un moyen intéressant qui a fait l'objet de nombreux travaux. Néanmoins, la modélisation du stockage thermocline peut être relativement lourde du fait de la nécessité de résoudre, bien souvent, un modèle monodimensionnel à deux températures tel que le modèle de Schumann. Afin d'intégrer le stockage thermocline dans un réseau d'énergie ou dans une centrale solaire et optimiser le fonctionnement nécessairement dynamique du stockage, il est nécessaire de pouvoir résoudre de façon performante un système à deux équations couplées, l'une pour le fluide et l'autre pour le solide. Certes, selon les hypothèses, une solution analytique existe. Mais lorsqu'il devient nécessaire de prendre en compte la diffusion thermique axiale, le modèle se complique et exige de mettre en œuvre une simulation numérique qui peut s'avérer longue en vue de l'optimisation. Dans un premier temps, nous proposons une méthode de résolution basée sur la double transformée de Laplace qui permet d'obtenir cette solution. Ainsi, la transformée est appliquée simultanément sur le temps et l'espace conduisant rapidement à l'obtention de la solution analytique dans le cas du modèle de Schumann. Cette méthode est mise en œuvre pour traiter le modèle qui tient compte de la diffusion thermique axiale. Dans un second temps, ces solutions sont comparées aux approximations obtenues par la modélisation du lit de particules par un ensemble de filtres en série appliqué à la température du fluide traversant le lit. Ainsi, une épaisseur donnée de lit de particules est assimilée à un filtre caractérisé par une résistance au transfert et une capacité thermique. Il s'agit alors d'appliquer la transformée de Laplace pour obtenir un modèle simple dont le temps caractéristique est lié aux résistances thermiques et à l'inertie thermique des filtres en série. A partir de ce modèle, il est possible de mener une analyse simplifiée conduisant à déterminer le comportement du stockage soumis à une entrée de chaleur fluctuante en température et/ou en débit. L'objectif est, à terme, de proposer un modèle de stockage thermocline rapide à calculer pour pouvoir non seulement faciliter le dimensionnement mais aussi mener une optimisation en régime dynamique pour l'intégration dans un réseau multi-énergies ou dans une centrale solaire à concentration.

 \vfill doi : \url{https://doi.org/10.25855/SFT2021-018}

}
 
