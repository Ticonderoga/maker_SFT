

    %=====================================
    %   WARNING
    %   FICHIER AUTOMATISE
    %   NE PAS MODIFIER
    %=====================================

    \newpage

%%%%%%%%%%%%%%%%%%%%%%%%%%%%%%%%%%%%%%%%%%%%
%% Papier 41
%%%%%%%%%%%%%%%%%%%%%%%%%%%%%%%%%%%%%%%%%%%%

% Indexations
\index{DellaliEmna@Dellali, Emna}
\index{LanzettaFrancois@Lanzetta, François}
\index{BegotSylvie@Bégot, Sylvie}
\index{RauchJean-Yves@Rauch, Jean-Yves}
%
% Titre
\begin{flushleft}
\phantomsection\addtocounter{section}{1}
\addcontentsline{toc}{section}{Optimisation d'un microéchangeur à partir d'un bilan entropique}
{\Large \textbf{Optimisation d'un microéchangeur à partir d'un bilan entropique}}\label{ref:41}
\end{flushleft}
%
% Auteurs
Emna Dellali$^{2}$, François Lanzetta$^{1,\star}$, Sylvie Bégot$^{1}$, Jean-Yves Rauch$^{1}$\\[2mm]
$^{\star}$ \Letter : \url{francois.lanzetta@univ-fcomte.fr}\\[2mm]
{\footnotesize $^{1}$ FEMTO-ST-UBFC-CNRS}\\
{\footnotesize $^{2}$ FEMTO-ST}\\
[4mm]
%
% Mots clés
\noindent \textbf{Mots clés : } entropie, micro-échangeur, écoulement, transfert de chaleur\\[4mm]
%
% Résumé
\noindent \textbf{Résumé : } 

{\normalsize
Le développement quasi exponentiel des MEMS (Micro-Electro-Mechanical-Systems) dans l'industrie permet d'accroître la densité de composants sur différents supports électroniques et informatiques. L'augmentation résultante des sources de chaleur nécessite alors d'intensifier les transferts thermiques dans le but de refroidir ces composants. Le développement des micro-systèmes tire profit de la technologie MEMS. En effet, à ces échelles miniatures, des procédés de micro-fabrication sont utilisés pour la réalisation des prototypes. Nous avons ainsi conçu un micro-échangeur à base d'assemblages de parois en verre et silicium constituants des canaux de dimensions micrométriques pouvant être traversés par un gaz tel que l'azote, l'argon ou l'hélium.



L'objectif de cet article est de déterminer les performances optimales d'un micro-échangeur. Les écoulements en microcanal avec transfert de chaleur sont le siège de transferts thermiques et de pertes de charge fonctions de conditions géométriques (diamètre hydraulique, longueur, surface) et thermofluidiques (masse volumique, viscosité, chaleur spécifique). La disparité dans les résultats obtenus dans la littérature concernant les micro-écoulements en régime permanent est essentiellement due aux effets de raréfaction, de rapport d'aspect du micro-canal et de la rugosité de la paroi. En effet, des études portées sur des micro-échangeurs thermiques montrent que les échanges thermiques sont nettement améliorés contre une augmentation de pertes de charge pour les micro-écoulements. Des études effectuées sur des micro-écoulements d'azote, d'hélium et d'argon montrent qu'en régime laminaire, pour des faibles débits massiques, le coefficient d'échange thermique convectif se trouve réduit par rapport à celui escompté par calculs, ce décalage étant plus notable pour des faibles débits. Il est également souligné l'importance de la conduction axiale dans les parois et sa prépondérance par rapport à celle dans le fluide ce qui favorise le mélange et diminue ainsi l'efficacité du micro-échangeur. 



Un bilan entropique permettra de distinguer, sous la forme d'un facteur adimensionnel de production d'entropie, la part entre les irréversibilités fluidiques et thermiques en fonction du débit de gaz, du gradient de température entre le gaz et les parois, du coefficient de frottement et du nombre de Reynolds.

 \vfill doi : \url{https://doi.org/10.25855/SFT2021-041}

}
 
