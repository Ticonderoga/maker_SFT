

    %=====================================
    %   WARNING
    %   FICHIER AUTOMATISE
    %   NE PAS MODIFIER
    %=====================================

    \newpage

\backgroundsetup{contents={Work In Progress},scale=7}
\BgThispage
%%%%%%%%%%%%%%%%%%%%%%%%%%%%%%%%%%%%%%%%%%%%
%% Papier 70
%%%%%%%%%%%%%%%%%%%%%%%%%%%%%%%%%%%%%%%%%%%%

% Indexations
\index{LetessierJordan@Letessier, Jordan}
\index{GardareinJean-Laurent@Gardarein, Jean-Laurent}
\index{VansonJean-Mathieu@Vanson, Jean-Mathieu}
\index{RigolletFabrice@Rigollet, Fabrice}
\index{DuguayChristelle@Duguay, Christelle}
\index{MassonRenaud@Masson, Renaud}
%
% Titre
\begin{flushleft}
\phantomsection\addtocounter{section}{1}
\addcontentsline{toc}{section}{Évolution de la conductivité thermique d'un lit de billes de plomb en fonction du chargement mécanique : Mesures et confrontation à la littérature}
{\Large \textbf{Évolution de la conductivité thermique d'un lit de billes de plomb en fonction du chargement mécanique : Mesures et confrontation à la littérature}}\label{ref:70}
\end{flushleft}
%
% Auteurs
Jordan Letessier$^{1}$, Jean-Laurent Gardarein$^{1,\star}$, Jean-Mathieu Vanson$^{2}$, Fabrice Rigollet$^{1}$, Christelle Duguay$^{2}$, Renaud Masson$^{2}$\\[2mm]
$^{\star}$ \Letter : \url{jean-laurent.gardarein@univ-amu.fr}\\[2mm]
{\footnotesize $^{1}$ Aix Marseille Univ, CNRS, IUSTI, Marseille}\\
{\footnotesize $^{2}$ CEA, DES, IRESNE, DEC, Cadarache}\\
[4mm]
%
% Mots clés
\noindent \textbf{Mots clés : } conductivité thermique équivalente, milieu granulaire, méthode inverse, propriétés thermomécaniques\\[4mm]
%
% Résumé
\noindent \textbf{Résumé : } 

{\normalsize
Dans les milieux granulaires l'estimation des propriétés thermiques équivalentes est un exercice complexe qui reste largement discuté dans la littérature. Les résultats donnés par les modèles usuellement employés souffrent d'une dispersion importante liée notamment à la méconnaissance de la microstructure à l'échelle locale, à l'empilement granulaire et aux transferts de chaleur aux interfaces.







Dans ce papier, nous proposons une méthode de mesure transitoire des propriétés thermiques d'un milieu granulaire constitué d'un lit de billes de plomb de diamètre moyen 2.8 mm. Un plan chaud permet d'imposer un flux de chaleur pilotable électriquement en face avant du lit de particules.  La température du milieu granulaire est mesurée aux deux extrémités du lit, ce qui permet notamment de s'affranchir de la forme temporelle exacte du flux de chaleur imposé en face avant. On modélise le lit comme un milieu homogène équivalent, et on utilise une résolution du problème par méthode inverse pour estimer les caractéristiques thermiques du milieu équivalent. Connaissant l'effusivité du porte échantillon semi-infini à l'arrière du lit, cette méthode permet d'estimer la diffusivité et l'effusivité du milieu, la conductivité thermique est déduite par la suite. La taille des surfaces d'échanges entre les particules peut varier  en imposant un chargement mécanique sur une des faces du lit. De cette manière nous évaluons l'influence des contacts thermiques entre les particules. Deux approches différentes peuvent servir à évaluer cette surface d'échange entre particules. La première utilise la théorie de contact de Hertz entre sphères. Cette méthode suppose que les billes sont sphériques et que leur déformation est élastique. L'autre approche est de définir le contact par la théorie de Greenwood et Williamson. Cette méthode a pour avantage de prendre en compte l'état de surface  des billes et par conséquent d'être plus réaliste.







Parmi les modèles connus de la littérature, Imura et Takegoshi définissent la conductivité équivalente d'un milieu granulaire, en modélisant les effets simultanés de trois mécanismes de transfert indépendants. Premièrement, le transfert par conduction et par rayonnement dans le gaz, puis, le transfert thermique dans le solide pour les contacts entre particules, et enfin le transfert thermique en série entre le solide et la fine couche de gaz présente autour des contacts entre particules. Dans ce papier, ce modèle est adapté en décrivant tous ces paramètres à l'aide des propriétés thermomécaniques du solide, et du gaz, en particulier en prenant en compte la rugosité des billes de plomb mesurée par ailleurs à l'aide d'un microscope confocal. 



L'évolution de la conductivité thermique équivalente, calculée avec ce nouveau modèle, en fonction du chargement mécanique est comparée aux résultats obtenus avec l'expérience. Les premiers résultats donnent de meilleurs accords avec la définition du contact par la théorie de Greenwood et Williamson.



 \vfill Work In Progress

}
 
