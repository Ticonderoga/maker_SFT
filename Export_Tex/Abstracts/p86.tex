

    %=====================================
    %   WARNING
    %   FICHIER AUTOMATISE
    %   NE PAS MODIFIER
    %=====================================

    \newpage

\backgroundsetup{contents={Work In Progress},scale=7}
\BgThispage
%%%%%%%%%%%%%%%%%%%%%%%%%%%%%%%%%%%%%%%%%%%%
%% Papier 86
%%%%%%%%%%%%%%%%%%%%%%%%%%%%%%%%%%%%%%%%%%%%

% Indexations
\index{LiemansBenoit@Liemans, Benoit}
\index{FeldheimVeronique@Feldheim, Véronique}
\index{RusseilSerge@Russeil, Serge}
\index{BougeardDaniel@Bougeard, Daniel}
%
% Titre
\begin{flushleft}
\phantomsection\addtocounter{section}{1}
\addcontentsline{toc}{section}{Etude de l'amélioration d'un panneau aérovoltaïque installé sur un bâtiment à faibles besoins énergétiques}
{\Large \textbf{Etude de l'amélioration d'un panneau aérovoltaïque installé sur un bâtiment à faibles besoins énergétiques}}\label{ref:86}
\end{flushleft}
%
% Auteurs
Benoit Liemans$^{1,\star}$, Véronique Feldheim$^{1}$, Serge Russeil$^{2}$, Daniel Bougeard$^{2}$\\[2mm]
$^{\star}$ \Letter : \url{benoit.liemans@umons.ac.be}\\[2mm]
{\footnotesize $^{1}$ UMONS – FPMs – Thermique et Combustion}\\
{\footnotesize $^{2}$ IMT Lille Douai - CERI Energie Environnement}\\
[4mm]
%
% Mots clés
\noindent \textbf{Mots clés : } photovoltaïque convection modèle simplifié\\[4mm]
%
% Résumé
\noindent \textbf{Résumé : } 

{\normalsize
Les panneaux photovoltaïques convertissent le flux solaire reçu en électricité avec un rendement de référence de l'ordre de 17\% pour les technologies répandues à base de cellules de silicium monocristallin. Malheureusement, ce rendement se dégrade dès lors que la température de ces dernières dépassent les 25$\dC$. Un panneau en pose libre ne peut se refroidir que majoritairement par convection naturelle et légèrement par rayonnement avec respectivement l'air ambiant et son environnement. Le problème connu est que généralement lorsque le flux solaire est important, la température de l'air extérieur est également plus élevée et le refroidissement est donc plus difficile. Dans la pratique, la situation est encore plus compliquée lorsque le panneau est posé sur une paroi de bâtiment, comme en toiture par exemple.



Afin de solutionner ce problème, de nombreux systèmes ont été étudiés comme le recours à la convection forcée de l'air, l'utilisation d'un fluide circulant dans un serpentin à l'arrière du panneau ou encore l'ajout d'ailettes mais rarement concernant l'amélioration possible du coefficient d'échange thermique par convection avec l'air ambiant autrement que par que par le passage en convection forcée par le biais de ventilateurs. 



Une étude récente de A. Khanjian a montré que l'utilisation d'ailettes génératrices de vortex permettait sous certaines conditions une amélioration du nombre de Nusselt de l'ordre de 40\% pour les transferts de chaleur dans un conduit de section rectangulaire. L'avantage de ce type d'élément est de ne nécessiter aucune source d'énergie supplémentaire, d'exiger peu de matière et de réduire au mieux les pertes de charges engendrées.



Dans le cadre de notre travail, nous allons étudier un panneau photovoltaïque modifié, à l'arrière duquel une structure de canal va être équipée d'ailettes génératrices de vortex. Le refroidissement du panneau PV va être réalisé grâce à l'air extérieur ambiant, dans un système éventuellement lié à la ventilation mécanique du bâtiment. 



Les premiers résultats obtenus concernent l'étude numérique dynamique du pavillon expérimental sur base des plans d'exécution mais également l'établissement d'un modèle simplifié du panneau équipé de sa structure additionnelle destiné à être affiné à partir des résultats de futures études CFD visant à optimiser le nombre et la répartition des générateurs de vortex.



Nous pouvons déjà constater d'une part que le pavillon a de faibles besoins en chauffage et d'autre part qu'il est sujet à un risque de surchauffe estivale relativement élevé. Le modèle simplifié du panneau amélioré avec les générateurs de vortex nous permet d'envisager à ce stade un gain probable de l'ordre de 30\% sur les gains thermiques.

 \vfill Work In Progress

}
 
