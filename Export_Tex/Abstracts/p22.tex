

    %=====================================
    %   WARNING
    %   FICHIER AUTOMATISE
    %   NE PAS MODIFIER
    %=====================================

    \newpage

\backgroundsetup{contents={Work In Progress},scale=7}
\BgThispage
%%%%%%%%%%%%%%%%%%%%%%%%%%%%%%%%%%%%%%%%%%%%
%% Papier 22
%%%%%%%%%%%%%%%%%%%%%%%%%%%%%%%%%%%%%%%%%%%%

% Indexations
\index{TadristLoic@Tadrist, Loïc}
\index{Darbois-TexierBaptiste@Darbois-Texier, Baptiste}
\index{LanzettaFrancois@Lanzetta, François}
\index{TadristLounes@Tadrist, Lounès}
%
% Titre
\begin{flushleft}
\phantomsection\addtocounter{section}{1}
\addcontentsline{toc}{section}{Thermodynamique des chocs d'une membrane pressurisée : le cas des ballons de sport.}
{\Large \textbf{Thermodynamique des chocs d'une membrane pressurisée : le cas des ballons de sport.}}\label{ref:22}
\end{flushleft}
%
% Auteurs
Loïc Tadrist$^{1,\star}$, Baptiste Darbois-Texier$^{2}$, François Lanzetta$^{3}$, Lounès Tadrist$^{4,\star}$\\[2mm]
$^{\star}$ \Letter : \url{loic.tadrist@univ-amu.fr}\\[2mm]
{\footnotesize $^{1}$ ISM, Aix-Marseille Université}\\
{\footnotesize $^{2}$ FAST, Université Paris-Saclay}\\
{\footnotesize $^{3}$ FEMTO-ST, Université de Franche Comté}\\
{\footnotesize $^{4}$ IUSTI, Aix-Marseille Université}\\
[4mm]
%
% Mots clés
\noindent \textbf{Mots clés : } Thermique rapide, Capteur embarqués, Interactions fluide-solide\\[4mm]
%
% Résumé
\noindent \textbf{Résumé : } 

{\normalsize
Les membranes pressurisées sont utilisées dans diverses situations impliquant l'homme (protections, sports, etc.). Dans ce contexte, la mécanique des impacts des membranes est critique car sous ou sur gonflées,



elles peuvent provoquer des blessures.







Les effets de la pression de gonflage et de la vitesse incidente sur le temps de contact sont étudiés sur une expérience modèle en considérant une membrane sphérique. Des essais expérimentaux ont été conduits sur une membrane industrielle en caoutchouc avec différents gaz de gonflage (Hélium, Argon, -- mono-atomiques et Air -- diatomique). Ces essais mettent en évidence une dépendance du temps de contact et du coefficient de restitution avec la pression et la vitesse d'impact. La dépendance du temps de contact avec la vitesse du choc est la marque d'un choc non-linéaire.



Pour comprendre la source de cette non-linéarité, des mesures synchronisées rapides de la pression, de la température du gaz (micro-thermocouples type K) et de l'indentation de la membrane lors du choc ont été réalisées. Ces mesures  montrent que sur le temps du choc, de l'ordre de quelques dizaines de millisecondes, a lieu une compression-détente adiabatique en interaction forte avec la membrane. Les variations de température enregistrées montrent des variations de température de plusieurs Kelvins à chaque rebond de la membrane pressurisée.







Une modélisation par équations différentielles du choc permet de montrer que le temps de contact d'une membrane gonflée est décrit par trois nombres sans dimension, (i) la pression de gonflage adimensionnée, (ii) le nombre presso-élastique qui considère l'interaction gaz-membrane et (iii) le nombre presso-inertiel qui compare l'inertie de la membrane aux forces de pression. Ce modèle est comparé aux résultats expérimentaux et permet de quantifier la dépendance du temps de contact avec la vitesse d'impact.







Ces résultats couplés thermomécaniques rapides permettent de comprendre le comportement non-linéaire d'un ballon de sport lors d'un choc avec des préconisations de pressurisation pour éviter les blessures.

 \vfill Work In Progress

}
 
