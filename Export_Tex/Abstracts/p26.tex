

    %=====================================
    %   WARNING
    %   FICHIER AUTOMATISE
    %   NE PAS MODIFIER
    %=====================================

    \newpage

%%%%%%%%%%%%%%%%%%%%%%%%%%%%%%%%%%%%%%%%%%%%
%% Papier 26
%%%%%%%%%%%%%%%%%%%%%%%%%%%%%%%%%%%%%%%%%%%%

% Indexations
\index{GaumeBenjamin@Gaume, Benjamin}
\index{RouiziYassine@Rouizi, Yassine}
\index{JolyFrederic@Joly, Frédéric}
\index{QuemenerOlivier@Quemener, Olivier}
%
% Titre
\begin{flushleft}
\phantomsection\addtocounter{section}{1}
\addcontentsline{toc}{section}{Mesure thermique indirecte en temps réel dans un four radiant par modèle réduit}
{\Large \textbf{Mesure thermique indirecte en temps réel dans un four radiant par modèle réduit}}\label{ref:26}
\end{flushleft}
%
% Auteurs
Benjamin Gaume$^{1,\star}$, Yassine Rouizi$^{1}$, Frédéric Joly$^{1}$, Olivier Quemener$^{1}$\\[2mm]
$^{\star}$ \Letter : \url{b.gaume@iut.univ-evry.fr}\\[2mm]
{\footnotesize $^{1}$ LMEE, Univ Evry, Université Paris-Saclay, 91020, Evry, France.}\\
[4mm]
%
% Mots clés
\noindent \textbf{Mots clés : } Problème inverse; Rayonnement thermique; Réduction de modèle;\\[4mm]
%
% Résumé
\noindent \textbf{Résumé : } 

{\normalsize
La réduction de modèle par la méthode AROMM (Amalgam Reduced Order Modal Model) permet une forte réduction du temps de simulation tout en permettant un accès à l'ensemble de la scène thermique avec une prise en compte des phénomènes radiatifs.  D'autre part, des travaux sur l'identification avec cette même méthode ont montré leur intérêt pour l'identification de flux de chaleur sur un disque de frein. 







Dans la continuité de ces travaux, nous proposons ici d'utiliser la méthode AROMM, afin (i) d'identifier une source radiative variable dans le temps à partir de quelques points de mesures avec un modèle d'ordre faible, (ii) de reconstruire la scène thermique complète avec un modèle réduit d'un ordre plus important.







L'exemple traité ici porte sur un four industriel dans lequel une pièce en titane est chauffée par deux tubes radiants ayant une température variable dans le temps comprise entre 293 et 1173K. Les données d'entrées pour l'identification sont les températures de deux points de mesure sur le four, caractérisées par un bruit blanc de 2K. A partir d'un modèle réduit d'ordre 20, la source rayonnante est identifiée en temps réel sur une fenêtre de temps glissante via une procédure de minimisation utilisant l'algorithme de région de confiance. Il est alors possible de reconstruire toujours en temps réel le champ de température complet sur l'intégralité de la pièce en titane à l'aide d'un modèle réduit d'ordre 150. Le champ erreur obtenu se distingue par une erreur moyenne inférieure à 2K et une erreur maximum ponctuelle en temps et en espace inférieure à 40K.


 \vfill doi : \url{https://doi.org/10.25855/SFT2021-026}

}
 
