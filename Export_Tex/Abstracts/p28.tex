

    %=====================================
    %   WARNING
    %   FICHIER AUTOMATISE
    %   NE PAS MODIFIER
    %=====================================

    \newpage

%%%%%%%%%%%%%%%%%%%%%%%%%%%%%%%%%%%%%%%%%%%%
%% Papier 28
%%%%%%%%%%%%%%%%%%%%%%%%%%%%%%%%%%%%%%%%%%%%

% Indexations
\index{FuentesAdrien@Fuentes, Adrien}
\index{GlouannecPatrick@Glouannec, Patrick}
\index{NoelHerve@Noël, Hervé}
%
% Titre
\begin{flushleft}
\phantomsection\addtocounter{section}{1}
\addcontentsline{toc}{section}{Conception de parois multi-couches pour véhicules utilitaires isothermes}
{\Large \textbf{Conception de parois multi-couches pour véhicules utilitaires isothermes}}\label{ref:28}
\end{flushleft}
%
% Auteurs
Adrien Fuentes$^{1,\star}$, Patrick Glouannec$^{1}$, Hervé Noël$^{1}$\\[2mm]
$^{\star}$ \Letter : \url{adrien.fuentes@univ-ubs.fr}\\[2mm]
{\footnotesize $^{1}$ Université de Bretagne Sud, IRDL, UMR CNRS 6027}\\
[4mm]
%
% Mots clés
\noindent \textbf{Mots clés : } Transfert de chaleur, parois isolante, caractérisation, expérimentation, modélisation\\[4mm]
%
% Résumé
\noindent \textbf{Résumé : } 

{\normalsize
L'exploitation de véhicules électriques pour la livraison urbaine de denrées périssables est appelée à fortement se développer dans les années à venir. Ce moyen de livraison permet de réduire les émissions carbonées liées au transport de produits alimentaires réfrigérées. 



Pour que ce scenario soit viable, il devient nécessaire de limiter l'usage du groupe frigorifique embarqué et de privilégier l'utilisation du stockage électrochimique (batteries) pour la motorisation du véhicule.



Dans le cadre de ces travaux, on s'intéresse à l'isolation thermique de véhicules de petits volumes, essentiellement destinés au transport urbain de produits alimentaires réfrigérés pendant une durée de quelques heures.



L'objet de cette communication est de présenter des études expérimentales et numériques destinées à la conception optimale de parois isolantes ne pénalisant pas la masse et le volume utile du véhicule. Dans ce travail, l'augmentation la capacité thermique effective de la paroi à l'aide de Matériaux à Changement de Phase (MCP) est également étudiée. Ces matériaux sous forme de plaques planes de mousse de polyuréthane (PU) contenant différents taux de MCP, ont notamment été fabriquées pour cette étude.



Cette communication s'initie avec la présentation des résultats d'une campagne de caractérisation des propriétés thermophysiques des matériaux employés dans ce travail, et se poursuit avec la présentation du dispositif expérimental permettant de tester une composition de paroi définie sous différentes sollicitations thermiques.



Le modèle numérique est ensuite présenté et confronté à une séquence expérimentale afin de valider la pertinence de celui-ci. Les résultats montrent une bonne adéquation entre le modèle et l'expérience, en particulier pour la prise en compte des changements d'état du MCP. Pour finir, ce modèle est utilisé dans une phase de prospection dont l'objectif est de déterminer une configuration de paroi optimale pour cette application, avec notamment la définition de l'épaisseur et de l'emplacement optimal du MCP dans cette paroi.

 \vfill doi : \url{https://doi.org/10.25855/SFT2021-028}

}
 
