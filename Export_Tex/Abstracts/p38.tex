

    %=====================================
    %   WARNING
    %   FICHIER AUTOMATISE
    %   NE PAS MODIFIER
    %=====================================

    \newpage

%%%%%%%%%%%%%%%%%%%%%%%%%%%%%%%%%%%%%%%%%%%%
%% Papier 38
%%%%%%%%%%%%%%%%%%%%%%%%%%%%%%%%%%%%%%%%%%%%

% Indexations
\index{MaksassiZiad@Maksassi, Ziad}
\index{GarnierBertrand@Garnier, Bertrand}
\index{OuldElMoctarAhmed@Ould El Moctar, Ahmed}
\index{SchoefsFranck@Schoefs, Franck}
%
% Titre
\begin{flushleft}
\phantomsection\addtocounter{section}{1}
\addcontentsline{toc}{section}{Caractérisation thermique du biofouling autour d'un câble électrique dynamique sous-marin}
{\Large \textbf{Caractérisation thermique du biofouling autour d'un câble électrique dynamique sous-marin}}\label{ref:38}
\end{flushleft}
%
% Auteurs
Ziad Maksassi$^{1,\star}$, Bertrand Garnier$^{2}$, Ahmed Ould El Moctar$^{3}$, Franck Schoefs$^{4}$\\[2mm]
$^{\star}$ \Letter : \url{Ziad.Maksassi@univ-nantes.fr}\\[2mm]
{\footnotesize $^{1}$ Université de Nantes, Laboratoire de thermique et énergie de Nantes, LTeN, UMR CNRS 6607 et Institut de Recherche en Génie Civil et Mécanique (GeM), UMR CNRS 6183}\\
{\footnotesize $^{2}$ CNRS, Laboratoire de thermique et énergie de Nantes, LTeN, UMR 6607}\\
{\footnotesize $^{3}$ Université de Nantes, Laboratoire de thermique et énergie de Nantes, LTeN, UMR CNRS 6607}\\
{\footnotesize $^{4}$ Université de Nantes, Institut de Recherche en Génie Civil et Mécanique (GeM), UMR CNRS 6183}\\
[4mm]
%
% Mots clés
\noindent \textbf{Mots clés : } Energie Marine Renouvelable (EMR), câble électrique dynamique, biofouling, analyse thermique, caractérisation thermique de biofouling\\[4mm]
%
% Résumé
\noindent \textbf{Résumé : } 

{\normalsize
Dans le monde entier, les projets d'éoliennes flottantes, à l'échelle du prototype, de la ferme pilote ou de la ferme commerciale sont en cours de développement. Un de leurs composants clés est le câble dynamique qui permettent la connexion électrique. Le système d'isolation électrique (polyéthylène réticulé) du câble dynamique haute tension sous-marin est conçu pour supporter une température maximale de 90~$\dC$. Cependant, la croissance de biofouling, en particulier les moules, peut modifier le transfert de chaleur autour du câble, ce qui pourrait entraîner une diminution ou une augmentation de la température du câble Dans ce travail, la conductivité thermique effective des différents types de moules (juvéniles, mixtes (juvéniles et adultes) et adultes) est mesurée, ainsi que le coefficient d'échange autour des moules. La conductivité thermique effective des moules est estimée à l'aide du modèle stationnaire analytique 1D (loi de Fourier) valable pour une distribution uniforme des moules autour du tube. La méthode de mesure est validée en mesurant également la conductivité thermique d'une mousse adhésive double face et par comparaison avec une mesure à partir d'un dispositif à plaque chauffante. De plus, le coefficient de transfert de chaleur de l'eau autour des moules est également calculé en utilisant (loi de Newton). En outre, il est comparé à deux corrélations de la littérature (Churchill \& Chu et Morgan). De plus, on a également considéré des distributions non uniformes des moules autour du tube en effet en pratique la croissance des moules se produit sous la mer principalement sur le dessus du câble électrique horizontal puisque la lumière vient d'en haut. Dans ce cas, en raison d'une géométrie plus compliquée, la croissance de la conductivité thermique effective de différents types de moules et le coefficient de transfert de chaleur de l'eau autour des moules sont estimés à l'aide de la méthode numérique (éléments finis via COMSOL) pour résoudre l'équation de transfert de chaleur 2D et d'une technique d'estimation des paramètres (méthode simplex) pour obtenir la conductivité thermique effective de moules. Nous obtenons que les moules juvéniles ont la plus petite conductivité thermique effective par rapport au mélange (juvénile et adulte) et aux moules adultes respectivement. En fait, une petite circulation d'eau entre les moules est attendue dans l'espace poreux, ce qui conduit à une augmentation de la conductivité thermique effective (cette hypothèse est confirmée en mesurant la conductivité thermique d'un matériau poreux constitué d'un amas de billes de verre). Par conséquent, comme les moules juvéniles ont la plus faible porosité de l'eau par rapport aux moules mixtes et adultes, elles ont donc la plus petite conductivité thermique effective. Ainsi, cette caractérisation thermique est une étude très intéressante, car elle permet à l'avenir de créer un modèle numérique pour vérifier si la température du conducteur dépasse sa température maximale (90~$\dC$) pour un fonctionnement continu ou non, en d'autres termes, on peut vérifier l'effet thermique de différents types de moules avec des épaisseurs différentes sur l'endommagement des câbles électriques dynamiques.

 \vfill doi : \url{https://doi.org/10.25855/SFT2021-038}

}
 
