

    %=====================================
    %   WARNING
    %   FICHIER AUTOMATISE
    %   NE PAS MODIFIER
    %=====================================

    \newpage

%%%%%%%%%%%%%%%%%%%%%%%%%%%%%%%%%%%%%%%%%%%%
%% Papier 57
%%%%%%%%%%%%%%%%%%%%%%%%%%%%%%%%%%%%%%%%%%%%

% Indexations
\index{MantaropoulosPatric@Mantaropoulos, Patric}
\index{GiraudFlorine@Giraud, Florine}
\index{TremeacBrice@Tréméac, Brice}
\index{TobalyPascal@Tobaly, Pascal}
%
% Titre
\begin{flushleft}
\phantomsection\addtocounter{section}{1}
\addcontentsline{toc}{section}{Bulles d'eau à pression sub-atmosphérique : étude expérimentale et analyse dimensionnelle dans un canal vertical confiné}
{\Large \textbf{Bulles d'eau à pression sub-atmosphérique : étude expérimentale et analyse dimensionnelle dans un canal vertical confiné}}\label{ref:57}
\end{flushleft}
%
% Auteurs
Patric Mantaropoulos$^{1,\star}$, Florine Giraud$^{1}$, Brice Tréméac$^{1}$, Pascal Tobaly$^{1}$\\[2mm]
$^{\star}$ \Letter : \url{patric.mantaropoulos@lecnam.net}\\[2mm]
{\footnotesize $^{1}$ Lafset, Conservatoire National des Arts et Métiers}\\
[4mm]
%
% Mots clés
\noindent \textbf{Mots clés : } bulles, r718, subatmosphérique, évaporateur à plaques\\[4mm]
%
% Résumé
\noindent \textbf{Résumé : } 

{\normalsize
Nombre de fluides de travail communément utilisés aujourd'hui pour la réfrigération sont soumis à des réglementations strictes, qui deviendront de plus en plus contraignantes dans le futur. En effet, l'impact environnemental de tels fluides, évalué par le PDO (Potentiel de Destruction de l'Ozone) et le PRG (Pouvoir de Réchauffement Global), est significatif. Dans ce contexte, la recherche de fluides alternatifs constitue un des efforts à fournir pour la transition énergétique. Parmi ces fluides alternatifs, l'eau apparaît comme un excellent candidat : naturel, sans impact environnemental et dont certaines propriétés thermo-physiques sont avantageuses . Néanmoins, l'utilisation de l'eau comme fluide frigorigène impose généralement de travailler à des pressions très faibles (de l'ordre du kPa). Or, bien que la dynamique de croissance de bulles à pression atmosphérique ait été étudiée pendant plus d'un siècle, les modèles développés semblent inadaptés pour décrire les phénomènes intervenant dans des conditions sub-atmosphériques. En effet, il a été mis en évidence que, dans ces conditions, des bulles de taille centimétrique se développent. Dans le cas particulier des échangeurs thermiques diphasiques, une étude expérimentale sur un canal d'évaporateur à plaques lisses a mis en lumière que la croissance et l'éclatement des bulles de vapeur forment un film liquide qui, par son évaporation, contribue en majorité aux échanges thermiques. Dans une optique d'optimisation de l'efficacité énergétique, il semble donc nécessaire d'avoir une meilleure compréhension de la formation de ce film liquide, et donc d'étudier le comportement des bulles le formant. Cette étude en est l'objectif. À partir d'une campagne expérimentale qui a permis de rassembler des enregistrements par caméra rapide dans une gamme de pressions allant de 10 à 50 mbar, plusieurs résultats ont pu être mis en évidence. Une première corrélation entre la hauteur de projection des gouttelettes et les caractéristiques physiques de la bulle (taille et vitesse) avait été précédemment établie (SFT2020). Depuis, afin d'affiner l'étude, une analyse dimensionnelle a été réalisée et l'impact de certains nombres adimensionnels sur la hauteur de projection des gouttelettes a été évalué. Il apparaît notamment que la hauteur de projection est corrélée aux nombres de Bond et de Weber.

 \vfill doi : \url{https://doi.org/10.25855/SFT2021-057}

}
 
