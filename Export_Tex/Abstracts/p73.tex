

    %=====================================
    %   WARNING
    %   FICHIER AUTOMATISE
    %   NE PAS MODIFIER
    %=====================================

    \newpage

%%%%%%%%%%%%%%%%%%%%%%%%%%%%%%%%%%%%%%%%%%%%
%% Papier 73
%%%%%%%%%%%%%%%%%%%%%%%%%%%%%%%%%%%%%%%%%%%%

% Indexations
\index{DeLarochelambertThierry@De Larochelambert, Thierry}
\index{BaillyYannick@Bailly, Yannick}
\index{GiurgeaStefan@Giurgea, Stefan}
\index{RoyJean-Claude@Roy, Jean-Claude}
\index{RamelDavid@Ramel, David}
\index{GirardotLaurent@Girardot, Laurent}
\index{GlisesRaynal@Glises, Raynal}
\index{DubasFrederic@Dubas, Frédéric}
\index{BarriereThierry@Barriere, Thierry}
\index{HirsingerLaurent@Hirsinger, Laurent}
\index{PlaitAntony@Plait, Antony}
\index{PerrinMickael@Perrin, Mickaël}
\index{IsmailAli@Ismail, Ali}
\index{MeunierAlexandre@Meunier, Alexandre}
\index{LiTianjiao@Li, Tianjiao}
\index{EustacheJulien@Eustache, Julien}
%
% Titre
\begin{flushleft}
\phantomsection\addtocounter{section}{1}
\addcontentsline{toc}{section}{La réfrigération magnétocalorique au défi du réchauffement climatique}
{\Large \textbf{La réfrigération magnétocalorique au défi du réchauffement climatique}}\label{ref:73}
\end{flushleft}
%
% Auteurs
Thierry De Larochelambert$^{1,\star}$, Yannick Bailly$^{1}$, Stefan Giurgea$^{2}$, Jean-Claude Roy$^{1}$, David Ramel$^{1}$, Laurent Girardot$^{1}$, Raynal Glises$^{1}$, Frédéric Dubas$^{1}$, Thierry Barriere$^{3}$, Laurent Hirsinger$^{4}$, Antony Plait$^{2}$, Mickaël Perrin$^{1}$, Ali Ismail$^{1}$, Alexandre Meunier$^{5}$, Tianjiao Li$^{5}$, Julien Eustache$^{1}$\\[2mm]
$^{\star}$ \Letter : \url{thierry.larochelambert@femto-st.fr}\\[2mm]
{\footnotesize $^{1}$ Institut FEMTO-ST (CNRS-UMR6174), Département Energie}\\
{\footnotesize $^{2}$ UTBM}\\
{\footnotesize $^{3}$ Institut FEMTO-ST (CNRS-UMR6174), Département Mécanique Appliquée}\\
{\footnotesize $^{4}$ Institut FEMTO-ST (CNRS-UMR6174), Département MN2S}\\
{\footnotesize $^{5}$ NextPac}\\
[4mm]
%
% Mots clés
\noindent \textbf{Mots clés : } magnétocalorique, réfrigération, régénérateurs, efficacité, matériaux magnétocaloriques, optimisation, échangeur, transfert thermique\\[4mm]
%
% Résumé
\noindent \textbf{Résumé : } 

{\normalsize
Au cours des vingt dernières années, de nombreuses recherches ont été menées dans le monde pour développer des technologies de réfrigération magnétocalorique prometteuses car plus efficaces et sans émissions de GES. Elles ont permis de bien comprendre les mécanismes de l’effet magnétocalorique (EMC) présentés par de nombreux matériaux, de fabriquer de nouveaux alliages susceptibles d’une utilisation massive pour la réfrigération et la climatisation au cours des prochaines années. Cependant, certains verrous scientifiques et techniques doivent encore être levés avant d’aboutir à des dispositifs magnétocaloriques suffisamment performants pour atteindre le stade industriel et devenir concurrentiels vis-à-vis des technologies classiques de réfrigération à compression de vapeur.

Après une synthèse des récents progrès dans l’élaboration des matériaux et la mise au point de prototypes magnétocaloriques, cet article présente les défis scientifiques propres à ces machines et leurs solutions possibles. Il décrit les travaux menés au Dpt Énergie de l’Institut FEMTO-ST pour optimiser la conception des régénérateurs magnétocaloriques au cœur de ces machines.

Un de ces axes de recherche porte sur l’intensification des échanges thermiques entre les plaques de matériaux des régénérateurs magnétocaloriques et l’écoulement alterné, à des fréquences plus élevées que celles habituellement utilisées dans les dispositifs actuels. Les calculs analytiques montrent que qu’elle se produit au-delà d’un seuil critique $Re_{\omega c}$ du nombre adimensionnel de Womersley, caractérisant l’apparition de l’effet annulaire induit dans ces écoulements.

L’utilisation de régénérateurs magnétocaloriques multicouches à matériaux étagés de températures de Curie croissantes est un autre axe de recherche dont les simulations numériques démontrent l’augmentation de l’écart de température produit aux extrémités lors des cycles actifs de réfrigération magnétique. D’autres travaux en cours (microstructuration, composites) peuvent conduire aux ruptures technologiques recherchées permettant atteindre l’efficacité et les baisses de coûts indispensables à la commercialisation des futures machines magnétocaloriques.
 \vfill doi : \url{https://doi.org/10.25855/SFT2021-073}

}
 
