

    %=====================================
    %   WARNING
    %   FICHIER AUTOMATISE
    %   NE PAS MODIFIER
    %=====================================

    \newpage

%%%%%%%%%%%%%%%%%%%%%%%%%%%%%%%%%%%%%%%%%%%%
%% Papier 15
%%%%%%%%%%%%%%%%%%%%%%%%%%%%%%%%%%%%%%%%%%%%

% Indexations
\index{MalagoniMarina@Malagoni, Marina}
\index{GallegoSandra@Gallego, Sandra}
\index{GinestetStephane@Ginestet, Stéphane}
\index{EscadeillasGilles@Escadeillas, Gilles}
%
% Titre
\begin{flushleft}
\phantomsection\addtocounter{section}{1}
\addcontentsline{toc}{section}{Réflexions sur l'utilisation de capteurs de densité de flux thermique sur un banc expérimental in situ}
{\Large \textbf{Réflexions sur l'utilisation de capteurs de densité de flux thermique sur un banc expérimental in situ}}\label{ref:15}
\end{flushleft}
%
% Auteurs
Marina Malagoni$^{1,\star}$, Sandra Gallego$^{2}$, Stéphane Ginestet$^{2}$, Gilles Escadeillas$^{2}$\\[2mm]
$^{\star}$ \Letter : \url{malagoni@insa-toulouse.fr}\\[2mm]
{\footnotesize $^{1}$ Institut Fédéral de Goiás}\\
{\footnotesize $^{2}$ LMDC}\\
[4mm]
%
% Mots clés
\noindent \textbf{Mots clés : } monitoring in-situ ; fluxmètres; métrologie ; étalonnage ; sensibilité.\\[4mm]
%
% Résumé
\noindent \textbf{Résumé : } 

{\normalsize
Ce travail s'intéresse à des problèmes soulevés pendant la mise en place d'un banc d'essais pour monitorer in situ des façades d'un bâtiment ancien avant et après rénovation, afin d'appréhender le comportement thermo-hydrique de ces parois. La littérature est riche en travaux décrivant les différentes méthodes d'analyse des propriétés thermo-physiques, surtout la valeur du coefficient de transmission thermique (U), et en simulations numériques décrivant le comportement de murs anciens. Cependant on constate aussi un écart entre les valeurs déterminées sur site en conditions réelles et les valeurs théoriques, souvent attribué aux modèles de transfert de chaleur et de masse et assez souvent corrigé avec des coefficients multiplicateurs (type « offset »). On observe aussi un manque d'informations sur la méthodologie de mise en place des capteurs, notamment en ce que concerne l'étalonnage, la dérive de la sensibilité avec le temps, les contraintes au niveau de la configuration des centrales d'acquisition, la forme du capteur elle-même et les longueurs de câblage électrique. L'objectif principal de ce travail est donc d'apporter une réflexion sur ces critères concernant des capteurs de densité de flux, ainsi que de présenter un nouveau type de fluxmètre perforé qui serait plus adapté pour des parois respirantes sur bâtiments anciens. Une nouvelle méthode d'étalonnage de ces capteurs est aussi présentée et comparée à celle de la norme américaine ASTM C1130-17. Une étude particulière est aussi menée en terme d'analyse systématique de l'ampleur et de l'importance relative des incertitudes sur la sensibilité des fluxmètres. Les résultats montrent que le critère sur la longueur du câblage électrique n'intervient quasiment pas sur les résultats de sensibilité des capteurs. Cependant, une dérive avec le temps du facteur d'étalonnage entre 5 et 10\% pour les capteurs traditionnels et entre 40 et 80\% pour les capteurs perforés a été constatée par rapport au facteur d'usine.

 \vfill doi : \url{https://doi.org/10.25855/SFT2021-015}

}
 
