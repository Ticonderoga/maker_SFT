

    %=====================================
    %   WARNING
    %   FICHIER AUTOMATISE
    %   NE PAS MODIFIER
    %=====================================

    \newpage

%%%%%%%%%%%%%%%%%%%%%%%%%%%%%%%%%%%%%%%%%%%%
%% Papier 3
%%%%%%%%%%%%%%%%%%%%%%%%%%%%%%%%%%%%%%%%%%%%

% Indexations
\index{KallioSonja@Kallio, Sonja}
\index{SirouxMonica@Siroux, Monica}
%
% Titre
\begin{flushleft}
\phantomsection\addtocounter{section}{1}
\addcontentsline{toc}{section}{Multi-objective design optimization of a hybrid renewable energy system}
{\Large \textbf{Multi-objective design optimization of a hybrid renewable energy system}}\label{ref:3}
\end{flushleft}
%
% Auteurs
Sonja Kallio$^{1}$, Monica Siroux$^{1,\star}$\\[2mm]
$^{\star}$ \Letter : \url{monica.siroux@insa-strasbourg.fr}\\[2mm]
{\footnotesize $^{1}$ INSA Strasbourg ICUBE Strasbourg University}\\
[4mm]
%
% Mots clés
\noindent \textbf{Mots clés : } renewable energy, photovoltaic-thermal, micro cogeneration, multi-objective optimization, hybrid energy system\\[4mm]
%
% Résumé
\noindent \textbf{Résumé : } 

{\normalsize
A photovoltaic-thermal (PVT) collector is a solar-based micro-cogeneration system which generates simultaneously heat and power for buildings. A hybrid renewable energy system consists of one or more renewable energy sources combined with energy storages and provides energy for a certain energy demand. To maximize the self-consumption and autonomy of renewable energy sources, the optimal size of the hybrid energy system components has to be determined. To find an optimal size of the components can lead conflicting objectives in terms of investment costs and system reliability. In recent years, the design methods have got alongside new design methods using nature inspired artificial intelligence methods, such as genetic algorithm (GA), evolutionary algorithm and particle swarm optimization (PSO). In this paper, a dynamic Matlab/Simulink model of the hybrid renewable energy system, including PVT collectors, lithium-ion battery system and thermal storage, is proposed. The system provides domestic hot water (DHW) and electricity for residential building use. The electricity demand profile of a residential user is taken from statistics for different seasons and day types. The hourly yearly weather data of Strasbourg is used to estimate PVT energy production. A new approach to optimize the sizing of the hybrid system is proposed. The elitist Non-Dominated Sorting Genetic algorithm (NSGA-II) is applied to the model to solve a multi-objective optimization problem. The formulation of the 3-dimensional optimization problem aims to minimize the initial investment costs and maximize thermal and electrical reliability of the system, simultaneously. The number of the PVT collectors in series and parallel, capacity of the battery system, thermal storage volume and coolant mass flow rate are applied as design parameters. As a result, the Pareto optimal front is generated with freedom to choose an optimal solution for a desired design. The variation between the optimal solutions was discussed and three solutions were selected. In the case with the best reliability, 41\% of the yearly heating demand and 60\% of the yearly electricity demand were covered by the PVT collectors, and the initial investment of the system was 22601 €.

 \vfill doi : \url{https://doi.org/10.25855/SFT2021-003}

}
 
