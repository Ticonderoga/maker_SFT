

    %=====================================
    %   WARNING
    %   FICHIER AUTOMATISE
    %   NE PAS MODIFIER
    %=====================================

    \newpage

%%%%%%%%%%%%%%%%%%%%%%%%%%%%%%%%%%%%%%%%%%%%
%% Papier 69
%%%%%%%%%%%%%%%%%%%%%%%%%%%%%%%%%%%%%%%%%%%%

% Indexations
\index{RetailleauFlorent@Retailleau, Florent}
\index{AllheilyVadim@Allheily, Vadim}
\index{MerlatLionel@Merlat, Lionel}
\index{HenryJean-Francois@Henry, Jean-François}
\index{RandrianalisoaJaona@Randrianalisoa, Jaona}
%
% Titre
\begin{flushleft}
\phantomsection\addtocounter{section}{1}
\addcontentsline{toc}{section}{Etude du transfert radiatif dans les matériaux composites semi-transparents}
{\Large \textbf{Etude du transfert radiatif dans les matériaux composites semi-transparents}}\label{ref:69}
\end{flushleft}
%
% Auteurs
Florent Retailleau$^{1,\star}$, Vadim Allheily$^{1}$, Lionel Merlat$^{1}$, Jean-François Henry$^{2}$, Jaona Randrianalisoa$^{2}$\\[2mm]
$^{\star}$ \Letter : \url{florent.retailleau@isl.eu}\\[2mm]
{\footnotesize $^{1}$ Institut franco-allemand de recherches de Saint-Louis}\\
{\footnotesize $^{2}$ Institut de Thermique, Mécanique et Matériaux (ITheMM)}\\
[4mm]
%
% Mots clés
\noindent \textbf{Mots clés : } laser, matériaux composites, fibres, diffusion\\[4mm]
%
% Résumé
\noindent \textbf{Résumé : } 

{\normalsize
Les matériaux composites sont aujourd'hui incontournables dans de nombreux domaines en raison de leurs caractéristiques techniques (légèreté, solidité, rigidité) et de la réduction des coûts de production. Ces matériaux ont permis l'émergence des drones civils et militaires qui représentent une nouvelle menace. L'utilisation de la technologie laser permet d'appliquer des contraintes thermiques au niveau de certaines zones afin d'engendrer des dégradations thermomécaniques critiques pour le système en vol. L'objectif de cette étude est de mieux comprendre la propagation d'un flux lumineux au sein d'un matériau composite semi-transparent diffusant en fonction de la température.



Ces dernières années, de nombreuses études ont porté sur les transferts radiatifs en milieux semi-transparents diffusants. L'équation de transfert radiatif (ETR) est conventionnellement utilisée pour modéliser la propagation du rayonnement. Dans la plupart de ces travaux, les frontières des matériaux sont traitées comme transparentes ou lisses. Une des difficultés sur l'étude du transfert radiatif au sein des milieux composites concerne la présence des frontières rugueuses provenant du procédé d'élaboration. Ces frontières rugueuses sont difficiles à appréhender du point de vue théorique et leurs présences affectent les mesures. 



Une meilleure connaissance des propriétés radiatives volumétriques (coefficient d'absorption, coefficient de diffusion et fonction de phase) et surfaciques (réflectivités et transmittivités aux interfaces) est indispensable. L'approche expérimentale permet de retrouver ces propriétés radiatives grâce à un dispositif de mesure adapté et à la maitrise de la température de l'échantillon.



Le présent travail fait suite aux travaux présentés en 2019. Il se focalise sur l'identification simultanée par méthode de Gauss-Newton des propriétés radiatives volumétriques et surfaciques de matériaux composites de la température ambiante jusqu'à la température de début de dégradation. Le modèle direct de résolution de l'ETR utilise la méthode de Monte Carlo et est basé sur six inconnus : trois propriétés volumétriques et trois propriétés de diffusion surfacique. La fonction de phase est modélisée par la fonction de Henyey-Greenstein. Cette étude montre que les propriétés identifiées permettent de reproduire les lobes de réflectance et de transmittance bidirectionnelles d'un matériau semi-transparent rugueux à différentes températures.

 \vfill doi : \url{https://doi.org/10.25855/SFT2021-069}

}
 
