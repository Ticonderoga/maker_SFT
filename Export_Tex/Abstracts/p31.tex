

    %=====================================
    %   WARNING
    %   FICHIER AUTOMATISE
    %   NE PAS MODIFIER
    %=====================================

    \newpage

%%%%%%%%%%%%%%%%%%%%%%%%%%%%%%%%%%%%%%%%%%%%
%% Papier 31
%%%%%%%%%%%%%%%%%%%%%%%%%%%%%%%%%%%%%%%%%%%%

% Indexations
\index{BarryElhadjBoubacar@Barry, Elhadj Boubacar}
\index{KangChangwoo@Kang, Changwoo}
\index{YoshikawaHarunori@Yoshikawa, Harunori}
\index{MutabaziInnocent@Mutabazi, Innocent}
%
% Titre
\begin{flushleft}
\phantomsection\addtocounter{section}{1}
\addcontentsline{toc}{section}{Transfert de chaleur par convection thermoélectrique dans des cavités rectangulaires horizontales}
{\Large \textbf{Transfert de chaleur par convection thermoélectrique dans des cavités rectangulaires horizontales}}\label{ref:31}
\end{flushleft}
%
% Auteurs
Elhadj Boubacar Barry$^{1}$, Changwoo Kang$^{2}$, Harunori Yoshikawa$^{3}$, Innocent Mutabazi$^{1,\star}$\\[2mm]
$^{\star}$ \Letter : \url{innocent.mutabazi@univ-lehavre.fr}\\[2mm]
{\footnotesize $^{1}$ Université Le Havre Normandie}\\
{\footnotesize $^{2}$ Université Nationale de Jeonbuk}\\
{\footnotesize $^{3}$ Institut de Physique de Nice - UMR 7010}\\
[4mm]
%
% Mots clés
\noindent \textbf{Mots clés : } Cavité rectangulaire ; Convection thermoélectrique ; Fluide diélectrique ; Force diélectrophorétique.\\[4mm]
%
% Résumé
\noindent \textbf{Résumé : } 

{\normalsize
La maîtrise des échanges thermiques dans les systèmes de refroidissement des équipements industriels (Aérospatial, Aéronautique ou Microfluidique) est un défi technologique majeur. La convection naturelle, l'un des modes de transfert de chaleur qui ne requiert aucun mécanisme externe (ventilateur ou pompe), est un moyen efficace et moins coûteux. Ce phénomène se déclenche dans un fluide lorsque la différence de température $\Delta$T atteint une valeur critique $\Delta$Tc, qui dépend du coefficient de dilatation thermique, de la diffusivité thermique et de la viscosité du fluide. En dessous de cette valeur ou dans des conditions de microgravité, il est impossible d'évacuer la chaleur par convection. L'application d'un champ électrique alternatif E à la couche du fluide induit une force de poussée électrique proportionnelle au carré de l'intensité du champ électrique. Cette force est semblable à la poussée d'Archimède (avec une gravité effective d'origine électrique) et peut induire un transfert de chaleur par convection. La compréhension des mécanismes générant la convection grâce au champ électrique dans un fluide revêt un intérêt crucial pour le contrôle thermique. Dans cette étude, nous avons réalisé une analyse linéaire pour déterminer les valeurs critiques du champ électrique marquant le début de la convection dans le fluide ; ensuite une étude faiblement non linéaire autour du seuil nous a permis de caractériser les différentes structures générées par le couplage thermoélectrique et ainsi quantifier le transfert de chaleur.







La stabilité d'une couche horizontale de fluide d'épaisseur d, de masse volumique $\rho$, de viscosité $\nu$, de diffusivité thermique $\kappa$ et de permittivité $\varepsilon$, soumise à une différence de température $\Delta$T et à un champ électrique d'intensité E est régie par la compétition entre les effets stabilisants (dissipation visqueuse et diffusion thermique) et les effets déstabilisants (la poussée d'Archimède et la poussée électrique). Pour mettre en évidence les effets de la poussée électrique, la quantité $\Delta$T est choisie telle que sa valeur soit inférieure à l'écart de température seuil $\Delta$Tc de la convection de Rayleigh-Bénard. Le système admet donc un état de base caractérisé par un régime de conduction pure en l'absence du champ électrique. Lorsque la valeur du champ électrique atteint une valeur critique Ec, la poussée électrique déstabilise l'état de base indépendamment de la nature du fluide. Le seuil de l'instabilité dépend du gradient de température $\Delta$T mais il est indépendant de la nature du fluide i.e. du nombre de Prandtl Pr=$\nu$⁄$\kappa$. Les modes critiques se manifestent sous forme de rouleaux de convection droits stationnaires d'axe horizontal. Au-delà du seuil critique, en augmentant l'intensité du champ électrique E, l'évolution des rouleaux de convection, initialement droits, se complexifie. Les rouleaux de convection deviennent modulés, ensuite oscillants avec des défauts de type dislocations, similaires à celles obtenues dans la convection de Rayleigh-Bénard. Le coefficient de transfert de chaleur augmente avec la valeur du champ électrique : ce qui prouve que la force DEP est un moyen alternatif d'évacuation de chaleur pour certains systèmes de refroidissement.

 \vfill doi : \url{https://doi.org/10.25855/SFT2021-031}

}
 
