

    %=====================================
    %   WARNING
    %   FICHIER AUTOMATISE
    %   NE PAS MODIFIER
    %=====================================

    \newpage

\backgroundsetup{contents={Work In Progress},scale=7}
\BgThispage
%%%%%%%%%%%%%%%%%%%%%%%%%%%%%%%%%%%%%%%%%%%%
%% Papier 67
%%%%%%%%%%%%%%%%%%%%%%%%%%%%%%%%%%%%%%%%%%%%

% Indexations
\index{MaireJeremie@Maire, Jérémie}
\index{Chavez-AngelEmigdio@Chavez-Angel, Emigdio}
\index{ArreguiGuillermo@Arregui, Guillermo}
\index{ColombanoMartinF.@Colombano, Martin F.}
\index{CapujNestorE.@Capuj, Nestor E.}
\index{GriolAmadeu@Griol, Amadeu}
\index{MartinezAlejandro@Martinez, Alejandro}
\index{AhopeltoJouni@Ahopelto, Jouni}
\index{Navarro-UrriosDaniel@Navarro-Urrios, Daniel}
\index{Sotomayor-TorresClivia@Sotomayor-Torres, Clivia}
%
% Titre
\begin{flushleft}
\phantomsection\addtocounter{section}{1}
\addcontentsline{toc}{section}{Double mesure des propriétés thermiques de nanostructures optomécaniques en silicium nanocristallin}
{\Large \textbf{Double mesure des propriétés thermiques de nanostructures optomécaniques en silicium nanocristallin}}\label{ref:67}
\end{flushleft}
%
% Auteurs
Jérémie Maire$^{1,\star}$, Emigdio Chavez-Angel$^{2}$, Guillermo Arregui$^{3}$, Martin F. Colombano$^{3}$, Nestor E. Capuj$^{4}$, Amadeu Griol$^{5}$, Alejandro Martinez$^{5}$, Jouni Ahopelto$^{6}$, Daniel Navarro-Urrios$^{7}$, Clivia Sotomayor-Torres$^{8}$\\[2mm]
$^{\star}$ \Letter : \url{jeremie.maire@u-bordeaux.fr}\\[2mm]
{\footnotesize $^{1}$ Institut Catalan des Nanoscience et Nanotechnologies (ICN2), CSIC et BIST;   * Adresse actuelle: I2M, CNRS UMR5295, Talence}\\
{\footnotesize $^{2}$ Institut Catalan des Nanoscience et Nanotechnologies (ICN2), CSIC et BIST}\\
{\footnotesize $^{3}$ ICN2, CSIC et BIST; (2) Département de Physique, UAB, Barcelone}\\
{\footnotesize $^{4}$ Département de physique, Université de la Laguna; (4) Institut Universitaire des Matériaux et Nanotechnologies, U. de la Laguna}\\
{\footnotesize $^{5}$ Nanophotonics Technology Center, Université Polytechnique de Valence}\\
{\footnotesize $^{6}$ VTT Technical Research Centre of Finland Ltd}\\
{\footnotesize $^{7}$ MIND-IN2UB, Département d'ingénierie électronique et biomédicale, Faculté de physique, U. de Barcelone}\\
{\footnotesize $^{8}$ ICREA, Barcelone}\\
[4mm]
%
% Mots clés
\noindent \textbf{Mots clés : } silicium nanocristallin; phonons; thermoréflectance; optomécanique;\\[4mm]
%
% Résumé
\noindent \textbf{Résumé : } 

{\normalsize
Dans de nombreuses situations, les propriétés thermiques limitent ou déterminent les performances de micro- et nano-objets, comme en atteste par exemple la microélectronique aujourd'hui ou le développement de récupérateurs d'énergie aux petites échelles. Nous nous intéressons ici au cas des propriétés thermiques de nanostructures optomécaniques. Ces structures, qui présentent des résonances optiques (~1500 nm) et mécaniques (GHz) colocalisées dans une cavité micrométrique, sont fortement impactées par leurs propriétés thermiques. En effet, la fréquence d'émission d'onde mécaniques cohérentes, le « laser mécanique », dépend directement de la dissipation thermique dans la structure.



 



Nous avons donc caractérisé les propriétés thermiques de ces structures, à savoir la constante de dissipation thermique et la conductivité thermique. Ces nanostructures optomécaniques sont constituées de nanofils perforés de trous cylindriques et possédant des ailettes et sont fabriquées de silicium nanocristallin. Dans un premier temps, nous utilisons une technique de thermoréflectance pour comparer la conductivité thermique de membranes suspendues de silicium nanocristallin de 220 nm d'épaisseur avec différentes tailles de grains. On observe une réduction de cette conductivité thermique par un facteur allant jusqu'à 8 par rapport à son équivalent cristallin pour une taille moyenne des grains de 165 nm. La même expérience est ensuite réalisée dans les nanostructures optomécaniques pour comparer l'importance de la nanostructuration par rapport à celle des grains sur la diminution de la conductivité thermique. La réduction observée reste présente mais plus faible que dans le matériau cristallin dû à la compétition entre la diffusion de surface et l'impact des joints de grains sur la diffusion des phonons.



 



Enfin, une nouvelle technique sans contact est introduite pour mesurer le temps de dissipation thermique et la conductivité thermique de structures possédant une résonance optique, qui pourra être adaptée à tous types de cristaux photoniques. Cette technique pompe-sonde consiste à mesurer la vitesse de refroidissement de la résonance optique grâce au changement de longueur d'onde et utilise également des simulations éléments finis pour extraire la conductivité thermique des grandeurs mesurées. On démontre l'équivalence de cette technique avec plusieurs géométries de résonances optiques et l'accord de ces mesures avec la thermoréflectance, ce qui ouvrent de nouvelles perspectives de mesure des propriétés thermiques dans des structures photoniques.

 \vfill Work In Progress

}
 
