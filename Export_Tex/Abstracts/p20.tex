

    %=====================================
    %   WARNING
    %   FICHIER AUTOMATISE
    %   NE PAS MODIFIER
    %=====================================

    \newpage

\backgroundsetup{contents={Work In Progress},scale=7}
\BgThispage
%%%%%%%%%%%%%%%%%%%%%%%%%%%%%%%%%%%%%%%%%%%%
%% Papier 20
%%%%%%%%%%%%%%%%%%%%%%%%%%%%%%%%%%%%%%%%%%%%

% Indexations
\index{MahamanLaoualiSouleyYoussoufou@Mahaman Laouali Souley, Youssoufou}
\index{Perlot-BascoulesCeline@Perlot-Bascoules, Céline}
\index{Castaing-LasvignottesJean@Castaing-Lasvignottes, Jean}
\index{YoussoufBenjamine@Youssouf, Benjamine}
\index{AdelardLaetitia@Adelard, Laetitia}
%
% Titre
\begin{flushleft}
\phantomsection\addtocounter{section}{1}
\addcontentsline{toc}{section}{Mesures des propriétés thermiques des bétons à base de copeaux de bois}
{\Large \textbf{Mesures des propriétés thermiques des bétons à base de copeaux de bois}}\label{ref:20}
\end{flushleft}
%
% Auteurs
Youssoufou Mahaman Laouali Souley$^{1,\star}$, Céline Perlot-Bascoules$^{2}$, Jean Castaing-Lasvignottes$^{1}$, Benjamine Youssouf$^{1}$, Laetitia Adelard$^{1}$\\[2mm]
$^{\star}$ \Letter : \url{youssoufou.mahaman-laouali-souley@univ-reunion.fr}\\[2mm]
{\footnotesize $^{1}$ PIMENT}\\
{\footnotesize $^{2}$ SIAME}\\
[4mm]
%
% Mots clés
\noindent \textbf{Mots clés : } Béton biosourcé, copeaux de bois, conductivité thermique, granulats végétaux\\[4mm]
%
% Résumé
\noindent \textbf{Résumé : } 

{\normalsize
Le secteur du BTP est l'un des secteurs les plus importants d'un point de vue empreinte écologique (consommation de matières premières et d'énergie, production de gaz à effet de serre), notamment du fait de l'utilisation du béton. Le béton est un mélange entre un liant, de l'eau et des granulats plus des adjuvants. Les activités liées au bâtiment et aux travaux publics constituent l'une des plus grosses sources de consommation de granulats. Ces dernières années, ce secteur connaît une évolution croissante à La Réunion. De plus, les fabricants de ciment locaux utilisent de la pouzzolane réunionnaise qui est un constituant important du mélange pour la fabrication du ciment. Afin de s'inscrire dans une démarche de développement durable et d'efficacité énergétique, des alternatives dans le processus de fabrication du béton doivent être trouvées. 







La substitution des granulats naturels par des coproduits ou sous-produits agricoles ou industriels locaux est une solution envisageable. Afin de définir dans quelles mesures l'utilisation de granulats végétaux dans le béton permettrait de limiter l'épuisement des ressources naturelles non renouvelables. Dans ce cadre, nous procédons à plusieurs études. Ces dernières concernent notamment la caractérisation thermique, hydrique et mécanique du béton biosourcé dans le but d'obtenir un meilleur confort thermique dans les bâtiments, tout en assurant une résistance mécanique acceptable d'un point de vue constructif. Plusieurs facteurs influent sur le comportement thermique des bétons composites, particulièrement la taille des fibres, le rapport massique eau/liant, les procédés de mise en forme, le malaxage, et surtout la nature du liant. Par conséquent, nous effectuons plusieurs tests sur des mélanges entre différentes tailles de copeaux et un liant reformulé (ciment plus chaux) dans le but de les caractériser. 







Dans cet article, nous présentons principalement les mesures de conductivité thermique (par la méthode fluxmétrique) du béton à base de copeaux de bois. Nous avons testé plusieurs échantillons contenant un taux de granulats biosourcés variable (de 0 \% à 20 \%). Les premiers résultats ainsi qu'une analyse comparative au béton traditionnel montrent des valeurs encourageantes.

 \vfill Work In Progress

}
 
