

    %=====================================
    %   WARNING
    %   FICHIER AUTOMATISE
    %   NE PAS MODIFIER
    %=====================================

    \newpage

%%%%%%%%%%%%%%%%%%%%%%%%%%%%%%%%%%%%%%%%%%%%
%% Papier 63
%%%%%%%%%%%%%%%%%%%%%%%%%%%%%%%%%%%%%%%%%%%%

% Indexations
\index{AoualiAbderezak@Aouali, Abderezak}
\index{ChevalierStephane@Chevalier, Stéphane}
\index{SommierAlain@Sommier, Alain}
\index{BatsaleJean-Christophe@Batsale, Jean-Christophe}
\index{PradereChristophe@Pradere, Christophe}
%
% Titre
\begin{flushleft}
\phantomsection\addtocounter{section}{1}
\addcontentsline{toc}{section}{Développement d'un fluxmètre imageur hyperspectrale sans contact  par thermographie InfraRouge}
{\Large \textbf{Développement d'un fluxmètre imageur hyperspectrale sans contact  par thermographie InfraRouge}}\label{ref:63}
\end{flushleft}
%
% Auteurs
Abderezak Aouali$^{1,\star}$, Stephane Chevalier$^{1}$, Alain Sommier$^{1}$, Jean-Christophe Batsale$^{1}$, Christophe Pradere$^{1}$\\[2mm]
$^{\star}$ \Letter : \url{abderezak.aouali@u-bordeaux.fr}\\[2mm]
{\footnotesize $^{1}$ I2M}\\
[4mm]
%
% Mots clés
\noindent \textbf{Mots clés : } Fluxmètrie hyperspectrale, thermographie infrarouge, méthodes inverses\\[4mm]
%
% Résumé
\noindent \textbf{Résumé : } 

{\normalsize
Ces travaux s'inscrivent dans le cadre du projet IGAR qui vise à caractériser thermiquement et chimiquement des torches à plasma. L'enjeu principal est la mesure des champs 3D de la température et du flux sans contact en vue de l'optimisation énergétique des torches.







La connaissance du flux thermique est souvent primordiale dans certains domaines, citons à titre d'exemple le domaine de la construction, l'aéronautique, l'aérospatiale, la métrologie ..., cette connaissance du flux thermique peut: (i), permettre la réalisation de bilans thermiques, (ii), servir de données d'entrées à des modèles ou (ii), contrôler les procédés.







Dans les travaux précédents, il existe de nombreuses méthodes d'estimation des sources de chaleur en fonction de la nature du transfert thermique : conduction, convection ou rayonnement; on s'intéressera dans cette étude uniquement aux méthodes inverses thermiques liées au transfert conductif.







Différents capteurs de flux (pour des estimations du flux ponctuel ou spatialement répartie) ont déjà été développés auparavant en se basant sur différentes méthodes (méthodes inverses analytiques, méthodes inverses numériques, méthode du gradient de température spatiale ...). Ces capteurs sont tous conçus pour l'estimation de source dans la gamme spectrale de l'infrarouge. De plus, l'estimation simultanée de la répartition spatiale ainsi que l'amplitude (densité d'énergie) de la source est rarement atteint. 







Ici l'objectif est de développer un capteur de flux hyperspectrale en utilisant un film  de carbone très fin et homogène appelé thermoconvertisseur hyperspectrale. Des études ont été faites auparavant sur le thermoconvertisseur et ont démontré sa capacité à absorber le rayonnement dans une très large gamme spectrale (du visible aux ondes radio) avec des proportionnalités différentes (selon la longueur d'onde). Ce capteur permettra d'estimer la répartition spatiale de la source ainsi que sa densité d'énergie. Dans cette communication, nous présenterons le montage expérimental ainsi que le modèle analytique et la méthode inverse utilisés. Ensuite, nous proposerons une méthode de calibration du thermoconvertisseur hyperspectrale afin de mettre en évidence l'aspect quantitatif du capteur, Enfin, nous présenterons des résultats obtenus par le capteur pour différentes sources (différentes gammes spectrales).

 \vfill doi : \url{https://doi.org/10.25855/SFT2021-063}

}
 
