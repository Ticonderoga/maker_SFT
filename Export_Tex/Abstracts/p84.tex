

    %=====================================
    %   WARNING
    %   FICHIER AUTOMATISE
    %   NE PAS MODIFIER
    %=====================================

    \newpage

\backgroundsetup{contents={Work In Progress},scale=7}
\BgThispage
%%%%%%%%%%%%%%%%%%%%%%%%%%%%%%%%%%%%%%%%%%%%
%% Papier 84
%%%%%%%%%%%%%%%%%%%%%%%%%%%%%%%%%%%%%%%%%%%%

% Indexations
\index{VanneremSegolene@Vannerem, Ségolène}
\index{FalcozQuentin@Falcoz, Quentin}
\index{NeveuPierre@Neveu, Pierre}
%
% Titre
\begin{flushleft}
\phantomsection\addtocounter{section}{1}
\addcontentsline{toc}{section}{Étude expérimentale de l'impact de la distribution fluide sur un stockage thermique de type thermocline}
{\Large \textbf{Étude expérimentale de l'impact de la distribution fluide sur un stockage thermique de type thermocline}}\label{ref:84}
\end{flushleft}
%
% Auteurs
Ségolène Vannerem$^{1,\star}$, Quentin Falcoz$^{2}$, Pierre Neveu$^{3}$\\[2mm]
$^{\star}$ \Letter : \url{segolene.vannerem@promes.cnrs.fr}\\[2mm]
{\footnotesize $^{1}$ PROMES-CNRS}\\
{\footnotesize $^{2}$ PROMES-CNRS, UPVD}\\
{\footnotesize $^{3}$ UPVD}\\
[4mm]
%
% Mots clés
\noindent \textbf{Mots clés : } thermocline, stockage chaleur sensible, distributeur fluide, centrales cylindro-paraboliques, expérimental\\[4mm]
%
% Résumé
\noindent \textbf{Résumé : } 

{\small
Dans le contexte de transition vers les énergies renouvelables, l'utilisation de la ressource solaire présente des avantages majeurs tels que sa gratuité et son abondance. Cependant, l'intermittence de la production solaire demeure problématique et rend nécessaire l'utilisation d'un système de stockage pour adapter l'offre à la demande énergétique. 







La méthode la plus couramment utilisée dans les centrales solaires thermodynamiques consiste à faire circuler un fluide caloporteur dans le champ de collecteurs solaires et d'ensuite transférer son énergie thermique à un matériau de stockage. La production d'électricité peut alors s'effectuer de façon différée en transférant l'énergie stockée dans le matériau à un fluide de travail utilisé dans un cycle thermodynamique. Le matériau de stockage peut être conservé dans une cuve unique utilisée aussi bien pour la charge que pour la décharge, ce qui permet une réduction conséquente du coût de stockage. Les zones chaude et froide sont en contact direct à l'intérieur de cette cuve unique (appelée thermocline) et il est attendu que la stratification thermique entre ces zones impacte les performances de stockage. C'est pourquoi la méthode de distribution du fluide caloporteur dans la cuve est étudiée afin de déterminer son influence sur la répartition thermique au sein de la cuve et sur le fonctionnement global du stockage. 







Pour ce faire, une campagne expérimentale a été menée sur un prototype de centrale solaire du laboratoire PROMES-CNRS afin de comparer trois méthodes de distribution. La centrale MicroSol-R comprend trois collecteurs cylindro-paraboliques et une résistance électrique permettant de chauffer le fluide caloporteur lors de la charge, un stockage thermocline de 220 kWh et un générateur de vapeur pour la production en décharge. 







Trois distributeurs ont été successivement placés en haut de la cuve afin de comparer différentes méthodes d'injection en charge et d'extraction en décharge : distribution uniforme du fluide sur la section de la cuve, distribution centrale et distribution périphérique. 







Des cycles charge/décharge identiques ont d'abord été comparés entre les distributeurs afin d'isoler l'influence de celui-ci sur les performances. Une étude paramétrique a ensuite été réalisée sur l'influence du débit et de la température. La performance du stockage est évaluée au moyen du taux d'utilisation qui compare la variation d'enthalpie au cours d'une charge à la variation d'enthalpie maximale qui aurait été obtenue si le stockage avait été chargé jusqu'à la température haute. Un taux d'utilisation de décharge est défini de façon similaire. 







Les expériences conduites ont permis de mettre en évidence l'influence de la distribution du fluide sur les performances d'un stockage de type thermocline. L'observation des profils de température radiaux montre que le système d'injection du fluide influence les transferts de chaleur et de masse au sein de la cuve. À l'échelle du procédé cependant, il a été montré qu'une amélioration de l'homogénéité radiale ne résulte pas nécessairement en un accroissement du taux d'utilisation et réciproquement. Le distributeur ne semble pas avoir d'influence significative sur le taux d'utilisation tandis que le débit et la température ont un impact en accord avec les prévisions numériques.

 \vfill Work In Progress

}
 
