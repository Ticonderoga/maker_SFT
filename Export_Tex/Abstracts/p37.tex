

    %=====================================
    %   WARNING
    %   FICHIER AUTOMATISE
    %   NE PAS MODIFIER
    %=====================================

    \newpage

%%%%%%%%%%%%%%%%%%%%%%%%%%%%%%%%%%%%%%%%%%%%
%% Papier 37
%%%%%%%%%%%%%%%%%%%%%%%%%%%%%%%%%%%%%%%%%%%%

% Indexations
\index{LunaValenciaJuanEsteban@Luna Valencia, Juan Esteban}
\index{V.SOliveiraArthur@V.S Oliveira, Arthur}
\index{LabergueAlexandre@Labergue, Alexandre}
\index{GlantzTony@Glantz, Tony}
\index{RepettoGeorges@Repetto, Georges}
\index{GradeckMichel@Gradeck, Michel}
%
% Titre
\begin{flushleft}
\phantomsection\addtocounter{section}{1}
\addcontentsline{toc}{section}{Simulation de l'écoulement dispersé vapeur/gouttes dans des conditions d'APRP}
{\Large \textbf{Simulation de l'écoulement dispersé vapeur/gouttes dans des conditions d'APRP}}\label{ref:37}
\end{flushleft}
%
% Auteurs
Juan Esteban Luna Valencia$^{1,\star}$, Arthur V.S Oliveira$^{1}$, Alexandre Labergue$^{1}$, Tony Glantz$^{2}$, Georges Repetto$^{3}$, Michel Gradeck$^{1}$\\[2mm]
$^{\star}$ \Letter : \url{lunavale1@univ-lorraine.fr}\\[2mm]
{\footnotesize $^{1}$ LEMTA, CNRS, UMR 7563, Vandœuvre-lès-Nancy, F-54500}\\
{\footnotesize $^{2}$ IRSN PSN/SEMIA/LEMC, B.P. 3, 13 115 St-Paul-Lez-Durance Cedex, France}\\
{\footnotesize $^{3}$ IRSN PRSN/RES/SEREX, B.P. 3, 13 115 St-Paul-Lez-Durance Cedex}\\
[4mm]
%
% Mots clés
\noindent \textbf{Mots clés : } APRP ; NECTAR ; Thermo-hydraulique ; Modèle mécaniste\\[4mm]
%
% Résumé
\noindent \textbf{Résumé : } 

{\normalsize
Lors d'un accident de perte de réfrigérant primaire (APRP), de l'eau est injectée dans le cœur du réacteur et un écoulement dispersé de vapeur et de gouttes se produit en aval du front de remouillage qui se propage dans les assemblages. Par conséquent, cet écoulement joue un rôle très important dans le refroidissement initial des crayons combustibles qui ne sont pas encore immergés dans l'eau. Ainsi, les assemblages peuvent être déformés avec certains sous-canaux bouchés en raison du gonflement des gaines combustibles. Cependant, le refroidissement des sous-canaux bouchés est dégradé en raison du flux de vapeur préférentiellement dévié vers les sous-canaux non ou moins bouchés. Dans un travail précédent, nous avons implémenté un modèle mécanistique dans un code de calcul (NECTAR) afin de calculer les transferts de chaleur et de masse mis en jeux entre l'écoulement polydispersé et les crayons ainsi que la dynamique des gouttes. Ce code a été validé par des mesures expérimentales avec trois géométries différentes représentant le gonflement de la gaine à l'échelle du sous-canal. Par ailleurs, l'IRSN a réalisé des mesures des taux de redistribution du débit dans un assemblage de 7 × 7 avec plusieurs crayons ballonnés pour différentes conditions géométriques (taux et longueur de bouchage et coplanarité) ainsi que pour différents débits. Les résultats ont montré que le taux de bouchage est le facteur prédominant pour la quantité de débit dévié. Dans cet article, l'objectif est d'analyser l'influence de la déviation du flux de vapeur sur les transferts de chaleur dans deux sous-canaux bouchés (61\% et 90\%) pour un débit de vapeur de 5kg/h à une température de 500 $^{\circ}$C, un débit des gouttes de 10kg/h à température de saturation, et avec des crayons à 1000 $^{\circ}$C comme température initiale, qui sont des conditions représentatives d'un APRP. La dissipation thermique interne totale est évaluée et les contributions des différents mécanismes impliqués sont analysées.

 \vfill doi : \url{https://doi.org/10.25855/SFT2021-037}

}
 
