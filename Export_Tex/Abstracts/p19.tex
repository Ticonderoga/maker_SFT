

    %=====================================
    %   WARNING
    %   FICHIER AUTOMATISE
    %   NE PAS MODIFIER
    %=====================================

    \newpage

%%%%%%%%%%%%%%%%%%%%%%%%%%%%%%%%%%%%%%%%%%%%
%% Papier 19
%%%%%%%%%%%%%%%%%%%%%%%%%%%%%%%%%%%%%%%%%%%%

% Indexations
\index{PoncetChristophe@Poncet, Christophe}
\index{FerrouillatSebastien@Ferrouillat, Sébastien}
\index{VignalLaure@Vignal, Laure}
\index{Bulliard-SauretOdin@Bulliard-Sauret, Odin}
\index{GondrexonNicolas@Gondrexon, Nicolas}
%
% Titre
\begin{flushleft}
\phantomsection\addtocounter{section}{1}
\addcontentsline{toc}{section}{Influence de l'interaction entre ultrasons et écoulement sur l'intensification du transfert thermique : effets de la fréquence des ondes selon le régime hydrodynamique}
{\Large \textbf{Influence de l'interaction entre ultrasons et écoulement sur l'intensification du transfert thermique : effets de la fréquence des ondes selon le régime hydrodynamique}}\label{ref:19}
\end{flushleft}
%
% Auteurs
Christophe Poncet$^{1,\star}$, Sébastien Ferrouillat$^{1}$, Laure Vignal$^{1}$, Odin Bulliard-Sauret$^{1}$, Nicolas Gondrexon$^{2}$\\[2mm]
$^{\star}$ \Letter : \url{christophe.poncet@univ-grenoble-alpes.fr}\\[2mm]
{\footnotesize $^{1}$ Université Grenoble-Alpes, CNRS, Grenoble INP, LEGI, 38000 Grenoble, France}\\
{\footnotesize $^{2}$ Université Grenoble-Alpes - Laboratoire Rhéologie et Procédés - LRP UMR 5520, Domaine Universitaire, 38610 Gières, France}\\
[4mm]
%
% Mots clés
\noindent \textbf{Mots clés : } Intensification des transferts thermiques ; Ultrasons ; Convection forcée\\[4mm]
%
% Résumé
\noindent \textbf{Résumé : } 

{\normalsize
Plusieurs études ont montré que l'intensification des transferts thermiques générée par les ultrasons est étroitement associée à la génération d'effets hydrodynamiques à des échelles et des intensités différentes selon la fréquence choisie. Les ultrasons basse fréquence (20 – 100 kHz) sont connus pour leur capacité à intensifier les transferts thermiques, notamment en convection forcée et naturelle. En effet, l'étude hydrodynamique d'écoulements en conduite en présence d'ultrasons basse fréquence a permis de mettre en évidence une perturbation du champ de vitesses et une augmentation de la turbulence en présence de cavitation acoustique, phénomène principalement induit par ce type d'ondes. Il a également été montré que la cavitation acoustique générée à proximité d'une paroi chauffante permet de perturber l'établissement de la couche limite, et d'induire ainsi une amélioration notable des transferts thermiques. 



Plus récemment, des études se sont intéressées au phénomène d'intensification thermique en présence d'ultrasons haute fréquence (> 1 MHz) pour lesquels aucune cavitation acoustique n'a été détectée. A ces hautes fréquences ultrasonores, ce sont majoritairement des courants acoustiques qui sont produits et qui sont à l'origine d'une mise en mouvement du fluide à l'échelle macroscopique. Appliqués à un écoulement, les courants acoustiques perturbent eux aussi le champ de vitesses et augmentent le mélange entre le cœur de l'écoulement et la paroi. Les courants acoustiques ont ainsi été identifiés comme les facteurs principaux responsables du phénomène d'intensification thermique observé avec des ondes de haute fréquence.



La comparaison de l'intensification du transfert thermique avec des ultrasons de fréquence différente a mis en évidence que les comportements thermiques et hydrodynamiques diffèrent. L'apport des ultrasons haute fréquence se dissipe rapidement à mesure que le débit et le nombre de Reynolds augmentent au-delà des limites du régime laminaire. A l'inverse les ultrasons basse fréquence maintiennent davantage leur capacité à intensifier les transferts thermiques même en régime turbulent. Des hypothèses ont été émises dans la littérature pour expliquer ces différences sans pour autant permettre de trancher précisément sur les mécanismes qui expliquent une telle différence. 



Le travail proposé dans cette étude vise à caractériser le comportement thermique et hydrodynamique via des mesures de nombres de Nusselt et de champs de vitesse afin d'étudier les influences respectives du débit et du nombre de Reynolds sur les phénomènes d'intensification des transferts thermiques et de la turbulence en présence d'ultrasons haute et basse fréquence.

 \vfill doi : \url{https://doi.org/10.25855/SFT2021-019}

}
 
