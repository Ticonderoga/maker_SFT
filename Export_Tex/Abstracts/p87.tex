

    %=====================================
    %   WARNING
    %   FICHIER AUTOMATISE
    %   NE PAS MODIFIER
    %=====================================

    \newpage

\backgroundsetup{contents={Work In Progress},scale=7}
\BgThispage
%%%%%%%%%%%%%%%%%%%%%%%%%%%%%%%%%%%%%%%%%%%%
%% Papier 87
%%%%%%%%%%%%%%%%%%%%%%%%%%%%%%%%%%%%%%%%%%%%

% Indexations
\index{YahiatFeriel@Yahiat, Feriel}
\index{BouvierPascale@Bouvier, Pascale}
\index{RusseilSerge@Russeil, Serge}
\index{AndreChristophe@André, Christophe}
\index{BougeardDaniel@Bougeard, Daniel}
%
% Titre
\begin{flushleft}
\phantomsection\addtocounter{section}{1}
\addcontentsline{toc}{section}{Influence du pas hélicoïdal sur les performances thermo-hydrauliques d'un tube annulaire à parois macro-déformées}
{\Large \textbf{Influence du pas hélicoïdal sur les performances thermo-hydrauliques d'un tube annulaire à parois macro-déformées}}\label{ref:87}
\end{flushleft}
%
% Auteurs
Feriel Yahiat$^{2}$, Pascale Bouvier$^{1,\star}$, Serge Russeil$^{3}$, Christophe André$^{4}$, Daniel Bougeard$^{3}$\\[2mm]
$^{\star}$ \Letter : \url{pascale.bouvier@yncrea.fr}\\[2mm]
{\footnotesize $^{1}$ IMT Lille Douai, Institut Mines-Télécom, Univ. Lille, Centre for Energy and Environment, F-59000, Lille \ Junia, Smart Systems \& Energy, F-59000, Lille}\\
{\footnotesize $^{2}$ IMT Lille Douai, Institut Mines-Télécom, Univ. Lille, Centre for Energy and Environment, F-59000, Lille \ Junia, Smart Systems \& Energy, F-59000, Lille}\\
{\footnotesize $^{3}$ IMT Lille Douai, Institut Mines-Télécom, Univ. Lille, Centre for Energy and Environment, F-59000, Lille}\\
{\footnotesize $^{4}$ Junia, Health \& Environment Department, F-59000, Lille \ U.Lille, CNRS, INRAE, Centrale ,UMR 8207 -UMET -Unité Matériaux et Transformations, F-59000 Lille}\\
[4mm]
%
% Mots clés
\noindent \textbf{Mots clés : } Optimisation du transfert thermique, géométrie annulaire, déformation de paroi, simulations numérique, régime laminaire\\[4mm]
%
% Résumé
\noindent \textbf{Résumé : } 

{\normalsize
Dans une configuration d'écoulement annulaire, la combinaison de déformations radiales successives et alternées sur la paroi externe et hélicoïdales sur la paroi interne améliore grandement les transferts thermiques et le mélange par rapport à un tube annulaire lisse classique. Dans ce poster, nous étudions plus précisément, par voie de simulations numériques, à la fois le mélange et les performances thermiques d'une telle configuration annulaire en jouant sur le pas de l'hélicoïde au niveau de la paroi interne. Les pas choisis sont 0; 0,015; 0,03; 0,0432; 0,06; 0,12 et 0,24. Les simulations sont réalisées pour des écoulements laminaires, incompressibles, anisothermes et stationnaires en utilisant des conditions limites de température constante (320 K) sur les deux parois. L'eau, de caractéristiques thermophysiques constantes, entre à une température de 300 K et est caractérisée par un nombre de Reynolds de 600. Les résultats montrent que les performances thermo-hydrauliques globales vont de 1 pour le tube annulaire simple à 2 pour la géométrie optimale caractérisée par un pas de 0,015. Ces performances thermo-hydrauliques sont évaluées par un facteur de performance: le PEC (Performance Evaluation Criterion). Ce dernier permet de comparer simultanément le gain en termes de transfert thermique rapporté à l'évolution de la puissance de pompage. Le mélange, quant à lui, est caractérisé par les sections de Poincaré. Pour mettre en évidence les sections de Poincaré numériquement, environ 15 000 particules sont injectées à l'entrée de la conduite et nous observons ce qu'elles deviennent à la sortie. Le mélange est caractérisé par une distribution de particules uniforme sur toute la section de passage d'un écoulement. C'est l'advection chaotique. Une analyse fine des champs de vitesse et de température permet d'appréhender les mécanismes physiques mis en jeu et leur variation en fonction de la variation du pas. Et pour certaines géométries, nous avons mis en évidence l'advection chaotique.

 \vfill Work In Progress

}
 
