

    %=====================================
    %   WARNING
    %   FICHIER AUTOMATISE
    %   NE PAS MODIFIER
    %=====================================

    \newpage

%%%%%%%%%%%%%%%%%%%%%%%%%%%%%%%%%%%%%%%%%%%%
%% Papier 11
%%%%%%%%%%%%%%%%%%%%%%%%%%%%%%%%%%%%%%%%%%%%

% Indexations
\index{WirtzMathilde@Wirtz, Mathilde}
\index{StutzBenoit@Stutz, Benoit}
\index{PhanHaiTrieu@Phan, Hai Trieu}
\index{BoudehennFrancois@Boudéhenn, François}
%
% Titre
\begin{flushleft}
\phantomsection\addtocounter{section}{1}
\addcontentsline{toc}{section}{Modélisation numérique et intégration d'un désorbeur à plaques et films tombants dans un prototype de machine à absorption eau-ammoniac}
{\Large \textbf{Modélisation numérique et intégration d'un désorbeur à plaques et films tombants dans un prototype de machine à absorption eau-ammoniac}}\label{ref:11}
\end{flushleft}
%
% Auteurs
Mathilde Wirtz$^{1,\star}$, Benoit Stutz$^{2}$, Hai Trieu Phan$^{3}$, François Boudéhenn$^{4}$\\[2mm]
$^{\star}$ \Letter : \url{mathilde.wirtz@cea.fr}\\[2mm]
{\footnotesize $^{1}$ CEA - Liten / LOCIE}\\
{\footnotesize $^{2}$ LOCIE}\\
{\footnotesize $^{3}$ CEA  - Liten}\\
{\footnotesize $^{4}$ CEA - Liten}\\
[4mm]
%
% Mots clés
\noindent \textbf{Mots clés : } Désorbeur ; Machine à absorption $\unit{NH_3}$/$\unit{H_2O}$; Modélisation numérique ; Prototype expérimental\\[4mm]
%
% Résumé
\noindent \textbf{Résumé : } 

{\normalsize
Depuis quelques années, la demande en climatisation connait un essor particulier, notamment dans les secteurs des bâtiments commerciaux et institutionnels, et dans le secteur industriel.  L'utilisation de machines frigorifiques à absorption alimentées par énergie solaire thermique, par des sources de chaleur fatale, ou via les réseaux de chaleur dispense à la fois la production de froid requise, tout en valorisant ces sources. Les machines à absorption $\unit{NH_3}$/$\unit{H_2O}$ offrent la possibilité de produire du froid négatif, ont une bonne compacité, et sont favorables à l'optimisation des transferts internes de chaleur et de masse. Cependant, le faible écart de volatilité entre l'absorbant ($\unit{H_2O}$) et le réfrigérant ($\unit{NH_3}$), induit la nécessité de l'utilisation d'un rectifieur en sortie du générateur, afin d'éliminer les traces d'eau dans la vapeur d'ammoniac produite. 







Afin de produire une vapeur d'ammoniac purifiée et de garantir une configuration de machine $\unit{NH_3}$/$\unit{H_2O}$ compacte et efficace, un nouveau composant appelé « désorbeur » a été développé.  Il s'agit d'un échangeur de chaleur à plaques et films tombants composé d'une section de génération de vapeur par un fluide caloporteur circulant à contre-courant qui chauffe la solution eau-ammoniac ; et d'une section adiabatique ayant pour rôle la rectification de la vapeur par réabsorption partielle afin d'améliorer sa pureté en $\unit{NH_3}$. 







Dans un premier temps, un modèle numérique de ce désorbeur est développé au travers des bilans de masse, d'espèces et d'enthalpie, des corrélations de transfert de masse et de chaleur, ainsi que des équations d'équilibre à l'interface. Pour une modélisation plus simple et robuste de ce composant, des corrélations d'efficacité caractérisant les performances internes du désorbeur sont proposées. Ensuite, un modèle numérique simulant une machine à absorption $\unit{NH_3}$/$\unit{H_2O}$ est développé, intégrant le désorbeur caractérisé par ses efficacités. 







Dans un deuxième temps, le désorbeur est conçu, fabriqué et implanté dans un prototype de machine à absorption $\unit{NH_3}$/$\unit{H_2O}$ de capacité 5 kW de froid. Il remplace l'ensemble générateur, rectifieur et bouteille de séparation liquide-vapeur initialement assemblés. Diverses études paramétriques sont menées, montrant l'impact des variables d'entrée du désorbeur, et des autres composants, sur les performances de la machine. Une comparaison numérique / expérimentale des performances de la machine à absorption $\unit{NH_3}$/$\unit{H_2O}$ est également développée dans cette étude.

 \vfill doi : \url{https://doi.org/10.25855/SFT2021-011}

}
 
