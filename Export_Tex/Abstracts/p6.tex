

    %=====================================
    %   WARNING
    %   FICHIER AUTOMATISE
    %   NE PAS MODIFIER
    %=====================================

    \newpage

\backgroundsetup{contents={Work In Progress},scale=7}
\BgThispage
%%%%%%%%%%%%%%%%%%%%%%%%%%%%%%%%%%%%%%%%%%%%
%% Papier 6
%%%%%%%%%%%%%%%%%%%%%%%%%%%%%%%%%%%%%%%%%%%%

% Indexations
\index{FrappeMathieu@Frappé, Mathieu}
\index{SempeyAlain@Sempey, Alain}
\index{ViotHugo@Viot, Hugo}
\index{MoraLaurent@Mora, Laurent}
%
% Titre
\begin{flushleft}
\phantomsection\addtocounter{section}{1}
\addcontentsline{toc}{section}{Méthodes de calibration et d'optimisation pour un système solaire combiné innovant}
{\Large \textbf{Méthodes de calibration et d'optimisation pour un système solaire combiné innovant}}\label{ref:6}
\end{flushleft}
%
% Auteurs
Mathieu Frappé$^{1,\star}$, Alain Sempey$^{1}$, Hugo Viot$^{2}$, Laurent Mora$^{1}$\\[2mm]
$^{\star}$ \Letter : \url{mathieu.frappe@u-bordeaux.fr}\\[2mm]
{\footnotesize $^{1}$ Univ.  Bordeaux,  CNRS,  I2M  Bordeaux,  351  cours  de  la  Libération,  F-33400 Talence, France}\\
{\footnotesize $^{2}$ NOBATEK/INEF4, Esplanade des Arts et Métiers, 33400 Talence, France}\\
[4mm]
%
% Mots clés
\noindent \textbf{Mots clés : } Système solaire combiné, calibration, optimisation, analyse exergétique, contrôle optimal\\[4mm]
%
% Résumé
\noindent \textbf{Résumé : } 

{\normalsize
En France, le secteur du bâtiment est responsable de 45% de la consommation finale d'énergie. Bien que les nouvelles réglementations environnementales imposent une isolation plus importante des bâtiments, diminuant de fait la demande en chauffage, l'utilisation de l'énergie solaire pour couvrir une part des besoins en chauffage et en eau chaude sanitaire demeure une solution intéressante. Ainsi, un concept de système solaire combiné innovant, équipé de capteurs solaires thermiques plans non vitrés, modulables, et intégrés en façade, a été développé et est actuellement en exploitation.







L'objectif de ce travail est d'augmenter les performances globales de cette installation grâce à des méthodes d'optimisation employées à la fois pour améliorer le dimensionnement et choisir les stratégies de contrôle.







L'étude d'un tel système dans son environnement, couplé à un bâtiment et soumis à la disponibilité solaire, fait intervenir différents modèles physiques et entraîne donc un haut degré de complexité. De par son approche acausale, qui offre la modularité nécessaire à l'étude de différentes combinatoires et de stratégies de contrôle variées, le langage Modelica a été choisi pour modéliser le comportement de l'installation. Le premier objectif consiste à calibrer ce dernier en le confrontant aux données issues de l'expérimentation, dans un premier temps à l'échelle du composant puis au niveau du modèle complet. Cette étape demande une grande expertise et peut s'avérer fastidieuse, le recours à des méthodes intelligentes telles que les algorithmes génétiques ou prédictifs est donc envisagé.







Des méthodes d'optimisation multi-objectifs seront ensuite mises en place afin d'orienter le choix du dimensionnement et des stratégies de contrôle. Pour cela, l'approche par analyse exergétique a fait l'objet de plusieurs recherches ces dernières années et a démontré un intérêt certain. En permettant de repérer et de quantifier les irréversibilités présentes dans l'installation, il est possible d'introduire des indicateurs de potentiel d'amélioration, et ainsi mieux orienter les choix d'optimisation. Cette approche comparée à une approche basée sur le bilan énergétique permettra de renseigner sur la pertinence d'une telle méthode. Enfin, les gains énergétiques d'un contrôle optimal seront estimés sur une longue période d'utilisation afin d'envisager  la mise en œuvre d'un contrôle avancé sur l'installation réelle.

 \vfill Work In Progress

}
 
