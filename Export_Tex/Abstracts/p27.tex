

    %=====================================
    %   WARNING
    %   FICHIER AUTOMATISE
    %   NE PAS MODIFIER
    %=====================================

    \newpage

%%%%%%%%%%%%%%%%%%%%%%%%%%%%%%%%%%%%%%%%%%%%
%% Papier 27
%%%%%%%%%%%%%%%%%%%%%%%%%%%%%%%%%%%%%%%%%%%%

% Indexations
\index{AdihouYolaine@Adihou, Yolaine}
\index{KaneMalick@Kane, Malick}
\index{RimeSimon@Rime, Simon}
\index{RamousseJulien@Ramousse, Julien}
\index{SouyriBernard@Souyri, Bernard}
%
% Titre
\begin{flushleft}
\phantomsection\addtocounter{section}{1}
\addcontentsline{toc}{section}{Méthodologie d'étude des performances exergétiques d'un réseau anergie - Application au réseau d'Estavayer-le-lac (Suisse).}
{\Large \textbf{Méthodologie d'étude des performances exergétiques d'un réseau anergie - Application au réseau d'Estavayer-le-lac (Suisse).}}\label{ref:27}
\end{flushleft}
%
% Auteurs
Yolaine Adihou$^{1,\star}$, Malick Kane$^{2}$, Simon Rime$^{3}$, Julien Ramousse$^{4}$, Bernard Souyri$^{4}$\\[2mm]
$^{\star}$ \Letter : \url{yolaine.adihou@hefr.ch}\\[2mm]
{\footnotesize $^{1}$ Haute école d'ingénierie et d'architecture de Fribourg  (HEIA-FR / HES-SO Fribourg), Laboratoire d'Optimisation de la Conception et Ingénierie de l'Environnement (USMB)}\\
{\footnotesize $^{2}$ Haute école d'ingénierie et d'architecture de Fribourg  (HEIA-FR / HES-SO Fribourg)}\\
{\footnotesize $^{3}$ Groupe E SA, Haute école d'ingénierie et d'architecture de Fribourg  (HEIA-FR / HES-SO Fribourg)}\\
{\footnotesize $^{4}$ Laboratoire d'Optimisation de la Conception et Ingénierie de l'Environnement (USMB)}\\
[4mm]
%
% Mots clés
\noindent \textbf{Mots clés : } Exergie, Réseaux thermiques, Modélisation, Pompe à chaleur, Réseau anergie\\[4mm]
%
% Résumé
\noindent \textbf{Résumé : } 

{\normalsize
Dans le but de répondre aux objectifs européens et mondiaux de réduction des émissions des gaz à effet de serre, les réseaux thermiques basse température contribuent à la transition vers des systèmes de chauffage et de refroidissement sobres. Parmi ces derniers, les réseaux anergie opèrent à des températures de l'ordre de 10$^{\circ}$C, ce qui permet de fournir simultanément des prestations de chauffage et de refroidissement mais aussi d'intégrer des ressources énergétiques renouvelables (hydrothermie, énergie de récupération, panneaux solaires …). 



Le niveau de température de tels réseaux étant bas, il est primordial de suivre l'évolution de la température en fonction des saisons et des sources d'énergie à disposition afin d'évaluer la performance des unités de production décentralisées. En effet, la performance des pompes à chaleur (PACs) dépend du niveau de température en entrée de l'évaporateur et de la température requise par les usagers. Côté réseau, il est donc nécessaire de prévoir la température d'arrivée en sous-station afin d'évaluer au mieux les performances de l'installation.



L'analyse exergétique est reconnue comme une méthode robuste de localisation des sources d'inefficacité d'un système thermodynamique. Elle permet, contrairement au critère d'efficacité énergétique, de comparer toutes les formes d'énergies (électricité, chaleur) entrantes et sortantes d'un système thermodynamique et de localiser explicitement les pertes internes au système (pertes de charge, pertes thermiques …). 



Dans le présent article, une méthodologie d'étude des performances exergétiques d'un réseau d'anergie couplé à des pompes à chaleur décentralisées est présentée. Premièrement, la distribution de températures et de débits sur le réseau est déterminée sur l'année selon l'évolution de l'environnement externe des conduites et du niveau de température des sources d'énergie. Les modèles thermiques et hydrauliques sont calibrés en fonction des données expérimentales d'une conduite sous-lacustre de la commune d'Estavayer-le-lac (Suisse). Cette première étape permet de déterminer la distribution des températures sur le réseau et ainsi la variation annuelle de la température en entrée de l'évaporateur des PACs. Ensuite, la performance exergétique globale du réseau et notamment des PACs est étudiée selon la variation de la température de fourniture de la prestation de chauffage aux clients. La méthodologie est appliquée pour l'extension du réseau anergie d'Estavayer-le-lac dans l'objectif de prévoir les variations saisonnières de température en entrée des PACs et des échangeurs de refroidissement destinés à alimenter l'Hôpital Intercantonal de la Broye. Les performances exergétiques globales du réseau sont évaluées en fonction des niveaux de température (de la source et de la demande) afin de localiser spécifiquement les sources d'inefficacité et de fournir des recommandations explicites quant aux caractéristiques des pompes à chaleur à déployer.

 \vfill doi : \url{https://doi.org/10.25855/SFT2021-027}

}
 
