

    %=====================================
    %   WARNING
    %   FICHIER AUTOMATISE
    %   NE PAS MODIFIER
    %=====================================

    \newpage

%%%%%%%%%%%%%%%%%%%%%%%%%%%%%%%%%%%%%%%%%%%%
%% Papier 43
%%%%%%%%%%%%%%%%%%%%%%%%%%%%%%%%%%%%%%%%%%%%

% Indexations
\index{MartiDavid@Marti, David}
\index{BastideAlain@Bastide, Alain}
\index{RamalingomDelphine@Ramalingom, Delphine}
\index{CocquetPierre-Henri@Cocquet, Pierre-Henri}
%
% Titre
\begin{flushleft}
\phantomsection\addtocounter{section}{1}
\addcontentsline{toc}{section}{Utilisation de fonctions objectives locales dans le cadre de l'optimisation topologique des échanges thermiques d'un canal vertical asymétriquement chauffé}
{\Large \textbf{Utilisation de fonctions objectives locales dans le cadre de l'optimisation topologique des échanges thermiques d'un canal vertical asymétriquement chauffé}}\label{ref:43}
\end{flushleft}
%
% Auteurs
David Marti$^{1,\star}$, Alain Bastide$^{1}$, Delphine Ramalingom$^{1}$, Pierre-Henri Cocquet$^{1}$\\[2mm]
$^{\star}$ \Letter : \url{david.marti@univ-reunion.fr}\\[2mm]
{\footnotesize $^{1}$ Université de la Réunion, laboratoire PIMENT}\\
[4mm]
%
% Mots clés
\noindent \textbf{Mots clés : } optimisation topologique mécanique des fluides convection naturelle\\[4mm]
%
% Résumé
\noindent \textbf{Résumé : } 

{\normalsize
L'économie d'énergie est un enjeu majeur des bâtiments en milieu tropical où les températures extérieures élevées imposent encore trop souvent d'utiliser des systèmes énergivores permettant d'assurer le confort des occupants d'un bâtiment.



Les présents travaux concernent les échangeurs de chaleurs passifs fonctionnant à partir du principe de siphon thermique et de ventilation naturelle. Le mur trombe est une application de génie climatique concrète de ce principe.



L'objectif est d'optimiser la forme d'un canal vertical chauffé asymétriquement dans lequel l'air s'écoule afin qu'il réponde au mieux à certains objectifs physiques tel que favoriser les échanges de chaleurs entre l'entrée et la sortie du canal. Ce problème physique est celui du mur trombe. On utilise pour cela une méthode d'optimisation topologique adjointe où l'on pénalise les équations de Navier-Stokes par un terme dit de Brinkman issue de la théorie des milieux poreux, tel que proposé par Othmer et al.



Ces travaux s'inscrivent dans la poursuite des travaux réalisés par Ramalingom et al. L'originalité apportée ici consiste à définir et à utiliser des fonctionnelles locales définies dans l'ensemble du domaine fluide. Une revue de la littérature a montré que cette approche n'a pas encore été proposée et que seuls des fonctionnelles globales sont utilisées à ce jour. Ces fonctionnelles locales permettent de venir renforcer ou atténuer localement les effets des fonctionnelles globales. Ce sont donc de nouveaux outils à la disposition du concepteur permettant d'optimiser un domaine en tenant compte de critères locaux.



L'objectif de la communication est de présenter les avantages et les perspectives d'avoir recours aux fonctionnelles locales en optimisation topologique. Les fonctionnelles globales « classiques » (définies comme la différence de quantités entre l'entrée et la sortie de l'échangeur) telles que la minimisation des pertes de charges du fluide et la maximisation des transferts de chaleurs sont reformulées en fonctionnelles locales (quantités calculées en chaque point du domaine). Les sorties du modèle sont la forme optimisée du domaine, les champs de température, et les champs de vitesse dans le domaine.

 \vfill doi : \url{https://doi.org/10.25855/SFT2021-043}

}
 
