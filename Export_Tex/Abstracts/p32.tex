

    %=====================================
    %   WARNING
    %   FICHIER AUTOMATISE
    %   NE PAS MODIFIER
    %=====================================

    \newpage

\backgroundsetup{contents={Work In Progress},scale=7}
\BgThispage
%%%%%%%%%%%%%%%%%%%%%%%%%%%%%%%%%%%%%%%%%%%%
%% Papier 32
%%%%%%%%%%%%%%%%%%%%%%%%%%%%%%%%%%%%%%%%%%%%

% Indexations
\index{LeRouxDiane@Le Roux, Diane}
\index{NeveuPierre@Neveu, Pierre}
\index{OlivesRegis@Olivès, Régis}
%
% Titre
\begin{flushleft}
\phantomsection\addtocounter{section}{1}
\addcontentsline{toc}{section}{Optimisation énergétique et environnementale d'un stockage thermique de type thermocline}
{\Large \textbf{Optimisation énergétique et environnementale d'un stockage thermique de type thermocline}}\label{ref:32}
\end{flushleft}
%
% Auteurs
Diane Le Roux$^{1,2,\star}$, Pierre Neveu$^{2}$, Régis Olivès$^{1,2}$\\[2mm]
$^{\star}$ \Letter : \url{diane.leroux@promes.cnrs.fr}\\[2mm]
{\footnotesize $^{1}$ Procédés, Matériaux et Energie Solaire (PROMES-CNRS), UPR 8521. Rambla de la Thermodynamique, 66 100 Perpignan, France}\\
{\footnotesize $^{2}$ Université de Perpignan Via Domitia, 52 av. P. Alduy, 66100 Perpignan, France}\\
[4mm]
%
% Mots clés
\noindent \textbf{Mots clés : } Stockage thermique de type thermocline, énergie solaire, chaleur fatale, analyse du cycle de vie, performances énergétiques, EcoStock®\\[4mm]
%
% Résumé
\noindent \textbf{Résumé : } 

{\normalsize
Pour le déploiement des énergies renouvelables et des systèmes industriels économes en énergie, le recours à des moyens de stockage, notamment le stockage thermique, devient une nécessité. En effet, ces derniers permettent de réguler les intermittences liées à l'utilisation d'énergies renouvelables, et de récupérer les rejets thermiques issus des procédés industriels.



L'objectif de cette étude est d'optimiser les impacts environnementaux et l'efficacité énergétique d'un réservoir thermocline. Dans un tel stockage sensible, un fluide caloporteur circule à travers un lit de particules. Pendant la phase de charge, le fluide chaud est injecté par le haut du réservoir et du fluide froid est extrait par le bas. Trois zones distinctes apparaissent : deux à température s uniformes (l'une haute et l'autre basse) séparées par une zone à fort gradient. Pendant la phase de décharge, le sens de l'écoulement du fluide est inversé.



Le réservoir industriel à optimiser a été développé et commercialisé par Eco Tech Ceram. Il utilise l'air comme fluide caloporteur et la bauxite en tant que solide de garnissage.







Pour décrire le comportement thermique du réservoir, six paramètres de conception définis par le cahier des charges et deux variables d'optimisation sans dimension sont utilisés. Ces deux variables d'optimisation, appelées facteurs de forme, sont liées à la géométrie du réservoir et la granulométrie des particules. Le modèle dynamique choisi est un modèle à une dimension et deux équations, une pour le fluide et une pour le solide. 



Pour évaluer les impacts environnementaux, l'Analyse du Cycle de Vie (ACV) utilise la cuve EcoStock® comme cuve de référence pour définir l'unité fonctionnelle : « Fournir une énergie thermique déchargée égale à l'énergie déchargée de la cuve EcoStock®, pendant sa durée de vie (25 ans) ». 







L'optimisation sur l'efficacité énergétique entraîne un choix de facteurs de forme amenant à une cuve effilée. Quant à l'optimisation en ACV, une cuve plus large et moins haute est obtenue. En optimisant simultanément l'efficacité énergétique et les impacts environnementaux selon différentes valeurs de pondération, un ensemble de Pareto est obtenu, limité par les deux optimisations monocritères. Plus le critère énergétique est privilégié, plus petit sont les diamètres de cuve et des particules. Concernant la hauteur de cuve, elle augmente avec la pondération énergétique. Ainsi, les résultats indiquent que l'ACV peut être améliorée, tout comme les performances énergétiques de la cuve thermocline.

 \vfill Work In Progress

}
 
