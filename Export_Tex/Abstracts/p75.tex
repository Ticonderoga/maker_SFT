

    %=====================================
    %   WARNING
    %   FICHIER AUTOMATISE
    %   NE PAS MODIFIER
    %=====================================

    \newpage

%%%%%%%%%%%%%%%%%%%%%%%%%%%%%%%%%%%%%%%%%%%%
%% Papier 75
%%%%%%%%%%%%%%%%%%%%%%%%%%%%%%%%%%%%%%%%%%%%

% Indexations
\index{HeyihinGeorgesA.@Heyihin, Georges A.}
\index{AwantoChristophe@Awanto, Christophe}
\index{LanzettaFrancois@Lanzetta, François}
\index{HounganComlanAristide@Houngan, Comlan Aristide}
%
% Titre
\begin{flushleft}
\phantomsection\addtocounter{section}{1}
\addcontentsline{toc}{section}{Optimisation de la machine Stirling duplex}
{\Large \textbf{Optimisation de la machine Stirling duplex}}\label{ref:75}
\end{flushleft}
%
% Auteurs
Georges A. Heyihin$^{1,\star}$, Christophe Awanto$^{1}$, François Lanzetta$^{2}$, Comlan Aristide Houngan$^{3}$\\[2mm]
$^{\star}$ \Letter : \url{gheyihin@gmail.com}\\[2mm]
{\footnotesize $^{1}$ Laboratoire d'Energétique et de Mécanique Appliquée, Université d'Abomey-Calavi}\\
{\footnotesize $^{2}$ 3FEMTO-ST, Université Bourgogne Franche-Comté, CNRS, Parc technologique, 2 avenue Jean Moulin, 90000 Belfort}\\
{\footnotesize $^{3}$ Laboratoire de Caractérisation Thermophysique des Matériaux et d'Appropriation Energétique, Université d'Abomey-Calavi}\\
[4mm]
%
% Mots clés
\noindent \textbf{Mots clés : } Stirling Duplex, optimisation, NSGA-II, MATLAB\\[4mm]
%
% Résumé
\noindent \textbf{Résumé : } 

{\normalsize
Le présent travail décrit l'optimisation d'une machine Stirling Duplex qui est une association de moteur et de récepteur Stirling, ayant un accouplement direct et partageant le même cylindre.



Le moteur Stirling est un système de conversion d'énergie à haut rendement, simple, silencieux et très fiable non polluant et nécessitant peu de maintenance. Sa compatibilité à tout type d'énergie thermique en tant que machine à apport de chaleur externe (combustion, valorisation de rejets industriels, soleil,etc.) contribue à des intérêts scientifique et industriel.



Le système de chauffage est un échangeur constitué de tubes minces parallèles afin de présenter une grande surface d'échange et de permettre l'écoulement d'un grand débit de gaz. Le régénérateur est composé d'une matrice tubulaire à mailles en fils fins. La réversibilité du cycle Stirling est utilisée pour la génération du froid ou de la chaleur. Entraînée par un moteur, la machine devient une machine frigorifique ou une pompe à chaleur.



Chaque demi-machine du Stirling Duplex (moteur ou refroidisseur) est une machine complète et comporte les principaux éléments d'une machine Stirling à savoir le réchauffeur, le régénérateur, le refroidisseur, le déplaceur et le piston



Les paramètres géométriques (diamètre du piston, les diamètres du déplaceur-moteur et du déplaceur-récepteur, et la course des déplaceurs sont déterminés par une optimisation multi-objectif utilisant l'algorithme NSGA2, implémenté sur MATLAB. Les fonctions-objectifs considérées sont : la maximisation du coefficient de performance et la minimisation de la masse de la machine.



Sous les contraintes dimensionnelles concernant les diamètres des pistons de puissance, des déplaceurs et de la course des déplaceurs, il ressort que les conditions optimales permettaient de dimensionner une machine Stirling Duplex de masse  de 17,08 kg présentant un coefficient de performance de 0,46. Les COP frigorifique et thermique sont étudiés et il en ressort que la machine Stirling Duplex est plus performante, lorsqu'elle est utilisée en pompe à chaleur qu'en extracteur thermique.

 \vfill doi : \url{https://doi.org/10.25855/SFT2021-075}

}
 
