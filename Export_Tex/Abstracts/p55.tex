

    %=====================================
    %   WARNING
    %   FICHIER AUTOMATISE
    %   NE PAS MODIFIER
    %=====================================

    \newpage

%%%%%%%%%%%%%%%%%%%%%%%%%%%%%%%%%%%%%%%%%%%%
%% Papier 55
%%%%%%%%%%%%%%%%%%%%%%%%%%%%%%%%%%%%%%%%%%%%

% Indexations
\index{LaribiAfef@Laribi, Afef}
\index{BegotSylvie@Bégot, Sylvie}
\index{LepillerValerie@Lepiller, Valérie}
\index{Ait-OumezianeYacine@Ait-Oumeziane, Yacine}
\index{DesevauxPhilippe@Désévaux, Philippe}
%
% Titre
\begin{flushleft}
\phantomsection\addtocounter{section}{1}
\addcontentsline{toc}{section}{Modélisation CFD des performances thermiques d'un mur Trombe}
{\Large \textbf{Modélisation CFD des performances thermiques d'un mur Trombe}}\label{ref:55}
\end{flushleft}
%
% Auteurs
Afef Laribi$^{1,\star}$, Sylvie Bégot$^{1}$, Valérie Lepiller$^{1}$, Yacine Ait-Oumeziane$^{1}$, Philippe Désévaux$^{1}$\\[2mm]
$^{\star}$ \Letter : \url{afef.laribi@univ-fcomte.fr}\\[2mm]
{\footnotesize $^{1}$ FEMTO-ST Institute, Univ. Bourgogne Franche-Comté, CNRS}\\
[4mm]
%
% Mots clés
\noindent \textbf{Mots clés : } Mur Trombe, CFD, énergies renouvelables, convection naturelle, transfert thermique conjugué\\[4mm]
%
% Résumé
\noindent \textbf{Résumé : } 

{\normalsize
Dans le monde et particulièrement en France, le secteur du bâtiment est un gros consommateur d'énergie. Une part importante de cette énergie est en effet utilisée pour assurer le confort hygrothermique de l'occupant. Afin de limiter la consommation d'énergie carbonée et de limiter les émissions de gaz à effet de serre, les réglementations énergétiques incitent au développement de  sources d'énergies propres.







Dans ce contexte, les systèmes solaires passifs apparaissent comme une alternative intéressante pour répondre aux besoins de chauffage, de ventilation et de climatisation en permettant d'économiser jusqu'à 30 \% de l'énergie consommée. L'une des techniques passives les plus performantes réside dans l'utilisation de murs Trombe. 







Un mur Trombe est un système composé, de l'extérieur vers l'intérieur, d'un vitrage séparé d'une paroi stockeuse opaque par une lame d'air ventilée. Afin de favoriser les échanges thermiques convectifs, des ouïes sont installées en parties haute et basse de la paroi stockeuse. Son intégration au bâtiment vise à valoriser le rayonnement solaire en associant deux phénomènes physiques : l'effet de serre à travers un vitrage et l'inertie thermique du mur. 







Cependant, cette technique, performante en saison hivernale, présente des inconvénients en été car elle engendre une surchauffe. C'est pourquoi, des modifications doivent être mises en place afin d'adapter le fonctionnement de ce système indépendamment des conditions climatiques. 







Notre étude consiste à réaliser un modèle numérique afin de simuler le comportement thermique d'un mur Trombe. Le comportement du mur Trombe en régime permanent est étudié via un modèle CFD en deux dimensions prenant en compte les transferts par conduction, convection et par rayonnement ainsi que la turbulence de l'écoulement d'air. 







Après une validation des simulations à partir de résultats issus de la littérature, une première étude paramétrique concernant l'influence de la largeur des ouïes et de l'épaisseur de la lame d'air est proposée. Les performances thermiques du mur Trombe sont analysées numériquement par le biais de plusieurs grandeurs (température au niveau au niveau de la lame d'air, des ouïes, de l'intérieur du local, vitesse au niveau de la lame d'air) décrivant les aspects thermo-aérauliques locaux et à travers la notion d'efficacité thermique du système.

 \vfill doi : \url{https://doi.org/10.25855/SFT2021-055}

}
 
