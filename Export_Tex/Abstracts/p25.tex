

    %=====================================
    %   WARNING
    %   FICHIER AUTOMATISE
    %   NE PAS MODIFIER
    %=====================================

    \newpage

%%%%%%%%%%%%%%%%%%%%%%%%%%%%%%%%%%%%%%%%%%%%
%% Papier 25
%%%%%%%%%%%%%%%%%%%%%%%%%%%%%%%%%%%%%%%%%%%%

% Indexations
\index{CressinMaxime@Cressin, Maxime}
\index{LochegniesDominique@Lochegnies, Dominique}
\index{NaceurHakim@Naceur, Hakim}
\index{BechetFabien@Béchet, Fabien}
\index{MoreauPhilippe@Moreau, Philippe}
\index{BoukhariNadir@Boukhari, Nadir}
%
% Titre
\begin{flushleft}
\phantomsection\addtocounter{section}{1}
\addcontentsline{toc}{section}{Influence des échanges radiatifs/convectifs sur l'étalement d'une goutte de verre}
{\Large \textbf{Influence des échanges radiatifs/convectifs sur l'étalement d'une goutte de verre}}\label{ref:25}
\end{flushleft}
%
% Auteurs
Maxime Cressin$^{1}$, Dominique Lochegnies$^{1}$, Hakim Naceur$^{1,\star}$, Fabien Béchet$^{1}$, Philippe Moreau$^{1}$, Nadir Boukhari$^{2}$\\[2mm]
$^{\star}$ \Letter : \url{Hakim.Naceur@uphf.fr}\\[2mm]
{\footnotesize $^{1}$ LAMIH}\\
{\footnotesize $^{2}$ SAVERGLASS}\\
[4mm]
%
% Mots clés
\noindent \textbf{Mots clés : } rayonnement verre Polyflow convection formage\\[4mm]
%
% Résumé
\noindent \textbf{Résumé : } 

{\normalsize
Le verre est un matériau semi-transparent dont le comportement mécanique varie fortement en fonction de la température. Dans le cas de la mise en forme du verre, et plus particulièrement lors du formage de bouteille via le procédé soufflé-soufflé, la température du verre varie de 1100$^{\circ}$C à de 600$^{\circ}$C environ. Il est donc nécessaire pour simuler efficacement ces opérations de mises en forme du verre de déterminer avec précision les échanges thermiques au sein du verre. Parmi ces échanges, les effets de convection et de rayonnement influent fortement sur la température et donc sur le comportement mécanique du verre. La prise en compte du rayonnement dans la simulation numérique est complexe et implique des temps de simulations plus importants. De plus, il est nécessaire de déterminer les propriétés optiques spécifiques du verre étudié à l'aide d'essais expérimentaux. En effet, des propriétés telles que l'indice de réfraction du verre ou le coefficient d'absorption sont dépendants de la composition du verre et de sa teinte. 



La simulation d'une chute de goutte de verre sur une plaque d'acier a été choisie afin d'étudier l'influence des échanges thermiques sur la géométrie et la température de la goutte de verre lors de l'étalement de cette dernière. Pour simuler ce problème, le logiciel Polyflow a été retenu en utilisant une hypothèse de modélisation 2D-axisymétrique pour simplifier le modèle et ainsi réduire le temps de calcul. Ce modèle sera alimenté à l'aide de données expérimentales fournies par Saverglass et de données bibliographiques. Plusieurs simulations numériques ont été réalisées dans différentes conditions de vitesse initiale de chute de la goutte. La première étude vise à étudier l'influence du rayonnement en utilisant les modèles de rayonnement mis à disposition par Polyflow qui sont la Méthode des Ordonnées Discretes et l'approximation de Rosseland. Ensuite, il s'agit d'étudier l'influence des effets convectifs en faisant varier l'intensité des échanges convectifs. Enfin, il s'agit de comparer les données numériques à des données expérimentales réalisées par Saverglass.

 \vfill doi : \url{https://doi.org/10.25855/SFT2021-025}

}
 
