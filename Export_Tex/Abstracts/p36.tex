

    %=====================================
    %   WARNING
    %   FICHIER AUTOMATISE
    %   NE PAS MODIFIER
    %=====================================

    \newpage

%%%%%%%%%%%%%%%%%%%%%%%%%%%%%%%%%%%%%%%%%%%%
%% Papier 36
%%%%%%%%%%%%%%%%%%%%%%%%%%%%%%%%%%%%%%%%%%%%

% Indexations
\index{BraccioSimone@Braccio, Simone}
\index{PhanHaiTrieu@Phan, Hai Trieu}
\index{TauveronNicolas@Tauveron, Nicolas}
\index{LePierresNolwenn@Le Pierrès, Nolwenn}
%
% Titre
\begin{flushleft}
\phantomsection\addtocounter{section}{1}
\addcontentsline{toc}{section}{Procédé de cogénération de froid et électricité à partir d'une source de chaleur basse température}
{\Large \textbf{Procédé de cogénération de froid et électricité à partir d'une source de chaleur basse température}}\label{ref:36}
\end{flushleft}
%
% Auteurs
Simone Braccio$^{1,\star}$, Hai Trieu Phan$^{2}$, Nicolas Tauveron$^{2}$, Nolwenn Le Pierrès$^{3}$\\[2mm]
$^{\star}$ \Letter : \url{simone.braccio@cea.fr}\\[2mm]
{\footnotesize $^{1}$ Univ. Grenoble Alpes, CEA, LITEN, DTBH. F-38000 Grenoble, France ; Laboratoire LOCIE, Université Savoie Mont Blanc CNRS UMR 5271, 73376 Le Bourget Du Lac, France}\\
{\footnotesize $^{2}$ Univ. Grenoble Alpes, CEA, LITEN, DTBH. F-38000 Grenoble, France}\\
{\footnotesize $^{3}$ Laboratoire LOCIE, Université Savoie Mont Blanc CNRS UMR 5271, 73376 Le Bourget Du Lac, France}\\
[4mm]
%
% Mots clés
\noindent \textbf{Mots clés : } Cogénération ; Absorption; eau / ammoniac, expandeur; Turbine axiale à action;\\[4mm]
%
% Résumé
\noindent \textbf{Résumé : } 

{\normalsize
Compte tenu de la demande mondiale d'énergie toujours croissante et de l'attention à porter aux problèmes environnementaux, la communauté scientifique se concentre de plus en plus sur la recherche de nouvelles technologies de conversion d'énergie plus efficaces basées sur des sources renouvelables ou de récupération. Dans ce contexte, les systèmes énergétiques à absorption, jusqu'ici des technologies relativement de niches, se prêtent bien à la valorisation d'énergie à basse température pour la production de froid et de travail mécanique.



Le présent travail se concentre sur un système de cogénération (froid et électricité) à basse température basé sur une machine à absorption eau/ammoniac. L'étude s'appuie en partie sur un pilote expérimental de puissance thermique du générateur de 10 kW pouvant fonctionner en cogénération, actuellement partiellement opérationnel. Dans une étude précédente, l'expandeur, qui utilise la vapeur pressurisée principalement constituée d'ammoniac du cycle et qui avait été initialement sélectionné (un détendeur volumétrique de type scroll) s'est montré inapproprié pour une application de si petite taille en raison d'un débit de fuite trop élevé. Une technologie différente de production d'électricité a donc été investiguée dans la présente étude, celle d'un turbo-expandeur à action à admission partielle. Le fait que dans ce type d'expandeur toute la détente a lieu dans le distributeur devrait limiter l'influence des fuites et garantir un bon niveau de production de travail.







Un modèle 1D compressible (avec loi de gaz réel) du turbo-expandeur axial a été développé. Le distributeur du turbo-expandeur étant une tuyère convergente-divergente, l'influence des conditions d'entrée du fluide détendu sur ses performances énergétiques et sur le débit traité par celui-ci a tout d'abord été étudiée. Ensuite, après évaluation des principaux facteurs de perte au point de dessin de la turbine, la performance énergétique de la turbomachine a été déterminée en conditions hors nominales. Cette analyse a été réalisée pour différentes vitesses de rotation de la turbine, déterminant la plage de fonctionnement optimale pour différentes conditions d'entrée fixées par le cycle d'absorption et par le surchauffeur. Pour des conditions de 16 bar et 120 $^{\circ}$C en entrée et 4 bar en sortie, le modèle intégré au cycle permet d'obtenir une puissance mécanique maximale de 750 W (rendement isentropique autour de 52 \%) à 70000 RPM et un débit réfrigérant de 23.5 kg/h. Ce résultat est cohérent avec les performances attendues pour une turbine à action de petite taille de ce type.











Ensuite, le comportement de l'expandeur intégré dans le cycle global de cogénération a été étudié pour optimiser les productions d'électricité et de froid. En particulier, la relation entre le débit destiné à la production d'électricité et celui destiné à la partie du cycle produisant le froid (condenseur/évaporateur) a été analysée, en fonction de la puissance du générateur et des conditions de fonctionnement tout en tenant compte des limites imposées par la turbine. Cette étude apporte de premières informations sur l'intégration de ce type d'expandeur dans un cycle thermodynamique de cogénération tritherme à absorption eau/ammoniac, dans l'attente de résultats expérimentaux du pilote.

 \vfill doi : \url{https://doi.org/10.25855/SFT2021-036}

}
 
