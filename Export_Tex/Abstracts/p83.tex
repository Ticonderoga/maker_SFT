

    %=====================================
    %   WARNING
    %   FICHIER AUTOMATISE
    %   NE PAS MODIFIER
    %=====================================

    \newpage

\backgroundsetup{contents={Work In Progress},scale=7}
\BgThispage
%%%%%%%%%%%%%%%%%%%%%%%%%%%%%%%%%%%%%%%%%%%%
%% Papier 83
%%%%%%%%%%%%%%%%%%%%%%%%%%%%%%%%%%%%%%%%%%%%

% Indexations
\index{KovcharJean@Kovchar, Jean}
\index{BlidiaAbdelhamid@Blidia, Abdelhamid}
\index{BarthesMagali@Barthès, Magali}
\index{LanzettaFrancois@Lanzetta, François}
\index{DeLabachelerieMichel@De Labachelerie, Michel}
%
% Titre
\begin{flushleft}
\phantomsection\addtocounter{section}{1}
\addcontentsline{toc}{section}{Study of permanent and alternate gas flow in microchannels}
{\Large \textbf{Study of permanent and alternate gas flow in microchannels}}\label{ref:83}
\end{flushleft}
%
% Auteurs
Jean Kovchar$^{1,\star}$, Abdelhamid Blidia$^{2}$, Magali Barthès$^{1}$, François Lanzetta$^{1}$, Michel De Labachelerie$^{1}$\\[2mm]
$^{\star}$ \Letter : \url{jean.kovchar@femto-st.fr}\\[2mm]
{\footnotesize $^{1}$ FEMTO-ST}\\
{\footnotesize $^{2}$ UFR-STGI}\\
[4mm]
%
% Mots clés
\noindent \textbf{Mots clés : } Microchannel, microfabrication, gas flow, Stirling\\[4mm]
%
% Résumé
\noindent \textbf{Résumé : } 

{\normalsize
Recent progress in the last decades in microfabrication technology led to a growing interest in miniaturized devices. With the dimension getting smaller, knowledge of the phenomena occurring in milli- and microchannels is required. This knowledge will allow designing properly the devices to optimize their performances. Previous works on the miniaturization of a Stirling engine were carried out in our institute. They have highlighted at these micro-scales some thermal issues, the importance of minor and major pressure losses and the difficulties to understand the alternating flows. In the literature, there are indeed very few results for alternating flows. Even in the case of permanent flows (i.e. non-alternating flow) which were more studied, literature's results show contradictions between the different authors, making the influence of relevant parameters unclear on the flows at this scale.







Thus, the objectives of the present study are to investigate permanent gas flow in microchannels (with and without wall temperature gradient). The aim is to obtain reference data that will be used, in the next step of our work, on alternate gas flows. The influence of geometrical parameters such as hydraulic diameter, length of the channel and aspect ratio (ratio of the width to the height of the channel) will be investigated, as well as the influence of the compressibility of the fluid. Moreover, in order to investigate on the minor losses in microchannels, we have considered two different channel designs: straight channels and channels with bends at 90$^{\circ}$.







Numerical investigations of gas flows in our micro-channels have been carried out for both isothermal and non-isothermal conditions. Correlations for the friction factor are established and compared with experimental results from the literature.







In addition to this numerical part, an experimental setup is being built to allow both studies for permanent and alternate gas flows. Pressure and temperature sensors at the inlet and outlet of the micro-channels will provide global measurements, and a mass flow sensor will measure the flow rate. The microchannels are fabricated in the MIMENTO clean room facilities of the FEMTO-ST Institute. They are designed with integrated temperature and pressure sensors that will provide local measurements inside the micro-channel.

 \vfill Work In Progress

}
 
