

    %=====================================
    %   WARNING
    %   FICHIER AUTOMATISE
    %   NE PAS MODIFIER
    %=====================================

    \newpage

\backgroundsetup{contents={Work In Progress},scale=7}
\BgThispage
%%%%%%%%%%%%%%%%%%%%%%%%%%%%%%%%%%%%%%%%%%%%
%% Papier 85
%%%%%%%%%%%%%%%%%%%%%%%%%%%%%%%%%%%%%%%%%%%%

% Indexations
\index{HauggAlbertThomas@Haugg, Albert Thomas}
%
% Titre
\begin{flushleft}
\phantomsection\addtocounter{section}{1}
\addcontentsline{toc}{section}{Logiciel pour optimisation des échangeurs à plaques avec variabilité et adaptabilité augmentées}
{\Large \textbf{Logiciel pour optimisation des échangeurs à plaques avec variabilité et adaptabilité augmentées}}\label{ref:85}
\end{flushleft}
%
% Auteurs
Albert Thomas Haugg$^{1,\star}$\\[2mm]
$^{\star}$ \Letter : \url{Albert.T.Haugg@akg-gruppe.de}\\[2mm]
{\footnotesize $^{1}$ Head of New Developments, AKG Thermotechnik International}\\
[4mm]
%
% Mots clés
\noindent \textbf{Mots clés : } Echangeurs optimisés-Réduction des émissions $\unit{CO_2}$-Consommation d'énergie minimisée\\[4mm]
%
% Résumé
\noindent \textbf{Résumé : } 

{\normalsize
La contribution de l'industrie des échangeurs de chaleur et des radiateurs à l'accord de Paris pour la réduction des émissions $\unit{CO_2}$ consiste principalement en l'augmentation de leur efficacité, d'une part en améliorant la transmission spécifique d'un échangeur et, d'autre part, en réduisant la perte de pression, ce qui engendre automatiquement une baisse de la consommation d'énergie pour cette transmission. L'idéal serait d'atteindre ces deux objectifs simultanément.



Les caractéristiques physiques d'un liquide en changement de phase varient substantiellement. Un échangeur doit par conséquent respecter ces modifications, ce qui n'est pas le cas pour les échangeurs à plaques largement en utilisation, si la source chaude ou froide pour l'évaporation ou la condensation est un liquide.



La société AKG développe actuellement un échangeur à plaques SSC (Stacked Shell Cooler / échangeur à plaques empilé) qui offre ces possibilités : plusieurs passages, élargissement ou réduction du volume par passage pour le réfrigérant avec également la faculté d'adaptation aux exigences de l'application en débit et perte de pression. De plus, et ceci diffère des échangeurs à plaques en inox, les structures internes (turbulateurs) sont échangeables. Ceci permet d'augmenter la transmission spécifique et de réduire la perte de pression.



L'adaptabilité a été présentée l'année passée lors du congrès « IIR International Conference Rankine 2020 – Applications of Cooling, Heating and Power Generation » avec des premiers résultats et actuellement un logiciel est en développement pour établir un mode de calcul dont l'objectif est l'optimisation de ces échangeurs avec leur complexité significativement accrue. Les buts essentiels consistent à augmenter le coefficient d'échange spécifique et par cela réduire le volume et les dimensions ainsi que la perte de pression, ce qui signifie une consommation d'énergie minimisée.



La variabilité permet a) d'avoir différents turbulateurs (structures internes) pour les deux fluides mais aussi d'utiliser divers modèles de turbulateurs dans les différentes couches pour un fluide - il est même pensable de changer la structure interne dans une seule couche - b) de configurer le passage dans une couche (en plan horizontal) en I, X ou U - c) de choisir entre uni- ou multi-passage et cela pour les deux fluides séparément et indépendamment - d) de distribuer le nombre des couches librement pour les deux fluides.



Le développement d'un logiciel qui permettra d'optimiser toutes ces possibilités se fera par étape avec plusieurs itérations nécessaires. D'abord, il s'agit de déterminer l'équation la plus appropriée pour une seule couche avec différents turbulateurs et différents passages. 



Cette adaptabilité accrue signifie un potentiel d'amélioration substantiel pour ces échangeurs et engendre une augmentation significative en rendement dans les applications pompe à chaleur, machine à réfrigérer, Cycle de Rankine Organique (CRO) ou autre,  ce qui accélère l'évolution de ces technologies tout comme leur propagation.  Ceci permet non seulement l'utilisation plus efficace de l'énergie mais aussi partiellement la réutilisation des chaleurs fatales  et contribue aux réductions des émissions $\unit{CO_2}$ dans notre industrie.

 \vfill Work In Progress

}
 
