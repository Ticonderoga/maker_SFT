

    %=====================================
    %   WARNING
    %   FICHIER AUTOMATISE
    %   NE PAS MODIFIER
    %=====================================

    \newpage

%%%%%%%%%%%%%%%%%%%%%%%%%%%%%%%%%%%%%%%%%%%%
%% Papier 2
%%%%%%%%%%%%%%%%%%%%%%%%%%%%%%%%%%%%%%%%%%%%

% Indexations
\index{OlayaG.RodrigoA.@Olaya G., Rodrigo A.}
\index{GarnierBertrand@Garnier, Bertrand}
\index{MishraKetaki@Mishra, Ketaki}
%
% Titre
\begin{flushleft}
\phantomsection\addtocounter{section}{1}
\addcontentsline{toc}{section}{Caractérisation de la conductivité thermique radiale d'une fibre de carbone de type PAN par la méthode 3$\omega$}
{\Large \textbf{Caractérisation de la conductivité thermique radiale d'une fibre de carbone de type PAN par la méthode 3$\omega$}}\label{ref:2}
\end{flushleft}
%
% Auteurs
Rodrigo A. Olaya G.$^{1,\star}$, Bertrand Garnier$^{1}$, Ketaki Mishra$^{1}$\\[2mm]
$^{\star}$ \Letter : \url{raolayag@unal.edu.co}\\[2mm]
{\footnotesize $^{1}$ Laboratoire de Thermique et Energie de Nantes (LTeN)  - Polytech Nantes}\\
[4mm]
%
% Mots clés
\noindent \textbf{Mots clés : } conductivité thermique radiale ; fibre de carbone ; méthode 3omega, quadrupoles\\[4mm]
%
% Résumé
\noindent \textbf{Résumé : } 

{\normalsize
La prédiction des transferts de chaleur dans les matériaux composites (époxy/fibre) nécessite la connaissance de leurs propriétés thermiques effectives. Dans un premier temps, il est important de mesurer la conductivité thermique axiale ainsi que radiale des fibres en l'occurrence en carbone et de type PAN. Contrairement à la conductivité thermique axiale des fibres de carbone, il n'y a pas beaucoup de travaux sur la mesure de leurs conductivités thermiques radiales. Le présent travail décrit la caractérisation de la conductivité thermique des fibres de carbone dans la direction radiale en utilisant la méthode 3$\omega$ avec une source de courant constante. Cette méthode utilise une seule fibre qui agit à la fois comme source de chaleur et comme capteur de température. La fibre de carbone est entourée d'un matériau à forte effusivité thermique pour améliorer le transfert de chaleur radial et donc a permis d'augmenter la sensibilité de la tension 3$\omega$ mesurée à la conductivité radiale recherchée. Un modèle thermique approprié est nécessaire pour estimer la conductivité thermique radiale. Par conséquent, un modèle analytique 2D décrivant les transferts thermiques dans la fibre en régime périodique établi a été développé en utilisant la méthode des quadripoles et a été validé à l'aide d'un modèle numérique 2D. Un modèle radial 1D a été mis au point afin d'analyser l'effet du transfert de chaleur axial résiduel à l'intérieur de la fibre pendant la mesure. Une étude de sensibilité détaillée du paramètre inconnu a été effectuée afin de trouver la meilleure plage de conditions opératoires en particulier la plage de fréquences et le type de matériau environnant le plus pertinent. Les mesures ont été effectuées avec de la fibre de carbone de type PAN (BT300) de 6 à 8 micromètres de diamètre et avec différentes longueurs de 0,5 à 2,5 mm. De l'eau dé-ionisée a été utilisée comme matériau environnant. Enfin, les valeurs de conductivité thermique radiale obtenues se sont avérées beaucoup plus petites que les valeurs axiales précédemment mesurées, typiquement plus d'un ordre de grandeur inférieur.

 \vfill doi : \url{https://doi.org/10.25855/SFT2021-002}

}
 
