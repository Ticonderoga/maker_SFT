

    %=====================================
    %   WARNING
    %   FICHIER AUTOMATISE
    %   NE PAS MODIFIER
    %=====================================

    \newpage

%%%%%%%%%%%%%%%%%%%%%%%%%%%%%%%%%%%%%%%%%%%%
%% Papier 12
%%%%%%%%%%%%%%%%%%%%%%%%%%%%%%%%%%%%%%%%%%%%

% Indexations
\index{BenElMekkiInes@Ben El Mekki, Ines}
\index{AndrichMarine@Andrich, Marine}
\index{WagnerMarc@Wagner, Marc}
\index{GiraudFlorine@Giraud, Florine}
\index{TremeacBrice@Tréméac, Brice}
\index{TobalyPascal@Tobaly, Pascal}
%
% Titre
\begin{flushleft}
\phantomsection\addtocounter{section}{1}
\addcontentsline{toc}{section}{Vaporisation ascendante d'un mélange binaire d'HFC dans un passage d'ailettes décalées}
{\Large \textbf{Vaporisation ascendante d'un mélange binaire d'HFC dans un passage d'ailettes décalées}}\label{ref:12}
\end{flushleft}
%
% Auteurs
Ines Ben El Mekki$^{1,\star}$, Marine Andrich$^{2}$, Marc Wagner$^{2}$, Florine Giraud$^{3}$, Brice Tréméac$^{3}$, Pascal Tobaly$^{3}$\\[2mm]
$^{\star}$ \Letter : \url{ines.benelmekki@airliquide.com}\\[2mm]
{\footnotesize $^{1}$ Doctorante AIR LIQUIDE / LAFSET CNAM}\\
{\footnotesize $^{2}$ AIR LIQUIDE}\\
{\footnotesize $^{3}$ LAFSET CNAM}\\
[4mm]
%
% Mots clés
\noindent \textbf{Mots clés : } ailettes décalées, vaporisation ascendante, fluide binaire\\[4mm]
%
% Résumé
\noindent \textbf{Résumé : } 

{\normalsize
L'utilisation des échangeurs à plaques et ailettes a été élargie aux procédés cryogéniques tel que la liquéfaction du gaz naturel et aux systèmes de réfrigération. Les ailettes les plus utilisées dans ces évaporateurs sont les ailettes décalées car elles favorisent l'homogénéisation de l'écoulement et empêchent l'établissement des couches limites thermiques afin de favoriser la turbulence et améliorer ainsi le transfert thermique. Bien que l'ébullition des fluides purs dans des géométries à ailettes décalées ait été intensivement étudiée dans la littérature, la combinaison entre les fluides binaires et les ailettes décalées n'a pas été largement investiguée. 



Cette présente étude vise ainsi à comprendre et analyser expérimentalement le couplage entre les différents phénomènes physiques ayant lieu lors de la vaporisation ascendante d'un mélange binaire dans un passage à ailettes décalées. Plusieurs aspects ont été examinés simultanément. Le premier aspect concerne l'hydrodynamique de l'écoulement diphasique en analysant l'impact des ailettes décalées sur les cartes et les régimes d'écoulement ainsi que la perte de charges diphasique. Le deuxième aspect consiste à l'étude thermodynamique de la vaporisation des mélanges binaires. Le dernier aspect concerne le transfert thermique en analysant l'impact des fluides binaires sur les mécanismes de transfert de chaleur.     



Afin d'étudier le couplage entre ces différents aspects, un banc expérimental a été conçu et fabriqué pour étudier la vaporisation ascendante d'un mélange binaire équimolaire de R-134a et R-245fa. La pression opératoire dans la section d'étude est comprise entre 2 et 5 bar absolu, la vitesse massique est comprise entre 11 et 35~$\unit{kg.m^{-2}.s^{-1}}$ et le flux thermique entre 1 et 29~$\unit{kW.m^{-2}}$.



La section d'étude est composée d'un seul passage d'ailettes décalées dont la densité est de 826 ailettes par mètre. Les sections de passage sont rectangulaires avec une surface de 7~$\unit{mm^2}$ et de diamètre hydraulique de 1,75 mm. Ce passage est compris entre une plaque de polycarbonate capable de visualiser l'écoulement diphasique et un bloc d'aluminium chauffé électriquement. Une résistance chauffante en film adhésif est collée à l'arrière du vaporiseur afin de fournir la chaleur nécessaire à la vaporisation du mélange binaire. 



Le coefficient de transfert thermique local est calculé à partir des mesures de flux thermiques et des températures dans le fluide et dans la paroi. Un capteur de pression différentiel permet de mesurer la perte de charge totale obtenue dans la section d'essai. Les résultats expérimentaux du coefficient de transfert thermique ainsi que la perte de charge sont comparés avec des corrélations trouvées dans la littérature.

 \vfill doi : \url{https://doi.org/10.25855/SFT2021-012}

}
 
