

    %=====================================
    %   WARNING
    %   FICHIER AUTOMATISE
    %   NE PAS MODIFIER
    %=====================================

    \newpage

%%%%%%%%%%%%%%%%%%%%%%%%%%%%%%%%%%%%%%%%%%%%
%% Papier 40
%%%%%%%%%%%%%%%%%%%%%%%%%%%%%%%%%%%%%%%%%%%%

% Indexations
\index{MahamoudouArnat@Mahamoudou, Arnat}
\index{LePierresNolwenn@Le Pierrès, Nolwenn}
\index{RamousseJulien@Ramousse, Julien}
%
% Titre
\begin{flushleft}
\phantomsection\addtocounter{section}{1}
\addcontentsline{toc}{section}{Analyse du profil de température et de création d'entropie dans un évaporateur à film tombant}
{\Large \textbf{Analyse du profil de température et de création d'entropie dans un évaporateur à film tombant}}\label{ref:40}
\end{flushleft}
%
% Auteurs
Arnat Mahamoudou$^{1,\star}$, Nolwenn Le Pierrès$^{1}$, Julien Ramousse$^{1}$\\[2mm]
$^{\star}$ \Letter : \url{arnat.mahamoudou@univ-smb.fr}\\[2mm]
{\footnotesize $^{1}$ Laboratoire LOCIE, CNRS UMR5271 – Université Savoie Mont-Blanc, 73370 Le Bourget-du-Lac, France}\\
[4mm]
%
% Mots clés
\noindent \textbf{Mots clés : } Echangeur de chaleur, Film tombant, Profil de température, Création d'entropie\\[4mm]
%
% Résumé
\noindent \textbf{Résumé : } 

{\normalsize
Les échangeurs de chaleur étant des composants clé de tout système de réfrigération et de chauffage, l'amélioration des transferts thermiques qui leur sont associés est importante pour obtenir une plus grande efficacité. Ce travail se focalise sur un évaporateur à film tombant qui peut être utilisé dans une machine à sorption. Un film laminaire de fluide incompressible s'écoulant sous l'effet de la gravité sur une plaque verticale chauffée au moyen d'un fluide caloporteur en co-courant est étudié. Ce film est le siège de phénomènes thermiques conduisant à une évaporation à la surface libre, supposée à température de saturation, constante le long du film. Afin d'identifier les leviers à l'échelle locale permettant d'améliorer les performances à l'échelle du composant, cette étude vise à déterminer le champ de température et les créations d'entropie locales en appliquant la Thermodynamique des Processus Irréversibles. Le profil de température est obtenu en résolvant l'équation de la chaleur par la méthode des différences finies, en supposant un profil de Nusselt pour la vitesse du fluide sous les hypothèses suivantes : régime stationnaire, phénomènes de conduction négligés dans le sens de l'écoulement, épaisseur et propriétés physiques du film constant. L'étude a été réalisée pour des nombres de Reynolds du fluide caloporteur (Refc) allant de 45 à 360 et pour un nombre de Reynolds du film tombant de l'ordre de 13. Les résultats démontrent que la masse évaporée est impactée par le débit du fluide caloporteur, la longueur du film et  la différence de température entre l'entrée du film tombant et la température d'évaporation (film surchauffé). En effet, lorsque la surchauffe du film en entrée est nulle, le débit évaporé représente 1.74\% du débit du film pour Refc = 45 et 2.93\% pour Refc = 360 sur 0.1 mètre de longueur. De plus, l'augmentation de la différence de température entre l'entrée du film tombant et la température d'évaporation (écart à l'équilibre thermique) conduit à un développement plus rapide de la couche limite thermique et donc, à un débit évaporé plus élevé. Dans le cas d'une surchauffe de 4.5K, il y'a une augmentation du débit évaporé de 0.46\% comparé au cas sans surchauffe pour un Refc = 360 sur 0.1 mètre. La création d'entropie du film étudié est due à la création d'entropie d'origine thermique et à la création d'entropie d'origine visqueuse due aux frottements. La création d'entropie visqueuse est plus grande au niveau de la plaque verticale et diminue à mesure que l'on se rapproche de la surface libre. Toutefois, cette création reste négligeable – ordre de $\unit{10^{-6}\ W/(K.m^3)}$ – devant la création d'entropie thermique – ordre de 1 à 500 $\unit{W/(K.m^3)}$. La création d'entropie totale est impactée par le débit et la conductivité des deux fluides ainsi que la température d'entrée du film tombant.

 \vfill doi : \url{https://doi.org/10.25855/SFT2021-040}

}
 
