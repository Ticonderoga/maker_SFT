

    %=====================================
    %   WARNING
    %   FICHIER AUTOMATISE
    %   NE PAS MODIFIER
    %=====================================

    \newpage

\backgroundsetup{contents={Work In Progress},scale=7}
\BgThispage
%%%%%%%%%%%%%%%%%%%%%%%%%%%%%%%%%%%%%%%%%%%%
%% Papier 82
%%%%%%%%%%%%%%%%%%%%%%%%%%%%%%%%%%%%%%%%%%%%

% Indexations
\index{ChouderRyma@Chouder, Ryma}
\index{StouffsPascal@Stouffs, Pascal}
\index{BenabdesselamAzzedine@Benabdesselam, Azzedine}
%
% Titre
\begin{flushleft}
\phantomsection\addtocounter{section}{1}
\addcontentsline{toc}{section}{Etude d'un moteur Ericsson à piston liquide libre.}
{\Large \textbf{Etude d'un moteur Ericsson à piston liquide libre.}}\label{ref:82}
\end{flushleft}
%
% Auteurs
Ryma Chouder$^{1}$, Pascal Stouffs$^{1,\star}$, Azzedine Benabdesselam$^{2}$\\[2mm]
$^{\star}$ \Letter : \url{pascal.stouffs@univ-pau.fr}\\[2mm]
{\footnotesize $^{1}$ Universite de Pau et des Pays de l'Adour, E2S UPPA, LaTEP}\\
{\footnotesize $^{2}$ 2Laboratoire des Transports Polyphasiques et Milieux Poreux (LTPMP), FGPGM, USTHB}\\
[4mm]
%
% Mots clés
\noindent \textbf{Mots clés : } Moteur Ericsson, moteur à piston libre, piston liquide\\[4mm]
%
% Résumé
\noindent \textbf{Résumé : } 

{\normalsize
Les moteurs à piston libre, généralement couplés à un générateur linéaire, sont des systèmes de conversion d'énergie très intéressants. Parmi cette famille, les moteurs Stirling à piston libre peuvent être distingués des moteurs à piston libre à combustion interne. Les moteurs Stirling à piston libre peuvent fonctionner avec différents types de sources de chaleur comme l'énergie solaire, l'énergie nucléaire ou la combustion de la biomasse. Une configuration spéciale de moteurs Stirling à piston libre utilisant des pistons liquides est connue sous le nom de Fluidyne. Ce dispositif est constitué de deux tubes en U. Les Fluidynes sont technologiquement simples mais ils fonctionnent habituellement à basse fréquence et sont généralement limités à des applications de pompage de faible puissance, avec un rendement médiocre (généralement inférieur à 1\%). Une nouvelle configuration d'un moteur à piston liquide libre consistant en un seul tube en U équipé de soupapes est présentée. Ce moteur thermique fait partie de la famille des moteurs Ericsson. Un modèle dynamique de ce moteur Ericsson à piston liquide libre a été développé dans l'environnement MATLAB/SIMULINK. Ce modèle est inspiré de la modélisation dynamique et thermodynamique des moteurs Stirling à piston libre et de la modélisation des écoulements dans les soupapes des moteurs à combustion interne. Le modèle permet de déterminer la fréquence de fonctionnement du moteur, la position instantanée du piston liquide et les propriétés instantanées du gaz de travail, de sorte que les performances globales du moteur peuvent être prédites. En exploitant ce modèle, il a été possible de déterminer un jeu de paramètres de dimensionnement et de conduite du système permettant un fonctionnement stable, et dont les performances énergétiques sont intéressantes. Un banc expérimental est en cours de montage pour confirmer les résultats de modélisation. Les choix technologiques et l'état d'avancement de la réalisation du banc sont présentés.

 \vfill Work In Progress

}
 
