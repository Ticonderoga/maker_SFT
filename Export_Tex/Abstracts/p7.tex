

    %=====================================
    %   WARNING
    %   FICHIER AUTOMATISE
    %   NE PAS MODIFIER
    %=====================================

    \newpage

%%%%%%%%%%%%%%%%%%%%%%%%%%%%%%%%%%%%%%%%%%%%
%% Papier 7
%%%%%%%%%%%%%%%%%%%%%%%%%%%%%%%%%%%%%%%%%%%%

% Indexations
\index{EustacheJulien@Eustache, Julien}
\index{PlaitAntony@Plait, Antony}
\index{DubasFrederic@Dubas, Frédéric}
\index{GlisesRaynal@Glises, Raynal}
%
% Titre
\begin{flushleft}
\phantomsection\addtocounter{section}{1}
\addcontentsline{toc}{section}{Étude bibliographique des dispositifs expérimentaux pour la réfrigération magnétique}
{\Large \textbf{Étude bibliographique des dispositifs expérimentaux pour la réfrigération magnétique}}\label{ref:7}
\end{flushleft}
%
% Auteurs
Julien Eustache$^{1,\star}$, Antony Plait$^{1}$, Frédéric Dubas$^{1}$, Raynal Glises$^{1}$\\[2mm]
$^{\star}$ \Letter : \url{julien.eustache@femto-st.fr}\\[2mm]
{\footnotesize $^{1}$ Institut Femto-ST}\\
[4mm]
%
% Mots clés
\noindent \textbf{Mots clés : } Réfrigération magnétocalorique, dispositifs expérimentaux, étude bibliographique\\[4mm]
%
% Résumé
\noindent \textbf{Résumé : } 
\SetBgContents{Draft}
{\normalsize
L'effet magnétocalorique se traduit par une variation de température des matériaux ferromagnétiques lorsqu'ils sont soumis à une variation de champ magnétique. La magnétisation d'un matériau ferromagnétique entraine un alignement des moments magnétiques de ses spins atomiques. Cette nouvelle organisation à l'échelle des atomes provoque une diminution de l'entropie globale et en conséquence une augmentation de température. Cet effet est réversible. En effet si le matériau est retiré du champ magnétique cela va se traduire par une désaimantation qui entraine une diminution de sa température.







La première observation d'un phénomène thermomagnétique revient à W. Gilbert en 1600, qui note une réduction de l'aimantation d'un fil de fer quand celui-ci est chauffé. Cependant, la découverte de l'effet magnétocalorique est attribuée à E. Warburg en 1881. La mise en évidence de l'effet magnétocalorique et de sa réversibilité par A. Picard et P. Weiss en 1917 a permis une grande avancée dans le domaine. Par la suite, un intérêt plus important a été porté sur cette technologie. L'effet magnétocalorique a d'abord été utilisé au début du 19ème siècle dans la réfrigération. Ce n'est que depuis quelques décennies que l'on s'intéresse à cette technologie pour des applications fonctionnant à température ambiante.







Cependant l'effet magnétocalorique seul est insuffisant pour proposer une alternative aux systèmes de production de chaud et de froid actuels. En effet, le matériau le plus utilisé et performant à température ambiante est le Gd (gadolinium). Or celui-ci présente un effet magnétocalorique ne dépassant pas 3 K/T. Afin de résoudre ce problème clé, il est nécessaire de réaliser des cycles AMRR (Active Magnetic Regenerative Refrigeration). Il s'agit de parcourir des cycles d'aimantation et de désaimantation du matériau, synchronisés avec des phases de circulation fluidique permettant un transfert de chaleur entre matériau magnétocalorique et fluide caloporteur circulant entre deux sources de chaleur. L'objectif est de créer un gradient de température au sein du matériau et in fine d'augmenter l'écart de température entre les sources chaude et froide. 







Cette technologie présente de nombreux avantages comme en particulier une absence de polluant atmosphérique, de bruit et surtout une efficacité énergétique supérieure à celle de l'effet Peltier ou d'un cycle thermodynamique classique. Le premier prototype basé sur l'effet magnétocalorique à température ambiante a été développé par Brown en 1976. Depuis de nombreux prototypes ont vu le jour proposant un dispositif de remplacement aux systèmes actuels, et présentant des caractéristiques toujours plus performantes. Aujourd'hui, il est donc nécessaire, si l'on souhaite que cette technologie soit commercialisée à grande échelle, de concevoir un dispositif qui répondrait aux problématiques environnementales actuelles tout en proposant une efficacité et un encombrement suffisamment intéressant et compétitif. 







Dans cet article, les avancés récentes dans la conception de dispositifs magnétocaloriques seront détaillées. Les prototypes les plus pertinents seront présentés et classés en fonction de leurs applications, leurs performances ou encore leur efficacité. Le but final est d'extraire une conception optimale de machine en visant un développement industriel.

 \vfill doi : \url{https://doi.org/10.25855/SFT2021-007}

}
 
