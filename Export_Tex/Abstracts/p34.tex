

    %=====================================
    %   WARNING
    %   FICHIER AUTOMATISE
    %   NE PAS MODIFIER
    %=====================================

    \newpage

%%%%%%%%%%%%%%%%%%%%%%%%%%%%%%%%%%%%%%%%%%%%
%% Papier 34
%%%%%%%%%%%%%%%%%%%%%%%%%%%%%%%%%%%%%%%%%%%%

% Indexations
\index{HarnaneYamina@Harnane, Yamina}
\index{BouzidSihem@Bouzid, Sihem}
\index{BerkaneSonia@Berkane, Sonia}
\index{BrimaAbdelhafid@Brima, Abdelhafid}
%
% Titre
\begin{flushleft}
\phantomsection\addtocounter{section}{1}
\addcontentsline{toc}{section}{Analyse du confort thermique dans une cavité ventilée selon la position de l'ouverture de sortie}
{\Large \textbf{Analyse du confort thermique dans une cavité ventilée selon la position de l'ouverture de sortie}}\label{ref:34}
\end{flushleft}
%
% Auteurs
Yamina Harnane$^{1,\star}$, Sihem Bouzid$^{2}$, Sonia Berkane$^{3}$, Abdelhafid Brima$^{1}$\\[2mm]
$^{\star}$ \Letter : \url{harnane_y@yahoo.fr}\\[2mm]
{\footnotesize $^{1}$ Université d'Oum El Bouaghi, Algérie. Laboratoire de Génie Mécanique (LGM), Biskra }\\
{\footnotesize $^{2}$ Université d'Oum El Bouaghi, Algérie. Laboratoire conception et modélisation avancée des systèmes mécaniques et thermo fluides. Oum El Bouaghi}\\
{\footnotesize $^{3}$ Université Batna 2}\\
[4mm]
%
% Mots clés
\noindent \textbf{Mots clés : } simulation; convection mixte; confort thermique\\[4mm]
%
% Résumé
\noindent \textbf{Résumé : } 

{\normalsize
La ventilation a une influence majeure sur le confort thermique et sur l'efficacité des installations thermiques. Ainsi, un bon système de ventilation peut fournir un environnement dont les conditions thermiques sont confortables avec une consommation d'énergie plus basse. La ventilation naturelle et mixte avec des ouvertures Entrée/Sortie choisit judicieusement est l'une des solutions qui s'avère intéressante. Dans cet article, nous présentons une étude numérique du transfert de chaleur par convection mixte dans une cavité carrée ventilée de hauteur 2,5~m et de largeur 2,5~m. L'écoulement est bidimensionnel en régime turbulent et stationnaire. Selon la disposition de l'ouverture de sortie de l'air, quatre configurations sont considérées, l'entrée étant fixe et disposée en bas de la paroi gauche. Toutes les parois sont considérées adiabatiques sauf la paroi gauche d'épaisseur l qui exposée à une densité de flux fixe donnant un nombre de Grashof égal $\unit{7\cdot 10^{13}}$. Le modèle de turbulence RNG k-$\varepsilon$ est adopté, les équations gouvernantes sont résolues numériquement en utilisant le code Fluent 14.0. L'algorithme SIMPLEC est choisi pour le couplage pression-vitesse sur un maillage structuré 140x150, raffiné près des parois, plus 90 cellules dans la paroi solide gauche. Pour évaluer les performances du confort thermique pour chaque cas d'étude, nous avons analysé la dynamique de l'écoulement qui est trouvée bicellulaire et sa thermique indique un bon rafraîchissement. L'objectif de cette étude est d'analyser l'influence de l'emplacement de l'ouverture de sortie sur le confort thermique à l'intérieur. Afin d'optimiser la meilleure configuration offrant le confort thermique, les résultats obtenus pour l'efficacité de la ventilation sont donnés en termes de température moyenne à l'intérieur de la cavité, de distribution de température $\varepsilon_T$. Afin de prévoir numériquement les zones de confort thermique, on a calculé température effective de tirage EDT. L'indice EDT montre une grande sensibilité de la température et de la vitesse de l'air sur la zone de confort thermique. L'indice EDT supérieur à +1,1 indique une zone d'inconfort chaud et lorsque EDT est inférieur à -1,7 on parle d'une zone d'inconfort froid. Lorsque cet indice est entre -1,7 et +1,1 ces zones sont en confort thermique.

 \vfill doi : \url{https://doi.org/10.25855/SFT2021-034}

}
 
