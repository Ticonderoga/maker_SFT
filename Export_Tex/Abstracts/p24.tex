

    %=====================================
    %   WARNING
    %   FICHIER AUTOMATISE
    %   NE PAS MODIFIER
    %=====================================

    \newpage

%%%%%%%%%%%%%%%%%%%%%%%%%%%%%%%%%%%%%%%%%%%%
%% Papier 24
%%%%%%%%%%%%%%%%%%%%%%%%%%%%%%%%%%%%%%%%%%%%

% Indexations
\index{RongierClement@Rongier, Clément}
\index{GilblasRemi@Gilblas, Rémi}
\index{SchmidtFabrice@Schmidt, Fabrice}
\index{LeMaoultYannick@Le Maoult, Yannick}
%
% Titre
\begin{flushleft}
\phantomsection\addtocounter{section}{1}
\addcontentsline{toc}{section}{Vers un modèle thermo-optique des LEDs utilisées dans les systèmes d'éclairage automobile}
{\Large \textbf{Vers un modèle thermo-optique des LEDs utilisées dans les systèmes d'éclairage automobile}}\label{ref:24}
\end{flushleft}
%
% Auteurs
Clément Rongier$^{1,\star}$, Rémi Gilblas$^{1}$, Fabrice Schmidt$^{1}$, Yannick Le Maoult$^{1}$\\[2mm]
$^{\star}$ \Letter : \url{clement.rongier@mines-albi.fr}\\[2mm]
{\footnotesize $^{1}$ IMT Mines Albi Carmaux, Institut Clément Ader}\\
[4mm]
%
% Mots clés
\noindent \textbf{Mots clés : } LED haute luminance, Simulations thermo-optiques, Thermographie infrarouge, Caractérisation optique, Eclairage automobile\\[4mm]
%
% Résumé
\noindent \textbf{Résumé : } 

{\normalsize
Depuis quelques années, les systèmes d'éclairage automobile sont constitués de dispositifs optoélectroniques, dans lesquels plusieurs diodes électroluminescentes (DEL) sont impliquées. Pour répondre aux nouvelles tendances du marché, les systèmes à LED sont remplacés par une seule LED haute luminance. Celle-ci émet jusqu'à dix fois plus de puissance optique que les LED classiques. L'énergie optique émise par le composant induit des densités de puissance thermique qui doivent être contrôlées thermiquement. De plus, les performances optiques et la fiabilité du composant sont directement liées à la température. Ensuite, une fois que la LED est intégrée dans son système optique, l'énergie lumineuse émise par le composant induit une concentration d'énergie à l'intérieur du système. En conséquence, le système optique s'auto-échauffe ce qui peut provoquer le début de sa défaillance. Il est donc crucial de prévoir comment l'énergie lumineuse interagit avec le système optique. Ainsi, des modèles numériques précis et robustes doivent être développés afin de prédire l'interaction lumière-matière et son couplage avec les autres modes de transfert de chaleur. 



Dans cet article, l'étude vise à développer et valider un modèle thermo-optique de LED haute luminance. L'approche adoptée consiste à résoudre les différents modes de transfert de manière couplée, en prenant en compte l'interaction rayonnement-matière. Dans un premier temps, l'approche numérique utilisée pour simuler l'interaction rayonnement-matière dans le logiciel commercial FVM FloEFD™ a été confrontée à une solution analytique de référence. Une fois la validation de la méthode numérique finalisée, la caractérisation optique complète du composant a été réalisée. Pour cela, un spectroradiomètre équipé d'une sphère intégrante a été utilisé. Il permet de mesurer l'énergie lumineuse émise ainsi que le spectre d'émission des LEDs. Cette caractérisation permet d'identifier les paramètres physiques jouant un rôle clé en vue de construire le modèle d'émission des LEDs. Un tel modèle a été développé et intégré dans le logiciel commercial FVM FloEFD™. Cela permet de résoudre à la fois les transferts de chaleur conducto-convecto-radiatifs et l'interaction rayonnement-matière de manière couplée.



Pour valider un modèle aussi complexe, une approche expérimentale a été conçue, basée sur la thermographie infrarouge. Une plaque est placée entre la LED et le dispositif IR et reçoit l'éclairement émis par le composant optoélectronique. Les résultats de la simulation thermo-optique ont ensuite été comparés aux données expérimentales, en termes de champs de température et de profils de température. Le modèle de simulation est prédictif des phénomènes physiques obtenus expérimentalement. Le modèle d'émission des LED couplé aux trois modes de transfert de chaleur peut ainsi être utilisé avec une précision acceptable.

 \vfill doi : \url{https://doi.org/10.25855/SFT2021-024}

}
 
