

    %=====================================
    %   WARNING
    %   FICHIER AUTOMATISE
    %   NE PAS MODIFIER
    %=====================================

    \newpage

%%%%%%%%%%%%%%%%%%%%%%%%%%%%%%%%%%%%%%%%%%%%
%% Papier 48
%%%%%%%%%%%%%%%%%%%%%%%%%%%%%%%%%%%%%%%%%%%%

% Indexations
\index{WeppeAlexandre@Weppe, Alexandre}
\index{MoreauFlorian@Moreau, Florian}
\index{SauryDidier@Saury, Didier}
%
% Titre
\begin{flushleft}
\phantomsection\addtocounter{section}{1}
\addcontentsline{toc}{section}{Analyse des transferts thermiques au sein d'un écoulement de convection naturelle  dans un espace confiné comportant un obstacle partiellement chauffé.}
{\Large \textbf{Analyse des transferts thermiques au sein d'un écoulement de convection naturelle  dans un espace confiné comportant un obstacle partiellement chauffé.}}\label{ref:48}
\end{flushleft}
%
% Auteurs
Alexandre Weppe$^{1,\star}$, Florian Moreau$^{1}$, Didier Saury$^{1}$\\[2mm]
$^{\star}$ \Letter : \url{alexandre.weppe@ensma.fr}\\[2mm]
{\footnotesize $^{1}$ Institut Pprime, UPR 3346 CNRS – ENSMA – Université de Poitiers, BP 40109, F-86961 Futuroscope Chasseneuil Cedex, France}\\
[4mm]
%
% Mots clés
\noindent \textbf{Mots clés : } convection naturelle, étude expérimentale, transfert de chaleur\\[4mm]
%
% Résumé
\noindent \textbf{Résumé : } 

{\normalsize
Les écoulements  soumis à des effets de flottabilité prépondérants sont observés dans de nombreuses applications industrielles, et en particulier dans les secteurs du nucléaire et de l'automobile. On peut citer la problématique du refroidissement d'un compartiment moteur qui est un point essentiel dans le dimensionnement d'un véhicule. Par exemple, suite à un arrêt brutal du moteur après une forte sollicitation, l'intégrité du moteur doit être préservée alors même qu'il n'est plus refroidi par un écoulement externe forcé. Pour de telles situations, la convection naturelle, souvent en régime turbulent, doit permettre d'assurer le refroidissement (cas dimensionnant).



Le projet, au sein duquel s'inscrit cette étude expérimentale, a pour ambition de résoudre les problèmes rencontrés par les partenaires industriels lors de simulations d'écoulements turbulents avec effets de flottabilité dominants en espace confiné ainsi que d'améliorer la compréhension des phénomènes physiques observés pour ce type d'écoulement. En effet, à ce jour, les modèles RANS développés pour obtenir des temps de calculs adaptés à un contexte industriel ne rendent pas compte avec précision suffisante des interactions entre la turbulence et les effets de flottabilité et par conséquent les transferts sont mal quantifiés.



Une configuration de référence est définie, permettant d'étudier un écoulement en espace confiné représentatif des régimes rencontrés dans le domaine automobile. Le compartiment moteur a été simplifié en une cavité cubique comportant en son sein un autre bloc cubique partiellement chauffé sur l'une de ses faces. 



Pour cette étude, un écoulement caractérisé par un nombre de Rayleigh basé sur la hauteur du bloc chauffant $Ra_H=1,33(\pm 0,06)\cdot 10^9$ est étudié. Pour obtenir cet écoulement, l'obstacle cubique de dimension $H\times H \times H=L_{obstacle}^3=(0,8 m)^3$ est placé au centre d'une cavité cubique de dimensions supérieures $L \times L \times L=(1 m)^3$. Cet obstacle est exclusivement chauffé sur l'une de ses faces latérales verticales à une température $\unit{T_c}$ tandis que deux parois verticales de la grande cavité le contenant sont maintenues à une température $\unit{T_f}$ de sorte que l'on obtienne une différence de température $\Delta T=T_c-T_f (=32,5\dC)$. Il s'agit de la paroi de la grande cavité en vis-à-vis de la face à $\unit{T_c}$ et de la paroi opposée à cette dernière. Les autres parois verticales ainsi que les deux parois horizontales haute et basse de la grande cavité sont adiabatiques.



Dans ce travail, des profils de température obtenus par micro-thermocouple dans le plan médian de la cavité ainsi que des flux thermiques en paroi sont présentés et analysés.

 \vfill doi : \url{https://doi.org/10.25855/SFT2021-048}

}
 
