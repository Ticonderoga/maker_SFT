

    %=====================================
    %   WARNING
    %   FICHIER AUTOMATISE
    %   NE PAS MODIFIER
    %=====================================

    \newpage

%%%%%%%%%%%%%%%%%%%%%%%%%%%%%%%%%%%%%%%%%%%%
%% Papier 47
%%%%%%%%%%%%%%%%%%%%%%%%%%%%%%%%%%%%%%%%%%%%

% Indexations
\index{GaumeBenjamin@Gaume, Benjamin}
\index{JolyFrederic@Joly, Frédéric}
\index{BoissiereBenjamin@Boissiere, Benjamin}
\index{GhazalGhassan@Ghazal, Ghassan}
\index{QuemenerOlivier@Quemener, Olivier}
%
% Titre
\begin{flushleft}
\phantomsection\addtocounter{section}{1}
\addcontentsline{toc}{section}{Simulation d'un four de recuit  par modèle réduit modal}
{\Large \textbf{Simulation d'un four de recuit  par modèle réduit modal}}\label{ref:47}
\end{flushleft}
%
% Auteurs
Benjamin Gaume$^{1,\star}$, Frédéric Joly$^{1}$, Benjamin Boissiere$^{2}$, Ghassan Ghazal$^{2}$, Olivier Quemener$^{1}$\\[2mm]
$^{\star}$ \Letter : \url{b.gaume@iut.univ-evry.fr}\\[2mm]
{\footnotesize $^{1}$ LMEE, Univ Evry, Université Paris-Saclay, 91020, Evry, France.}\\
{\footnotesize $^{2}$ ArcelorMittal, Global Research \& Development, Maizières Process Voie Romaine, BP 30320 - F-57283 Maizières-lès-Metz Cedex}\\
[4mm]
%
% Mots clés
\noindent \textbf{Mots clés : } Convection-diffusion; Rayonnement thermique; Réduction de modèle\\[4mm]
%
% Résumé
\noindent \textbf{Résumé : } 

{\normalsize
La méthode AROMM (Amalgam Reduced Order Modal Model) est une technique de réduction modale qui permet d'obtenir rapidement l'évolution de la température en fonction du temps en tout point d'une géométrie qui peut être complexe. Cette méthode utilise la décomposition du champ de température recherché sur une base modale de taille réduite calculée préalablement et adaptée à tout type de problèmes non linéaires. L'étude présentée propose d'appliquer cette technique pour une scène thermique caractérisée par un couplage entre un problème de diffusion-convection des parties solides en mouvement (équation de la chaleur) et le rayonnement entre les différentes surfaces de la scène thermique (par la méthode de radiosité). 



L'application traitée est un problème de four de recuit de bandes métalliques, pour lequel la géométrie simplifiée d'une petite partie du four fait l'objet d'une simulation numérique. Dans cette portion de four, une bande en acier est mise en mouvement par un rouleau et est chauffée par rayonnement à l'aide de quatre tubes radiants maintenus à hautes température (900$^{\circ}$C). Un écran de protection permet de protéger le rouleau du flux radiatif. Enfin, le contact imparfait entre la bande et le rouleau est modélisé par un couplage par zone pour une prise en compte de la résistance thermique de contact. La base modale est construite à partir d'un problème de référence simplifié, l'ensemble des surfaces sont considérées comme des corps noirs et la vitesse de la bande est considérée comme fixe. 



Les résultats obtenus montrent la pertinence de cette technique en présence de phénomènes couplés de conduction dans les parois, de mouvement de la bande et du rouleau, et de rayonnement entre les différentes surfaces. Les premiers résultats amènent un gain en temps de calcul de l'ordre de 50 par rapport à une technique classique utilisant les éléments finis, pour une erreur de température maximum sur la bande inférieure à 2\%.

 \vfill doi : \url{https://doi.org/10.25855/SFT2021-047}

}
 
