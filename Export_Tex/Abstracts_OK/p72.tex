

    %=====================================
    %   WARNING
    %   FICHIER AUTOMATISE
    %   NE PAS MODIFIER
    %=====================================

    \newpage

%%%%%%%%%%%%%%%%%%%%%%%%%%%%%%%%%%%%%%%%%%%%
%% Papier 72
%%%%%%%%%%%%%%%%%%%%%%%%%%%%%%%%%%%%%%%%%%%%

% Indexations
\index{LethuillierJeremie@Lethuillier, Jérémie}
\index{MiscevicMarc@Miscevic, Marc}
\index{LavieillePascal@Lavieille, Pascal}
\index{TopinFrederic@Topin, Frédéric}
%
% Titre
\begin{flushleft}
\phantomsection\addtocounter{section}{1}
\addcontentsline{toc}{section}{Modélisation individu centré de la condensation en goutte}
{\Large \textbf{Modélisation individu centré de la condensation en goutte}}\label{ref:72}
\end{flushleft}
%
% Auteurs
Jérémie Lethuillier$^{1,\star}$, Marc Miscevic$^{1}$, Pascal Lavieille$^{1}$, Frédéric Topin$^{2}$\\[2mm]
$^{\star}$ \Letter : \url{lethuillier@laplace.univ-tlse.fr}\\[2mm]
{\footnotesize $^{1}$ Laplace}\\
{\footnotesize $^{2}$ IUSTI}\\
[4mm]
%
% Mots clés
\noindent \textbf{Mots clés : } Condensation en goutte\\[4mm]
%
% Résumé
\noindent \textbf{Résumé : } 

{\normalsize
La condensation en gouttes est un régime extrêmement efficace dans l'optique de gérer d'important flux de chaleur. De récentes études ont obtenu des coefficients de transfert allant jusqu'à 250~$\unit{kW.m^{-2}}$. Afin de modéliser le coefficient de transfert lors de la condensation en gouttes, deux sous modèles sont nécessaires. Un premier modèle évaluant les transferts thermiques à l'intérieur de la goutte, ce qui permet par la suite de déterminer sa vitesse de croissance. Le modèle classiquement utilisé dans la littérature est basé sur un réseau de plusieurs résistances thermiques entre la phase vapeur et le substrat sous-refroidi. 



Le second modèle concerne la détermination de la distribution de la taille des gouttes sur la surface. Actuellement cette distribution est calculée à partir d'un modèle hybride basé sur une loi semi-empirique pour les gouttes observables expérimentalement (c'est à dire les gouttes supérieures à quelques microns) et sur un modèle statistique pour les plus petites gouttes. Comme aucunes données n'existent à ce jour pour valider la distribution des petites gouttes, une approche numérique de type individu centré a été utilisée afin de confronter ce modèle statistique. 



L'approche de type individu centré permet de retrouver la distribution des grosses gouttes qui a été validée expérimentalement. En revanche, la distribution des gouttes de petites tailles est relativement différente. Dans les configurations considérées dans la présente étude, une analyse des principales hypothèses retenues dans l'approche statistique (en particulier, l'hypothèse d'un temps caractéristique de renouvellement constant $\tau$ quelle que soit la taille des gouttes) a révélé que le mécanisme principal de renouvellement de la surface est en réalité uniquement lié au coalescence des gouttes immobiles et non au balayage des gouttes en mouvement. A partir de ces résultats, une modification du modèle statistique est proposée et discutée, ainsi qu'une analyse de l'influence de ces hypothèses sur le coefficient de transfert.

 \vfill doi : \url{https://doi.org/10.25855/SFT2021-072}

}
 
