

    %=====================================
    %   WARNING
    %   FICHIER AUTOMATISE
    %   NE PAS MODIFIER
    %=====================================

    \newpage

%%%%%%%%%%%%%%%%%%%%%%%%%%%%%%%%%%%%%%%%%%%%
%% Papier 23
%%%%%%%%%%%%%%%%%%%%%%%%%%%%%%%%%%%%%%%%%%%%

% Indexations
\index{JadalMariam@Jadal, Mariam}
\index{SotoJerome@Soto, Jérôme}
\index{DelaunayDidier@Delaunay, Didier}
%
% Titre
\begin{flushleft}
\phantomsection\addtocounter{section}{1}
\addcontentsline{toc}{section}{Validation expérimentale d'un modèle cinétique de solidification d'une plaque MCP/GNE}
{\Large \textbf{Validation expérimentale d'un modèle cinétique de solidification d'une plaque MCP/GNE}}\label{ref:23}
\end{flushleft}
%
% Auteurs
Mariam Jadal$^{1}$, Jérôme Soto$^{2}$, Didier Delaunay$^{1,\star}$\\[2mm]
$^{\star}$ \Letter : \url{didier.delaunay@univ-nantes.fr}\\[2mm]
{\footnotesize $^{1}$ Université de Nantes, CNRS, LTeN, UMR 6607, Polytech'Nantes, BP 50609, 44306 NANTES Cedex 3}\\
{\footnotesize $^{2}$ ICAM Nantes, 35 avenue du Champ de Manoeuvres, 44470 Carquefou}\\
[4mm]
%
% Mots clés
\noindent \textbf{Mots clés : } changement de phase, stockage d'énergie, modélisation\\[4mm]
%
% Résumé
\noindent \textbf{Résumé : } 

{\normalsize
Nous présentons le comportement thermique lors de son changement de phase d'un matériau composé d'une fraction massique de graphite naturel expansé (GNE) de 20% pour 80% de paraffine RT70 HC de RUBITHERM®. La géométrie étudiée est une plaque de ce matériau composite pour laquelle un dispositif expérimental a été conçu. Il permet de reproduire les conditions de transferts lors d'un processus de stockage thermique, dans une plaque qui représente un élément unitaire d'un module de stockage. Le système permet d'imposer une variation de température au matériau composite, dans le plan de la plaque de dimensions 143×143×20mm, afin de privilégier les transferts suivant la direction dans laquelle la conductivité thermique est la plus importante. Le dispositif expérimental comporte une pièce de cuivre dans laquelle circule le fluide caloporteur (eau) placée sur la tranche de la plaque. Des plaques isolantes entourent l'échantillon afin de limiter les pertes thermiques. Ce dispositif expérimental permet, dans un premier temps, d'identifier des propriétés thermiques du matériau et les résistances entre la plaque et son environnement hors du domaine de changement de phase à l'aide d'une méthode inverse, et ensuite, de valider expérimentalement le modèle numérique de changement de phase. Le champ de température dans la plaque est mesuré à l'aide de micro-thermocouples. Il est crucial de maitriser le contact entre l'échantillon et la pièce de cuivre dans laquelle circule le fluide caloporteur.  Pour cela, un vérin a été placé sur la face latérale de l'échantillon (côté opposé à la pièce de cuivre) pour appliquer une pression régulée entre l'échantillon et la pièce de cuivre. 



Un modèle numérique 3D sous Comsol multiphysics a été développé. Le MCP étant encapsulé dans la matrice de GNE anisotrope, les phénomènes de transferts thermiques dans la plaque sont conductifs. La fusion est classiquement modélisée par méthode enthalpique, la fusion ne présentant pas de cinétique, un pic de fusion assez étalé étant observé en DSC, indépendant de la vitesse de chauffage. Le matériau présente en solidification un comportement complexe, avec deux pics, se déplaçant en fonction de la vitesse de refroidissement. Nous avons utilisé une méthode de cinétique de cristallisation qui consiste à ajouter un terme source à l'équation de l'énergie. Au cours de la phase de solidification deux transformations exothermiques ont été mises en évidence. L'évolution de la fonction de cinétique de cristallisation de chaque transformation a été définie. La classique formulation différentielle de Nakamura a été utilisée pour décrire les cinétiques de cristallisation pour un refroidissement quelconque. Une pondération entre chaque transformation a été introduite, basée sur le rapport d'enthalpie du pic de chaque transformation par rapport à l'enthalpie totale de changement de phase. 



En utilisant les conditions aux limites obtenues par des thermocouples placés sur les faces latérales de l'échantillon, les températures à cœur ont été calculées à la position de plusieurs thermocouples La méthode montre des résultats excellents, avec un écart quadratique de l'ordre de 0.03K entre mesures et calcul, inférieur aux incertitudes de mesures.

 \vfill doi : \url{https://doi.org/10.25855/SFT2021-023}

}
 
