

    %=====================================
    %   WARNING
    %   FICHIER AUTOMATISE
    %   NE PAS MODIFIER
    %=====================================

    \newpage

\backgroundsetup{contents={Work In Progress},scale=7}
\BgThispage
%%%%%%%%%%%%%%%%%%%%%%%%%%%%%%%%%%%%%%%%%%%%
%% Papier 54
%%%%%%%%%%%%%%%%%%%%%%%%%%%%%%%%%%%%%%%%%%%%

% Indexations
\index{LouWanruo@Lou, Wanruo}
\index{LuoLingai@Luo, Lingai}
\index{FanYilin@Fan, Yilin}
\index{BaudinNicolas@Baudin, Nicolas}
\index{SotoJerome@Soto, Jérôme}
\index{XieBaoshan@Xie, Baoshan}
%
% Titre
\begin{flushleft}
\phantomsection\addtocounter{section}{1}
\addcontentsline{toc}{section}{Comparison between simulations and experimental measurements for the optimisation of a thermocline tank distributor}
{\Large \textbf{Comparison between simulations and experimental measurements for the optimisation of a thermocline tank distributor}}\label{ref:54}
\end{flushleft}
%
% Auteurs
Wanruo Lou$^{1}$, Lingai Luo$^{1}$, Yilin Fan$^{1}$, Nicolas Baudin$^{1,\star}$, Jérôme Soto$^{2}$, Baoshan Xie$^{1}$\\[2mm]
$^{\star}$ \Letter : \url{nicolas.baudin@univ-nantes.fr}\\[2mm]
{\footnotesize $^{1}$ Laboratoire de Thermique et Energie de Nantes (LTéN), UMR CNRS 6607, Polytech' Nantes - Université de Nantes}\\
{\footnotesize $^{2}$ Laboratoire de Thermique et Energie de Nantes (LTéN), et Laboratoire Énergétique Mécanique et Matériaux (ICAM)}\\
[4mm]
%
% Mots clés
\noindent \textbf{Mots clés : } Thermal energy storage, thermocline, optimum manifold configurations, experimental measurements\\[4mm]
%
% Résumé
\noindent \textbf{Résumé : } 

{\normalsize
Thermocline energy storage single tanks are a low-cost alternative to the conventional two-tank systems for the concentrated solar power plant. In the single tank, both cold and hot heat transfer fluids are stored together and a thermal stratification is formed between two fluids by buoyancy force due to the different densities. This stratification region, called the thermocline, as an indicator of the thermal performance of the tank system, can be influenced by factors, such as operating conditions, tank geometry, and especially inlet and outlet distributors. Indeed, improper inlet/outlet manifolds may cause the mixing of hot and cold fluids and disturb the temperature stratification, resulting in reduced thermal performances of the storage tank. Distributor as one of the effective solutions can be applied to the storage tank to achieve a more even flow distribution. Our goal is to optimize the distributor holes diameter and position to get an even flow distribution and thus a thinner and more stable thermocline.



This work had been conducted in two parts. At first, an algorithm coupling Fluent simulations and an optimization method was designed and used to predict optimum manifold configurations. These results were presented by Lou et al. in 2019. Secondly, an experimental setup had been built to reproduce a lab-scale storage tank: Particle Image Velocimetry measurements and temperature measurements were compared with the simulation results for a base and optimized manifolds. In the experimental part, the measurements of the optimized system were conducted at different operating temperatures and with different inlet flow rates. Moreover, the system was also compared to the no optimized system with no distributors. In the end, a good agreement was found between experiments and simulations. All these efforts help to provide generalized design guidelines for a possible upscaling from the laboratory to the industrial scale for real-world CSP application.

 \vfill Work In Progress

}
 
