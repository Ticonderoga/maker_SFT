

    %=====================================
    %   WARNING
    %   FICHIER AUTOMATISE
    %   NE PAS MODIFIER
    %=====================================

    \newpage

%%%%%%%%%%%%%%%%%%%%%%%%%%%%%%%%%%%%%%%%%%%%
%% Papier 68
%%%%%%%%%%%%%%%%%%%%%%%%%%%%%%%%%%%%%%%%%%%%

% Indexations
\index{BegotSylvie@Bégot, Sylvie}
\index{GetieMuluken@Getie, Muluken}
\index{LanzettaFrancois@Lanzetta, François}
\index{BarthesMagali@Barthès, Magali}
\index{DeLabachelerieMichel@De Labachelerie, Michel}
%
% Titre
\begin{flushleft}
\phantomsection\addtocounter{section}{1}
\addcontentsline{toc}{section}{Etude théorique d'un micro-moteur Stirling}
{\Large \textbf{Etude théorique d'un micro-moteur Stirling}}\label{ref:68}
\end{flushleft}
%
% Auteurs
Sylvie Bégot$^{1,\star}$, Muluken Getie$^{2}$, François Lanzetta$^{1}$, Magali Barthès$^{1}$, Michel De Labachelerie$^{1}$\\[2mm]
$^{\star}$ \Letter : \url{sylvie.begot@univ-fcomte.fr}\\[2mm]
{\footnotesize $^{1}$ FEMTO-ST}\\
{\footnotesize $^{2}$ FEMTO-ST - Bahir Dar Universitéy}\\
[4mm]
%
% Mots clés
\noindent \textbf{Mots clés : } Stirling ; Microtechnologie ; Modélisation\\[4mm]
%
% Résumé
\noindent \textbf{Résumé : } 

{\normalsize
La recherche d'économies d'énergie nous conduit à étudier des solutions de production d'énergie locales et de faible puissance destinées notamment à alimenter les objets connectés. Parmi les solutions technologiques possibles, l'utilisation d'un moteur Stirling fabriqué en micro-technologie et fonctionnant en récupération de chaleur à basse température est étudiée dans cette communication. Pour cela, une adaptation des techniques et concepts utilisés pour les machines macroscopiques est nécessaire. L'article présente tout d'abord les technologies, géométries et matériaux utilisables en micro-fabrication. A partir de cette étude, un concept de base de machine de type Alpha est proposé. Des membranes sont utilisées en lieu et place des pistons qui ne sont pas envisagés à cette échelle. Les chambres de compression et de détente et les canaux des échangeurs sont gravés dans des galettes ou wafers.  Les matériaux utilisés sont le silicone pour les membranes, le silicium pour les pièces où un bon échange thermique est nécessaire, c'est-à-dire les échangeurs, et le verre pour les autres pièces.  Le régénérateur utilise des micro-piliers en silicium. Un dimensionnement permettant l'obtention théorique d'une puissance de 2 mW est ensuite proposé grâce à une modélisation adiabatique du moteur incluant les pertes principales : conduction directe entre les pièces chaudes et froides, inefficacité du régénérateur, pertes de charge, pertes par hystérésis. Ensuite, une étude paramétrique est menée dans le but d'observer des différences entre le comportement théorique de cette micro-machine et celui des machines macroscopiques. Ainsi, l'influence de la fréquence de fonctionnement qui pourrait être théoriquement élevée en raison de la faible inertie des pièces de la machine est étudiée. L'utilisation des 3 gaz de travail (air, hélium, hydrogène) principalement employés dans les machines fait également l'objet de simulations. Les paramètres du régénérateur comme sa longueur et sa porosité sont étudiés. A partir de résultats cette étude en modélisation, des recommandations de conception de micro-machines Stirling sont déduites.

 \vfill doi : \url{https://doi.org/10.25855/SFT2021-068}

}
 
