

    %=====================================
    %   WARNING
    %   FICHIER AUTOMATISE
    %   NE PAS MODIFIER
    %=====================================

    \newpage

%%%%%%%%%%%%%%%%%%%%%%%%%%%%%%%%%%%%%%%%%%%%
%% Papier 44
%%%%%%%%%%%%%%%%%%%%%%%%%%%%%%%%%%%%%%%%%%%%

% Indexations
\index{SghuriAnas@Sghuri, Anas}
\index{BillaudYann@Billaud, Yann}
\index{SignorLoic@Signor, Loïc}
\index{GadaudPascal@Gadaud, Pascal}
\index{SauryDidier@Saury, Didier}
\index{MilhetXavier@Milhet, Xavier}
%
% Titre
\begin{flushleft}
\phantomsection\addtocounter{section}{1}
\addcontentsline{toc}{section}{Optimisation de la caractérisation thermique de l'argent fritté par la méthode flash}
{\Large \textbf{Optimisation de la caractérisation thermique de l'argent fritté par la méthode flash}}\label{ref:44}
\end{flushleft}
%
% Auteurs
Anas Sghuri$^{1}$, Yann Billaud$^{1,\star}$, Loïc Signor$^{1}$, Pascal Gadaud$^{1}$, Didier Saury$^{1}$, Xavier Milhet$^{1}$\\[2mm]
$^{\star}$ \Letter : \url{yann.billaud@ensma.fr}\\[2mm]
{\footnotesize $^{1}$ Institut Pprime UPR CNRS 3346 - CNRS / ENSMA / Univ. Poitiers}\\
[4mm]
%
% Mots clés
\noindent \textbf{Mots clés : } Argent fritté, conduction thermique, estimation de paramètres, méthode flash, méthodes inverses.\\[4mm]
%
% Résumé
\noindent \textbf{Résumé : } 

{\normalsize
Ce travail présente une optimisation de la caractérisation thermique par la méthode flash de l'argent fritté, utilisé notamment pour le report de composants électroniques. L'étude s'appuie sur une évaluation comparative de la durée et du niveau d'intensité d'excitation du laser, par rapport à la face de mesure, et de leurs effets sur la précision de l'estimation des diffusivités thermiques du matériau. La méthode d'estimation des diffusivités repose sur la résolution d'un problème inverse en minimisant l'écart quadratique entre la réponse d'un modèle semi-analytique 3D et les mesures issues d'une unique expérience de type méthode flash. Pour cette étude préliminaire, des données synthétiques représentatives seront utilisées. L'estimateur retenu, connu sous le nom de ENH (Estimation par Normalisation des Harmoniques), consiste à diviser les mesures par une harmonique de référence de manière à transformer le problème inverse non-linéaire en un problème inverse linéaire par rapport aux paramètres à estimer. Cette méthode permet l'estimation simultanée des composantes de la diffusivité dans le plan de l'excitation ($\alpha_x$ et $\alpha_y$) ainsi que les paramètres associés à l'excitation non uniforme. La précision de la méthode est évaluée à partir de données synthétiques, et les résultats de l'identification correspondants aux configurations étudiées, en termes de face de mesure et d'excitation, sont présentés et commentés.



 \vfill doi : \url{https://doi.org/10.25855/SFT2021-044}

}
 
