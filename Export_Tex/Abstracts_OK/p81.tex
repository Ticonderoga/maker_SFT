

    %=====================================
    %   WARNING
    %   FICHIER AUTOMATISE
    %   NE PAS MODIFIER
    %=====================================

    \newpage

\backgroundsetup{contents={Work In Progress},scale=7}
\BgThispage
%%%%%%%%%%%%%%%%%%%%%%%%%%%%%%%%%%%%%%%%%%%%
%% Papier 81
%%%%%%%%%%%%%%%%%%%%%%%%%%%%%%%%%%%%%%%%%%%%

% Indexations
\index{BarryElhadjBoubacar@Barry, Elhadj Boubacar}
\index{KangChangwoo@Kang, Changwoo}
\index{YoshikawaHarunori@Yoshikawa, Harunori}
\index{MutabaziInnocent@Mutabazi, Innocent}
%
% Titre
\begin{flushleft}
\phantomsection\addtocounter{section}{1}
\addcontentsline{toc}{section}{Convection thermoélectrique dans des cavités rectangulaires}
{\Large \textbf{Convection thermoélectrique dans des cavités rectangulaires}}\label{ref:81}
\end{flushleft}
%
% Auteurs
Elhadj Boubacar Barry$^{1}$, Changwoo Kang$^{2}$, Harunori Yoshikawa$^{3}$, Innocent Mutabazi$^{1,\star}$\\[2mm]
$^{\star}$ \Letter : \url{innocent.mutabazi@univ-lehavre.fr}\\[2mm]
{\footnotesize $^{1}$ Université Le Havre Normandie, LOMC UMR CNRS 6294}\\
{\footnotesize $^{2}$ Jeonbuk National University}\\
{\footnotesize $^{3}$ Institut de Physique de Nice - UMR 7010}\\
[4mm]
%
% Mots clés
\noindent \textbf{Mots clés : } Cavité rectangulaire ; Convection thermoélectrique ; Fluide diélectrique ; Gravité électrique.\\[4mm]
%
% Résumé
\noindent \textbf{Résumé : } 

{\normalsize
La convection est un mode de transfert de chaleur qui fait intervenir le mouvement macroscopique d'un fluide. On retrouve ce phénomène de transfert thermique dans certains appareils de la vie quotidienne tels que les radiateurs ou les circuits de refroidissement des ordinateurs ou systèmes industriels. La convection thermique est l'une des méthodes de transfert de chaleur les plus efficaces, plus économiques et très simples à mettre en œuvre. Cependant, son utilisation devient impossible lorsque l'on réduit considérablement la taille des appareils ou en l'absence de gravité terrestre. En effet, la miniaturisation des appareils permet la mise au point de composants électroniques ou électromécaniques de tailles millimétriques et submillimétriques. À ces échelles, les techniques ordinaires de refroidissement sont obsolètes en raison de la réduction de la quasi-totalité des surfaces de contacts et de l'impossibilité d'installer des ventilateurs. Cependant, ces dispositifs sont capables de supporter des champs électriques élevés sans nuire à leur fonctionnement. L'application d'un champ électrique alternatif de haute fréquence à un fluide diélectrique inhomogène est donc une méthode alternative permettant de générer la convection thermoélectrique ; et elle représente un bon candidat pour la maitrise des échanges thermiques dans les systèmes miniaturisés dans des conditions terrestres et en micropesanteur. La présente étude est motivée par l'utilisation de la poussée électrique pour contrôler la convection thermique dans des cavités rectangulaires horizontales ou verticales qui peuvent être implémentées dans des systèmes de refroidissement des dispositifs électroniques en aéronautique, aérospatiale ou en microfluidique.



Lorsqu'on applique un champ électrique alternatif de haute fréquence à un liquide diélectrique différentiellement chauffé, la différence de polarisation dans le fluide et le couplage thermoélectrique créent une force de poussée électrique avec une gravité électrique effective. Cette gravité électrique, qui représente le gradient de l'énergie potentielle par unité de masse, emmagasinée dans le condensateur que représente la cavité, est proportionnelle au carré de la tension effective. La poussée électrique a des effets similaires à la poussée d'Archimède dans la génération de la convection thermique. Nous réalisons actuellement des études numériques de la convection thermoélectrique dans une huile de silicone. Par une analyse de stabilité linéaire, le système d'équations linéarisées est résolu avec un code interne utilisant un schéma de discrétisation et la méthode de collocation spectrale de Tchebychev pour déterminer les seuils critiques de stabilité marquant le début de la convection dans le fluide. Certains résultats obtenus viennent d'être publiés dans la revue « Microgravity Science and Technology, 33, 16(2021).



Le travail en cours consiste à caractériser les régimes de convection au-delà du seuil critique. Les équations sont résolues par un code de simulations numériques directes développé au Laboratoire pour déterminer les champs de vitesses, de vorticité et de température dans l'écoulement du fluide diélectrique et le coefficient de transfert thermique adimensionné (nombre de Nusselt) en fonction de l'intensité du champ électrique appliqué. Parallèlement, nous rédigeons actuellement un article sur les résultats obtenus et le manuscrit de thèse qui devra être soutenue avant décembre 2021.

 \vfill Work In Progress

}
 
