

    %=====================================
    %   WARNING
    %   FICHIER AUTOMATISE
    %   NE PAS MODIFIER
    %=====================================

    \newpage

%%%%%%%%%%%%%%%%%%%%%%%%%%%%%%%%%%%%%%%%%%%%
%% Papier 33
%%%%%%%%%%%%%%%%%%%%%%%%%%%%%%%%%%%%%%%%%%%%

% Indexations
\index{LapertotArnaud@Lapertot, Arnaud}
\index{KadochBenjamin@Kadoch, Benjamin}
\index{LeMetayerOlivier@Le Metayer, Olivier}
%
% Titre
\begin{flushleft}
\phantomsection\addtocounter{section}{1}
\addcontentsline{toc}{section}{Optimisation multicritère d'un échangeur de chaleur air-sol pour différents climats mondiaux}
{\Large \textbf{Optimisation multicritère d'un échangeur de chaleur air-sol pour différents climats mondiaux}}\label{ref:33}
\end{flushleft}
%
% Auteurs
Arnaud Lapertot$^{1,\star}$, Benjamin Kadoch$^{1}$, Olivier Le Metayer$^{1}$\\[2mm]
$^{\star}$ \Letter : \url{arnaud.lapertot@univ-amu.fr}\\[2mm]
{\footnotesize $^{1}$ Aix-Marseille Université, CNRS, IUSTI UMR 7343, 13453 Marseille, France}\\
[4mm]
%
% Mots clés
\noindent \textbf{Mots clés : } Echangeur de chaleur air-sol ; Optimisation ; Pompe à chaleur ; Analyse de Sensibilité ; Stockage de chaleur\\[4mm]
%
% Résumé
\noindent \textbf{Résumé : } 

{\normalsize
La consommation d'énergie dans le monde augmente très rapidement et représente environ 35 \% dans le secteur résidentiel. Les stratégies mondiales ont prévu de réduire la consommation d'énergie dans le bâtiment afin de lutter contre le réchauffement climatique. Pour cela, il est possible d'utiliser des échangeurs de chaleur air-sol (EAHE) basés sur des énergies renouvelables. L'EAHE est un système de chauffage, de ventilation et de climatisation qui utilise l'air extérieur pour circuler dans des tubes souterrains afin de récupérer l'énergie du sol. Cette énergie permet de préchauffer ou refroidir l'air dans le bâtiment par ventilation. De plus, l'ajout d'une ventilation à récupération de chaleur (HRV) améliore la performance du système en récupérant l'énergie de l'air vicié. Cependant, le système nécessite de l'énergie auxiliaire pour satisfaire les besoins énergétiques globaux. Ce système énergétique EAHE/HRV peut également être couplé à une pompe à chaleur (HP) et à des collecteurs photovoltaïques (PV) afin de combler les besoins de chauffage en hiver et de refroidissement en été. En outre, des capteurs solaires thermiques peuvent être couplés à système de stockage d'eau chaude pour satisfaire la demande d'eau chaude sanitaire. Dans cette étude, le système énergétique est modélisé dynamiquement sur une année complète en utilisant des modèles d'un échangeur de chaleur air-sol, d'une ventilation à récupération de chaleur, d'une pompe à chaleur, des collecteurs photovoltaïques et thermiques, et d'un stockage d'eau chaude sanitaire. Pour étudier les performances énergétiques du système, deux critères énergétiques et un critère économique sont considérés. L'objectif principal est de déterminer le dimensionnement et la régulation optimale du système en utilisant une procédure d'optimisation. Celle-ci est composée d'une analyse de sensibilité, d'une optimisation multicritère et d'une aide à la décision. Tout d'abord, l'analyse de sensibilité est réalisée avec la méthode FAST et permet de sélectionner les paramètres les plus influents. Puis, une étude d'optimisation multicritère est effectuée avec les algorithmes génétiques pour déterminer les meilleurs compromis. Ensuite, la méthode d'aide à la décision TOPSIS sélectionne le dimensionnement et la régulation optimale. Enfin, cette procédure d'optimisation est appliquée pour différents climats liés à plusieurs capitales mondiales. Les résultats montrent que la combinaison des différents composants permet d'obtenir un système énergétique rentable et performant qui satisfait les demandes de chaleur, de rafraichissement et d'eau chaude sanitaire.

 \vfill doi : \url{https://doi.org/10.25855/SFT2021-033}

}
 
