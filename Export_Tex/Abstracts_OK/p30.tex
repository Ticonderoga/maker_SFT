

    %=====================================
    %   WARNING
    %   FICHIER AUTOMATISE
    %   NE PAS MODIFIER
    %=====================================

    \newpage

%%%%%%%%%%%%%%%%%%%%%%%%%%%%%%%%%%%%%%%%%%%%
%% Papier 30
%%%%%%%%%%%%%%%%%%%%%%%%%%%%%%%%%%%%%%%%%%%%

% Indexations
\index{KziazykTheo@Kziazyk, Théo}
\index{PhilippeBaucour@Philippe, Baucour}
\index{GavignetEric@Gavignet, Eric}
\index{ChamagneDidier@Chamagne, Didier}
%
% Titre
\begin{flushleft}
\phantomsection\addtocounter{section}{1}
\addcontentsline{toc}{section}{Caractérisation expérimentale d'un contact électrique glissant représentatif de la liaison pantographe-caténaire}
{\Large \textbf{Caractérisation expérimentale d'un contact électrique glissant représentatif de la liaison pantographe-caténaire}}\label{ref:30}
\end{flushleft}
%
% Auteurs
Théo Kziazyk$^{1,\star}$, Baucour Philippe$^{1}$, Eric Gavignet$^{1}$, Didier Chamagne$^{1}$\\[2mm]
$^{\star}$ \Letter : \url{theo.kziazyk@femto-st.fr}\\[2mm]
{\footnotesize $^{1}$ Institut FEMTO-ST (CNRS-UMR6174), Département Energie}\\
[4mm]
%
% Mots clés
\noindent \textbf{Mots clés : } Banc d'essais, Mesures thermiques, Instrumentation, Résistances de contact, Coefficient de partage de flux, Electrothermie\\[4mm]
%
% Résumé
\noindent \textbf{Résumé : } 

{\normalsize
Les problématiques liées  à la liaison pantographe-caténaire, c'est à dire le contact entre le train et les fils conducteurs l'alimentant, sont aujourd'hui encore nombreuses.



    Ainsi on constate une usure prématurée du fil et de la bande de captage liée à l'échauffement au point de contact. La caractérisation du vieillissement des matériaux d'une liaison pantographe-caténaire, l'anticipation des travaux de maintenance ou la prévention de casse sont difficiles à prendre en compte. Afin d'éviter tout risque de casse une maintenance accrue doit être mise en place. 



    Un outil de simulation permettant de prédire la distribution de température dans une bande de captage a été récemment élaboré.



    Pour alimenter celui-ci, de nombreux paramètres d'entrée sont encore nécessaires. Étant électrifié, imparfait et glissant, ceux relatifs au contact sont difficiles à évaluer et à considérer. Il s'agit essentiellement des résistances de contact électrique et thermique , du coefficient de partage de flux thermique, du glissement magnétique, du coefficient de frottement.



    



    Pour répondre à ces besoins, un banc d'essais reproduisant un contact entre un pantographe et une caténaire est réalisé. Pour cela, un courant électrique important (50-150~A) passe dans un échantillon de bande de captage en contact avec un disque de cuivre en rotation. Le banc d'essais est ainsi conçu pour être représentatif d'une liaison pantographe-caténaire réelle mais aussi modulable et flexible en termes de densité de courant au niveau de la liaison, de type de matériaux au contact ou d'effort appliqué entre les deux structures. 



    Dans cet article, la configuration d'une liaison pantographe caténaire est expliquée. Les facteurs influençant l'échauffement et l'usure de la bande de captage sont énuméré. L'ensemble du banc d'essai est ensuite décrit et sa représentativité critiquée. Les outils et techniques de pilotages, d'instrumentations et de mesures (résistivité thermique et électrique, coefficient de frottement) sont explicités. Enfin, Les perspectives à court et moyen termes sont abordées suivies d'une conclusion et de perspectives à court et moyen termes.

 \vfill doi : \url{https://doi.org/10.25855/SFT2021-030}

}
 
