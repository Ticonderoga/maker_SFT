

    %=====================================
    %   WARNING
    %   FICHIER AUTOMATISE
    %   NE PAS MODIFIER
    %=====================================

    \newpage

\backgroundsetup{contents={Work In Progress},scale=7}
\BgThispage
%%%%%%%%%%%%%%%%%%%%%%%%%%%%%%%%%%%%%%%%%%%%
%% Papier 46
%%%%%%%%%%%%%%%%%%%%%%%%%%%%%%%%%%%%%%%%%%%%

% Indexations
\index{OuzzineBadr@Ouzzine, Badr}
\index{BadinierThibault@Badinier, Thibault}
\index{DeSauvageJean@De Sauvage, Jean}
\index{SzymkiewiczFabien@Szymkiewicz, Fabien}
%
% Titre
\begin{flushleft}
\phantomsection\addtocounter{section}{1}
\addcontentsline{toc}{section}{Influence d'un écoulement souterrain sur les performances d'un système de fondations géothermiques}
{\Large \textbf{Influence d'un écoulement souterrain sur les performances d'un système de fondations géothermiques}}\label{ref:46}
\end{flushleft}
%
% Auteurs
Badr Ouzzine$^{1,\star}$, Thibault Badinier$^{1}$, Jean De Sauvage$^{1}$, Fabien Szymkiewicz$^{1}$\\[2mm]
$^{\star}$ \Letter : \url{badr.ouzzine@univ-eiffel.fr}\\[2mm]
{\footnotesize $^{1}$ GERS-SRO, Univ Gustave Eiffel, IFSTTAR, F-77447 Marne-la-Vallée}\\
[4mm]
%
% Mots clés
\noindent \textbf{Mots clés : } Géothermie, écoulement, coefficient de performance, modélisation numérique, modélisation physique\\[4mm]
%
% Résumé
\noindent \textbf{Résumé : } 

{\normalsize
L'évolution actuelle des villes nécessite des quantités d'énergie grandissantes et les enjeux écologiques actuels, soutenus par les réglementations thermiques régissant qui régissent les constructions poussent au développement d'énergies propres et renouvelables. C'est dans ce contexte que se développent les fondations géothermiques depuis les années 80.



Il s'agit d'une solution de géothermie de basse énergie où des tubes en polyéthylène sont fixés aux cages d'armature de fondations telles que des pieux. Un fluide calorifique circule alors dans ces tubes du sol jusqu'au bâtiment en passant par une pompe à chaleur inversible pour puiser la chaleur du sol afin de chauffer le bâtiment en hiver et vice-versa en été. On confère donc un rôle énergétique d'échange de chaleur aux fondations qui n'avaient jusqu'alors qu'un rôle mécanique de stabilité.



En comparaison d'autres solutions de géothermie, cette technique présente l'avantage de pouvoir être installée en temps masqué (donc réduire le temps de chantier) et sans forage supplémentaire (donc réduire les émissions de $\unit{CO_2}$). Enfin, il s'agit d'une source d'énergie renouvelable, non intermittente et que l'onpeut aisément coupler avec des apports photovoltaïques ou éoliens puisque reposant sur le fonctionnement d'une pompe à chaleur.



L'injection ou le prélèvement de chaleur dans le sol génère une modification de la température qui se déplace dans ce milieu poreux, saturé ou non. Très fréquemment, un écoulement d'eau souterrain dans une ou plusieurs épaisseurs du sol impacte fortement le déplacement de cette anomalie thermique et peut générer des interactions positives ou négatives entre les fondations d'un bâtiment. 



Une modélisation physique en semi-vraie grandeur a été réalisée dans SenseCity, une chambre à climat contrôlable pouvant recouvrir deux espaces de 400~$\unit{m^2}$. Sur chacun de ces espaces, est construite une portion de territoire, appelée Mini-Ville et l'un d'eux est équipé d'une fosse géothermique au sein de laquelle un écoulement peut être imposé. Un groupe de neuf pieux géothermiques y a été réalisé et instrumenté à l'aide de fibre optique.



Une modélisation numérique thermo-hydraulique a été effectuée par éléments finis et confrontée aux résultats expérimentaux précédents. Le couplage de ce modèle numérique à un programme simulant la demande de puissance d'une pompe à chaleur a ensuite permis de déterminer les puissances extractibles par ce groupe de pieux géothermiques. L'influence des caractéristiques de l'écoulement souterrain sur le coefficient de performance de la pompe à chaleur a également été étudiée.

 \vfill Work In Progress

}
 
