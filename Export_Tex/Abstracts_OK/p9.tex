

    %=====================================
    %   WARNING
    %   FICHIER AUTOMATISE
    %   NE PAS MODIFIER
    %=====================================

    \newpage

%%%%%%%%%%%%%%%%%%%%%%%%%%%%%%%%%%%%%%%%%%%%
%% Papier 9
%%%%%%%%%%%%%%%%%%%%%%%%%%%%%%%%%%%%%%%%%%%%

% Indexations
\index{PlaitAntony@Plait, Antony}
\index{DubasFrederic@Dubas, Frédéric}
%
% Titre
\begin{flushleft}
\phantomsection\addtocounter{section}{1}
\addcontentsline{toc}{section}{Estimation des pertes par courants de Foucault dans des parties massives conductrices à partir de mesures thermiques}
{\Large \textbf{Estimation des pertes par courants de Foucault dans des parties massives conductrices à partir de mesures thermiques}}\label{ref:9}
\end{flushleft}
%
% Auteurs
Antony Plait$^{1,\star}$, Frédéric Dubas$^{1}$\\[2mm]
$^{\star}$ \Letter : \url{antony.plait@gmail.com}\\[2mm]
{\footnotesize $^{1}$ Institut FEMTO-ST, UBFC}\\
[4mm]
%
% Mots clés
\noindent \textbf{Mots clés : } Pertes, courants de Foucault, transferts thermiques\\[4mm]
%
% Résumé
\noindent \textbf{Résumé : } 

{\normalsize
L'industrie automobile doit s'adapter à une profonde mutation de la mobilité et des transports individuels : une attente sociétale forte pour des solutions plus respectueuses de l'homme et de son environnement, qui mettent au centre l'énergie décarbonée et la réduction des consommations d'énergie. Dans ce sens, l'étude thermique des machines électriques a toujours été un domaine très attractif.







La recherche, réalisée dans cet article, s'inscrivent dans le projet COCTEL (Conception Optimale de Chaînes de Tractions Électriques) financé par l'ADEME, où il s'agit de créer un ensemble de solutions technologiques pour la traction électrique qui soient économiquement viables, et dont les performances soient comparables aux systèmes existants. Dans les machines électriques, l'estimation précise de la distribution des courants de Foucault dans les parties massives conductrices (i.e., aimants, cuivre, aluminium…) a toujours été un défi scientifique. Ces courants de Foucault, générés par une variation de champ magnétique spatio-temporel, impliquent des pertes volumiques et par conséquent des échauffements thermiques. Cela peut engendrer des conséquences irréversibles, tels que la démagnétisation des aimants dans les machines électriques. Un modèle analytique complexe a été conçu dans le but de déterminer de manière exacte la distribution des courants de Foucault, ainsi que des pertes engendrées.







À moyen terme, notre objectif est d'évaluer de manière expérimentale l'évolution de la température dans des parties massives conductrices (électrique et thermique) pour estimer les pertes par courants Foucault. Pour cela, un électroaimant à armature plate à entrefer variable, permettant ainsi d'insérer des parties massives conductrices de différentes hauteurs, est utilisé. Le champ magnétique (non-)uniforme spatialement généré est de forme sinusoïdale avec une fréquence de 50 Hz. En l'appliquant sur les parties massives conductrices en aluminium introduites dans l'entrefer, celles-ci vont s'échauffer. L'évolution temporelle de la température dT/dt mesurée expérimentalement permet d'estimer les pertes par courants de Foucault selon :



P = C.V.D.(dT/dt)



avec C la chaleur massique, V le volume et D la densité spécifique des parties massives conductrices.







La mesure de l'augmentation de température des parties massives conductrices se fait à l'aide de différents thermocouples de type E (non sensible aux variations du champ magnétique), qui sont fixés sur le matériau en différents points. Plusieurs mesures permettent obtenir la variation de température de manière plus fortement avérée.







L'influence de la segmentation sur l'évolution de la température et donc par conséquent sur les pertes générés par courants de Foucault est étudiée. Il est question de valider la modélisation analytique réalisée, à travers la comparaison des résultats correspondant aux différents cas d'étude.

 \vfill doi : \url{https://doi.org/10.25855/SFT2021-009}

}
 
