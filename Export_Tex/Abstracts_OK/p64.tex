

    %=====================================
    %   WARNING
    %   FICHIER AUTOMATISE
    %   NE PAS MODIFIER
    %=====================================

    \newpage

\backgroundsetup{contents={Work In Progress},scale=7}
\BgThispage
%%%%%%%%%%%%%%%%%%%%%%%%%%%%%%%%%%%%%%%%%%%%
%% Papier 64
%%%%%%%%%%%%%%%%%%%%%%%%%%%%%%%%%%%%%%%%%%%%

% Indexations
\index{MameriFateh@Mameri, Fateh}
\index{SchifflerJesse@Schiffler, Jesse}
\index{DelacourtEric@Delacourt, Eric}
\index{MorinCeline@Morin, Céline}
%
% Titre
\begin{flushleft}
\phantomsection\addtocounter{section}{1}
\addcontentsline{toc}{section}{Etude expérimentale et modélisation dynamique 0D d'un échangeur air-gaz brûlés pour une unité de micro-cogénération biomasse}
{\Large \textbf{Etude expérimentale et modélisation dynamique 0D d'un échangeur air-gaz brûlés pour une unité de micro-cogénération biomasse}}\label{ref:64}
\end{flushleft}
%
% Auteurs
Fateh Mameri$^{1}$, Jesse Schiffler$^{2}$, Eric Delacourt$^{1}$, Céline Morin$^{1,\star}$\\[2mm]
$^{\star}$ \Letter : \url{celine.morin@uphf.fr}\\[2mm]
{\footnotesize $^{1}$ LAMIH UMR CNRS 8201 - Université Polytechnique Hauts-de-France}\\
{\footnotesize $^{2}$ Laboratoire ICube UMR 8201 - Université de Strasbourg}\\
[4mm]
%
% Mots clés
\noindent \textbf{Mots clés : } échangeur air-gaz brûlés, étude expérimentale, modélisation dynamique, transferts convectifs et radiatifs, méthode inverse\\[4mm]
%
% Résumé
\noindent \textbf{Résumé : } 

{\normalsize
L'unité de micro-cogénération développée est constituée d'une chaudière biomasse d'une puissance de 30 kW, d'un moteur à air chaud de type Ericsson dédié à la production d'électricité et d'un échangeur de chaleur air-gaz brûlés inséré dans la chambre de combustion de la chaudière, échangeur qui permet d'alimenter en air chaud le moteur Ericsson. Une étude a été menée sur un échangeur de chaleur modèle qui doit résister à des températures élevées et à un environnement sévère (risque d'encrassement dans le pot de combustion de la chaudière). Une approche expérimentale a été conduite pour caractériser les échanges thermiques entre les gaz brûlés de la chaudière et l'air circulant à l'intérieur de l'échangeur, ainsi qu'une approche numérique avec le développement d'un modèle dynamique 0D sous le formalisme Bond Graph. L'échangeur de chaleur « modèle » est un tube en U et peut être divisé en trois zones suivant la direction de l'écoulement des deux fluides (air et gaz brûlés) : une zone counter-flow, une zone cross-flow et une zone co-flow. L'échangeur a été instrumenté à l'aide de plusieurs thermocouples afin de mesurer la température de l'air et de la paroi durant les essais et de capteurs de pression piézorésistifs pour évaluer les pertes de charge. L'effet de quatre paramètres (température de l'air, pression de l'air, débit de l'air et température de consigne imposée à la chaudière) sur les échanges de chaleur air – paroi a été étudié. La température de l'air mesurée dans la zone counter-flow demeure très faible par rapport à la zone co-flow alors que l'échangeur est plongé dans une chambre cylindrique et que la flamme se développe de façon axisymétrique. Un modèle dynamique 0D de l'échangeur air – gaz brûlés a été développé avec le formalisme Bond Graph pour simuler le comportement thermique de l'air circulant à l'intérieur. Une première étape a été de mener une étude CFD afin d'estimer la valeur des coefficients convectifs globaux dans chaque partie de l'échangeur pour alimenter le modèle 0D. Les coefficients des échanges de chaleur entre les gaz brûlés et la paroi de l'échangeur ramenés à la surface d'échanges côté gaz brûlés sont obtenus par méthode inverse. Ils incluent les effets convectifs et radiatifs. Le modèle 0D a été validé en comparant les profils expérimentaux et calculés de la température de l'air et de la paroi du tube dans les deux zones counter-flow et co-flow. Un très bon accord a été obtenu.

 \vfill Work In Progress

}
 
