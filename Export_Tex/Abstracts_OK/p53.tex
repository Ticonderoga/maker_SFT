

    %=====================================
    %   WARNING
    %   FICHIER AUTOMATISE
    %   NE PAS MODIFIER
    %=====================================

    \newpage

%%%%%%%%%%%%%%%%%%%%%%%%%%%%%%%%%%%%%%%%%%%%
%% Papier 53
%%%%%%%%%%%%%%%%%%%%%%%%%%%%%%%%%%%%%%%%%%%%

% Indexations
\index{BourgesColine@Bourgès, Coline}
\index{ChevalierStephane@Chevalier, Stéphane}
\index{SommierAlain@Sommier, Alain}
\index{PradereChristophe@Pradère, Christophe}
%
% Titre
\begin{flushleft}
\phantomsection\addtocounter{section}{1}
\addcontentsline{toc}{section}{Mesure transitoire et sans contact de champ de température par thermotransmittance dans des milieux semi-transparents à l'infrarouge}
{\Large \textbf{Mesure transitoire et sans contact de champ de température par thermotransmittance dans des milieux semi-transparents à l'infrarouge}}\label{ref:53}
\end{flushleft}
%
% Auteurs
Coline Bourgès$^{1,\star}$, Stéphane Chevalier$^{1}$, Alain Sommier$^{1}$, Christophe Pradère$^{1}$\\[2mm]
$^{\star}$ \Letter : \url{coline.bourges@u-bordeaux.fr}\\[2mm]
{\footnotesize $^{1}$ I2M}\\
[4mm]
%
% Mots clés
\noindent \textbf{Mots clés : } Thermotransmittance, Transitoire, Mesure sans contact, Infrarouge, Température absolue, Semi-transparent\\[4mm]
%
% Résumé
\noindent \textbf{Résumé : } 

{\normalsize
La compréhension des propriétés thermiques de matériaux est un atout dans nombreux domaines, y compris en biologie pour étudier des réactions chimiques dans des milieux vivants. L'imagerie multispectrale infrarouge associée à des méthodes inverses permet la caractérisation non destructive et sans contact des phénomènes thermiques. Cependant, lorsque la géométrie du système est trop complexe, comme dans les milieux vivants, les méthodes actuelles ne suffisent plus. Il devient alors nécessaire de développer un système d'imagerie infrarouge 3D afin d'avoir accès aux champs thermiques au sein de l'échantillon à caractériser.







Afin de réaliser ces mesures 3D dans des milieux semi-transparents à l'infrarouge, une étude préliminaire sur la compréhension de la thermo-dépendance de la transmission optique d'un matériau dans l'infrarouge est nécessaire. Ce phénomène est appelé thermotransmittance. Le coefficient de thermotransmittance est une propriété intrinsèque au matériau, par conséquent il ne nécessite pas de calibrer la caméra avec un corps noir pour mesurer le champ de température au sein du matériau. 







Dans un premier temps, il est nécessaire de mesurer le coefficient de thermotransmittance. Pour cela, l'échantillon à caractériser est éclairé par une source monochromatique dont la longueur d'onde est comprise dans la gamme de 3 à 6~$\unit{\mu m}$. Le flux transmis est collecté par une caméra infrarouge (FLIR SC7000). Le signal capté est la somme du flux transmis et de l'émission propre du matériau. Une méthode type lock-in à deux images est mise en place pour mesurer périodiquement l'émission propre (source IR incidente éteinte) puis la somme de l'émission propre et de la source incidente transmise (source IR incidente allumée). Ainsi, il devient possible de séparer les deux contributions par soustraction. Dans ce travail, seul le faisceau infrarouge transmis est étudié. Ensuite, à l'aide d'un four résistif, un flux de chaleur est imposé à l'échantillon afin de faire varier sa température. La mesure de l'absorbance en fonction de la température permet d'obtenir le coefficient de thermotransmittance pour une longueur d'onde. 







Une fois cette étape de caractérisation effectuée, il est désormais possible de mesurer la température absolue en chaque point de l'échantillon.







Dans cette communication, la calibration du coefficient de thermotransmittance sera détaillée pour plusieurs matériaux. Puis, des mesures de champs de température transitoires dans plusieurs matériaux semi-transparents seront présentées afin de valider la méthode, ouvrant la voie vers l'imagerie 3D de température dans les milieux complexes.

 \vfill doi : \url{https://doi.org/10.25855/SFT2021-053}

}
 
