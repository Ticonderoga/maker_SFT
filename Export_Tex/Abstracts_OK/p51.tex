

    %=====================================
    %   WARNING
    %   FICHIER AUTOMATISE
    %   NE PAS MODIFIER
    %=====================================

    \newpage

%%%%%%%%%%%%%%%%%%%%%%%%%%%%%%%%%%%%%%%%%%%%
%% Papier 51
%%%%%%%%%%%%%%%%%%%%%%%%%%%%%%%%%%%%%%%%%%%%

% Indexations
\index{ChenPin@Chen, Pin}
\index{HarmandSouad@Harmand, Souad}
\index{RiahiAli@Riahi, Ali}
\index{OuenzerfiSafouene@Ouenzerfi, Safouene}
\index{VeluGabriel@Velu, Gabriel}
\index{RomaryRaphael@Romary, Raphaël}
%
% Titre
\begin{flushleft}
\phantomsection\addtocounter{section}{1}
\addcontentsline{toc}{section}{Caractérisation thermique des fils et bobines isolés pour les machines électriques à haute température}
{\Large \textbf{Caractérisation thermique des fils et bobines isolés pour les machines électriques à haute température}}\label{ref:51}
\end{flushleft}
%
% Auteurs
Pin Chen$^{1,\star}$, Souad Harmand$^{1}$, Ali Riahi$^{1}$, Safouene Ouenzerfi$^{1}$, Gabriel Velu$^{2}$, Raphaël Romary$^{2}$\\[2mm]
$^{\star}$ \Letter : \url{Pin.Chen@uphf.fr}\\[2mm]
{\footnotesize $^{1}$ Laboratoire d'Automatique, de Mécanique et d'Informatique Industrielles et Humaines (LAMIH-UMR CNRS 8201), Université Polytechnique Hauts-de-France}\\
{\footnotesize $^{2}$ Laboratoire Systèmes Electrotechniques et Environnement, Faculté des Sciences Appliquées, Technoparc Futura, Université d'Artois}\\
[4mm]
%
% Mots clés
\noindent \textbf{Mots clés : } Isolation thermique,Technique infrarouge, Haute température\\[4mm]
%
% Résumé
\noindent \textbf{Résumé : } 

{\normalsize
Les machines électriques sont de plus en plus soumises à des conditions de travail de haute température et haute tension, entre autres dans des applications type  moteurs avioniques, les réacteurs nucléaires et les systèmes géothermiques. Dans ces domaines, la température nominale des bobines électriques peut augmenter jusqu'à 500 $^{\circ}$C. Ce qui s'avère être un verrou technologique  pour  l'isolation diélectrique des enroulements et des bobines. Les matériaux isolants organiques communs ne peuvent pas résister très longtemps à des  niveaux de température supérieur à 300 $^{\circ}$C, avec pour conséquence un vieillissement accéléré de l'isolant et une oxydation du conducteur métallique. Ainsi les matériaux composites inorganiques sont des remplaçants potentiels pour l'isolation des enroulements de machines électriques à haute température.

Dans cette étude, deux types de fils d'isolation de matériaux inorganiques sont caractérisés : la composite céramique et le ruban en fibre inorganique à base de mica. Si l'isolation en céramique présente une bonne résistance aux températures élevées, elle possède un faible rayon de courbure augmentant la difficulté de l'enroulement de bobine conventionnel. L'autre fil isolé inorganique est un fil enroulé en ruban de mica-verre, qui peut être fabriqué pour des bobines par un procédé d'enroulement conventionnel en raison d'une plus grande flexibilité. Cependant, une grande épaisseur de ruban (environ 200 $\unit{\mu m}$) est souvent nécessaire pour la stabilité mécanique ce qui créera des défis pour la miniature des équipements électriques notamment pour les fils de petites sections.

Dans le présent article, les performances thermiques d'un fil de cuivre nickelé isolé en céramique et du fil cuivre nickelé enrubanné  d'un ruban de mica-fibre de verre ont été étudiées avec des mesures en infrarouge. Le  fil de cuivre nickelé non isolé permet d'être un élément de référence. Les deux méthodes d'isolation peuvent réduire sensiblement la température de surface. Le calcul théorique et la simulation numérique démontrent des résultats similaires avec les mesures expérimentaux pour les trois fils.

Ensuite, quatre bobines enroulées par ces deux fils isolés (deux avec une imprégnation supplémentaire de ciment) ont été testées. Le but est de tester la performance de ces fils dans un cas applicatif  : cas du bobinage électrique (motorette). Les résultats de la cartographie thermique révèlent que l'assemblage des fils est défavorable à la dissipation thermique et que la température de surface augmente considérablement par rapport au fil unique (non bobiné). De plus, l'imprégnation au ciment réduit considérablement la température de surface et réagit lentement au changement de puissance de chauffage, produisant un effet d'inertie. Ces résultats de l'étude fournissent des informations intéressantes sur le management thermique d'une machine électrique à haute température.

 \vfill doi : \url{https://doi.org/10.25855/SFT2021-051}

}
