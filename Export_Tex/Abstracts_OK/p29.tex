

    %=====================================
    %   WARNING
    %   FICHIER AUTOMATISE
    %   NE PAS MODIFIER
    %=====================================

    \newpage

%%%%%%%%%%%%%%%%%%%%%%%%%%%%%%%%%%%%%%%%%%%%
%% Papier 29
%%%%%%%%%%%%%%%%%%%%%%%%%%%%%%%%%%%%%%%%%%%%

% Indexations
\index{GambadeJulien@Gambade, Julien}
\index{NoelHerve@Noël, Hervé}
\index{GlouannecPatrick@Glouannec, Patrick}
%
% Titre
\begin{flushleft}
\phantomsection\addtocounter{section}{1}
\addcontentsline{toc}{section}{Estimation « in situ » de l'efficacité de capteurs solaires sous vide pour la production d'eau chaude}
{\Large \textbf{Estimation « in situ » de l'efficacité de capteurs solaires sous vide pour la production d'eau chaude}}\label{ref:29}
\end{flushleft}
%
% Auteurs
Julien Gambade$^{1,\star}$, Hervé Noël$^{1}$, Patrick Glouannec$^{1}$\\[2mm]
$^{\star}$ \Letter : \url{julien.gambade@univ-ubs.fr}\\[2mm]
{\footnotesize $^{1}$ UBS / IRDL}\\
[4mm]
%
% Mots clés
\noindent \textbf{Mots clés : } Solaire Thermique, Mesures sur site, Bilan énergétique, Tubes sous vides\\[4mm]
%
% Résumé
\noindent \textbf{Résumé : } 

{\normalsize
L'Union Européenne s'est engagée à augmenter de 17 à 27\% la part d'énergie renouvelable produite en Europe pour 2030 afin de diminuer les émissions de gaz à effet de serre. C'est dans ce contexte que le laboratoire est impliqué dans le projet Interreg NWE ICaRE4Farms dont l'objectif est de réduire l'usage des procédés de chauffage traditionnels à énergies fossiles par la production d'eau chaude solaire. 







Nos travaux concernent la mise en œuvre d'une instrumentation adaptée à des suivis in situ pour poser des bilans thermiques et développer des modèles numériques. 







Le site étudié dans le cadre de cette étude est situé dans les Cotes d'Armor, il est destiné à produire de l'eau chaude pour un élevage. 







L'installation solaire thermique utilise 24 Capteurs solaires à tubes sous vide non pressurisés avec circulation directe du fluide caloporteur dans les tubes par thermosiphon et ballon de stockage (CSSV).







Cette installation est constituée de CSSV disposés tout d'abord en série pour la montée en température de l'eau puis en parallèle pour le stockage et le chauffage complémentaire. En fonctionnement standard, deux soutirages sont effectués quotidiennement. L'installation est non pressurisée, son soutirage est effectué par pompage et son remplissage par le réseau d'adduction en eau potable.







L'objet de la communication est de présenter la métrologie mise en œuvre à l'échelle d'un CSSV et de montrer la démarche mise en œuvre afin de bien appréhender son fonctionnement et ses performances énergétiques.







L'instrumentation mise en place, outre la station météorologique implantée sur le site, comprend des capteurs de température en entrée et sortie ainsi qu'à plusieurs niveaux dans le réservoir de stockage afin d'en étudier la stratification. Un débitmètre en entrée de l'installation nous donne le débit de circulation dans la rangée de CSSV.







L'étude a pour objectif de quantifier, pour un CSSV en série, l'énergie fournie par la circulation, l'énergie stockée dans le ballon de stockage ainsi que ses déperditions et ainsi déterminer son efficacité en usage. Deux séquences expérimentales sont présentées et analysées dans cette communication.

 \vfill doi : \url{https://doi.org/10.25855/SFT2021-029}

}
 
