

    %=====================================
    %   WARNING
    %   FICHIER AUTOMATISE
    %   NE PAS MODIFIER
    %=====================================

    \newpage

%%%%%%%%%%%%%%%%%%%%%%%%%%%%%%%%%%%%%%%%%%%%
%% Papier 8
%%%%%%%%%%%%%%%%%%%%%%%%%%%%%%%%%%%%%%%%%%%%

% Indexations
\index{DavidMartin@David, Martin}
\index{ToutantAdrien@Toutant, Adrien}
\index{BatailleFrancoise@Bataille, Françoise}
%
% Titre
\begin{flushleft}
\phantomsection\addtocounter{section}{1}
\addcontentsline{toc}{section}{Développement et analyse d'une corrélation pour estimer les transferts de chaleur en situation de fort chauffage asymétrique d'un écoulement en canal.}
{\Large \textbf{Développement et analyse d'une corrélation pour estimer les transferts de chaleur en situation de fort chauffage asymétrique d'un écoulement en canal.}}\label{ref:8}
\end{flushleft}
%
% Auteurs
Martin David$^{1,\star}$, Adrien Toutant$^{1}$, Françoise Bataille$^{1}$\\[2mm]
$^{\star}$ \Letter : \url{martin.david@promes.cnrs.fr}\\[2mm]
{\footnotesize $^{1}$ CNRS-PROMES (UPR 8521), Technosud-Rambla de la Thermodynamique, 66100 Perpignan, Université de Perpignan Via Domitia, 52 avenue Paul Alduy, 66100 Perpignan}\\
[4mm]
%
% Mots clés
\noindent \textbf{Mots clés : } Simulations des grandes échelles, Transferts thermique, Chauffage asymétrique, Corrélation\\[4mm]
%
% Résumé
\noindent \textbf{Résumé : } 

{\normalsize
Les travaux concernent l'étude de la sensibilité des transferts de chaleur, dans des conditions de forts chauffages asymétriques d'un écoulement turbulent en canal caractérisé par de hauts niveaux de température, par l'intermédiaire d'une corrélation. Cette étude vise à quantifier la précision de l'estimation des transferts de chaleur dans les récepteurs solaires à gaz sous-pression des centrales solaires à tour à l'aide d'un outil simple d'utilisation pour le pré-dimensionnement de ces centrales. Elle peut également être appliquée à des échangeurs de chaleur, à des systèmes électroniques ou dans le domaine du nucléaire par exemple. La corrélation utilisée est développée grâce à des Simulations numériques des Grandes Échelles (SGE). Pour mener à bien ces simulations, les équations de Navier-Stokes sont résolues sous l'hypothèse de bas nombre de Mach dans un canal plan bi-périodique mesurant 6 mm de hauteur, modélisant le récepteur solaire. Les couplages entre la thermique et la dynamique de l'écoulement (en particulier les effets de dilatation liés à la température) sont pris en compte afin de se rapprocher des conditions réelles de fonctionnement des récepteurs solaires. Les petites échelles de turbulence, non résolues en SGE, sont modélisées à l'aide du modèle sous-maille fonctionnel AMD. La corrélation proposée est en cohérence avec des expressions de références existantes et fait intervenir un nouveau terme, dédié au chauffage asymétrique. Ce terme devient neutre en cas de chauffage symétrique du fluide rendant la corrélation également utilisable dans ces conditions. La nouvelle corrélation permet d'estimer le flux de chaleur avec une précision de 10\% à partir des nombres de Reynolds et de Prandtl ainsi que des températures de fluide et de parois du canal. Cette corrélation, proposée dans une publication antérieure, couvre un large domaine d'application : le nombre de Reynolds est compris entre 10 000 et 180 000, les températures de fluide étudiées s'étendent de 340 K à 1240 K et le flux de chaleur atteint 580~$\unit{kW\cdot m^{-2}}$. Le nombre de Prandtl est dans la gamme 0,76-3,18. Dans ce travail, la sensibilité des résultats à la précision des données est étudiée puis analysée en fonction de la position dans le récepteur solaire. L'erreur commise sur le flux, en fonction de l'incertitude des mesures, est calculée. Il apparaît que les incertitudes sur les températures de fluide et de paroi froide sont critiques. En effet, une faible incertitude sur une de ces températures induit une erreur conséquente sur le flux estimé à la paroi froide. La sensibilité augmente de façon exponentielle à mesure que l'écart entre ces températures diminue, et donc que l'on progresse dans le récepteur solaire. Étant donné que le flux suit une tendance opposée, les plus grandes erreurs sont commises pour les flux les plus faibles. À la paroi chaude, l'erreur commise est moindre en raison du grand écart de température. L'influence des incertitudes sur le débit est elle aussi moins importante. Cette étude montre qu'il est possible d'adapter la précision des mesures de température en fonction de la position dans le récepteur solaire.

 \vfill doi : \url{https://doi.org/10.25855/SFT2021-008}

}
 
