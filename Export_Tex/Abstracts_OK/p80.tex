

    %=====================================
    %   WARNING
    %   FICHIER AUTOMATISE
    %   NE PAS MODIFIER
    %=====================================

    \newpage

\backgroundsetup{contents={Work In Progress},scale=7}
\BgThispage
%%%%%%%%%%%%%%%%%%%%%%%%%%%%%%%%%%%%%%%%%%%%
%% Papier 80
%%%%%%%%%%%%%%%%%%%%%%%%%%%%%%%%%%%%%%%%%%%%

% Indexations
\index{VoroncaStefan-Dominic@Voronca, Stefan-Dominic}
\index{SirouxMonica@Siroux, Monica}
\index{DarieGeorge@Darie, George}
\index{BouvenotJean-Baptiste@Bouvenot, Jean-Baptiste}
%
% Titre
\begin{flushleft}
\phantomsection\addtocounter{section}{1}
\addcontentsline{toc}{section}{Analyse d'un système de micro-cogénération biomasse}
{\Large \textbf{Analyse d'un système de micro-cogénération biomasse}}\label{ref:80}
\end{flushleft}
%
% Auteurs
Stefan-Dominic Voronca$^{1,\star}$, Monica Siroux$^{1}$, George Darie$^{2}$, Jean-Baptiste Bouvenot$^{1}$\\[2mm]
$^{\star}$ \Letter : \url{sdvoronca@gmail.com}\\[2mm]
{\footnotesize $^{1}$ INSA Strasbourg Laboratoire ICUBE, Université de Strasbourg}\\
{\footnotesize $^{2}$ Département de Génération et d'Usage de Puissance, Université Politehnica de Bucarest}\\
[4mm]
%
% Mots clés
\noindent \textbf{Mots clés : } micro-cogénération, biomasse, moteur Stirling, bâtiment, production d'énergie\\[4mm]
%
% Résumé
\noindent \textbf{Résumé : } 

{\normalsize
La micro-cogénération est une technologie par laquelle on produit simultanément de l'électricité et de la chaleur pour le chauffage et l'eau chaude sanitaire des logements, d'une manière décentralisée. La puissance électrique produite est inférieure à 50 kWel. En utilisant la biomasse en tant que combustible on a plusieurs avantages : on utilise de l'énergie renouvelable, la neutralité carbon, la disponibilité et le bas-prix. Dans cette étude, une installation de micro-cogénération biomasse à moteur Stirling développée par l'entreprise ÖkoFEN a été testée pour caractériser ses performances énergétiques. Le moteur Stirling est un moteur à piston libre de type dynamique. En plus du moteur Stirling, l'unité comprend une chambre combustion et des échangeurs de chaleur pour récupérer la chaleur des gaz de combustion. Le système utilise la technologie de condensation pour augmenter son efficacité. La puissance électrique nominale de l'installation de micro-cogénération biomasse est de 1 kW. La puissance thermique du système de micro-cogénération varie entre 10 et 14 kW. L'installation de micro-cogénération biomasse a été testée au sein du laboratoire INSA Strasbourg ICUBE. L'objectif de l'étude expérimentale a été d'observer  l'influence du débit de l'agent thermique et de la puissance de sortie thermique sur le comportement  de l'installation de micro-cogénération. Les essais ont été effectués pour neuf configurations différentes correspondant à trois débit de l'agent thermique différents et à trois puissances de sortie thermique différentes : 10 kW, 12 kW et 14 kW. Les températures et les débits ont été mesurés à l'aide des capteurs de température et de débitmètres. Le rendement global et les rendements thermique et électrique ont été déterminés pour chaque configuration. Les résultats obtenus ont montré que, si le débit n'est pas suffisant pour évacuer la puissance thermique produite par l'unité de micro-cogénération, la température de l'agent thermique va augmenter, réduisant ainsi les performances thermiques du système. Ces travaux expérimentaux ont permis de réaliser un modèle du système de micro-cogénération biomasse.

 \vfill Work In Progress

}
 
