

    %=====================================
    %   WARNING
    %   FICHIER AUTOMATISE
    %   NE PAS MODIFIER
    %=====================================

    \newpage

%%%%%%%%%%%%%%%%%%%%%%%%%%%%%%%%%%%%%%%%%%%%
%% Papier 50
%%%%%%%%%%%%%%%%%%%%%%%%%%%%%%%%%%%%%%%%%%%%

% Indexations
\index{SerreLilian@Serre, Lilian}
\index{BastideAlain@Bastide, Alain}
\index{VauquelinOlivier@Vauquelin, Olivier}
\index{JuhoorKarimKhan@Juhoor, Karim Khan}
\index{VarrallKevin@Varrall, Kévin}
%
% Titre
\begin{flushleft}
\phantomsection\addtocounter{section}{1}
\addcontentsline{toc}{section}{Propagation de fumées en façade avec contamination d'un local supérieur}
{\Large \textbf{Propagation de fumées en façade avec contamination d'un local supérieur}}\label{ref:50}
\end{flushleft}
%
% Auteurs
Lilian Serre$^{1,\star}$, Alain Bastide$^{1}$, Olivier Vauquelin$^{2}$, Karim Khan Juhoor$^{3}$, Kévin Varrall$^{2}$\\[2mm]
$^{\star}$ \Letter : \url{lilian.serre@univ-reunion.fr}\\[2mm]
{\footnotesize $^{1}$ Laboratoire PIMENT}\\
{\footnotesize $^{2}$ IUSTI}\\
{\footnotesize $^{3}$ Efuzif}\\
[4mm]
%
% Mots clés
\noindent \textbf{Mots clés : } Propagation de fumées, sécurité incendie, ventilation naturelle, façades, contamination\\[4mm]
%
% Résumé
\noindent \textbf{Résumé : } 

{\normalsize
Les constructions dites « bioclimatiques » sont en plein essor en milieu tropical, comme sur l'île de La Réunion. L'importante porosité aéraulique des façades de ces constructions permet de rafraîchir l'ambiance des bâtiments par la ventilation naturelle. Cependant, cette utilisation de la ventilation naturelle à des fins de confort thermique peut nuire aux objectifs de sécurité incendie.



L'écoulement interne d'une cellule où se développent les fumées d'un incendie dépend de la puissance de la source et de la géométrie du local (et en particulier la position des ouvrants). Ainsi, selon les conditions, l'ambiance peut être stratifiée ou mélangée, et les écoulements fluides aux ouvrants sont unidirectionnels (entrant ou sortant) ou bidirectionnels.



Le présent travail a pour objectif d'étendre ces études et de comprendre dans quelles mesures le panache de fumée issu de ce local dit « source » peut, par propagation en façade [4-5], contaminer un local qui lui est superposé.



Une étude numérique, réalisée sur le logiciel de MFN Fire Dynamics Simulator, permet d'aborder la dynamique de cette contamination. Dans une configuration académique, dans laquelle les locaux sont de mêmes dimensions et ouverts sur une face commune, le panache déversant est défini par ses conditions de température et vitesse d'injection imposées à la source (Ui = 0.5, 1, 1.5 et 2 m/s, et Ti = 80, 100, 200 et 300$^{\circ}$C). Le transport d'un polluant passif (ici le monoxyde de carbone) permet une quantification de la contamination du local supérieur.



Pour une température d'injection fixée, le taux de contamination décroit systématiquement avec l'augmentation de la vitesse d'injection. A contrario, l'augmentation de la température n'engendre que des variations peu significatives du taux de contamination pour de faibles vitesses d'injection, mais a un impact important pour les vitesses d'injection élevées.



L'aérodynamisme joue un rôle important dans cette contamination, la modélisation en Froude permettant classiquement de traiter la dynamique de panache n'est plus suffisante pour retranscrire la physique rencontrée ici.

 \vfill doi : \url{https://doi.org/10.25855/SFT2021-050}

}
 
