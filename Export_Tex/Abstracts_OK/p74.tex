

    %=====================================
    %   WARNING
    %   FICHIER AUTOMATISE
    %   NE PAS MODIFIER
    %=====================================

    \newpage

\backgroundsetup{contents={Work In Progress},scale=7}
\BgThispage
%%%%%%%%%%%%%%%%%%%%%%%%%%%%%%%%%%%%%%%%%%%%
%% Papier 74
%%%%%%%%%%%%%%%%%%%%%%%%%%%%%%%%%%%%%%%%%%%%

% Indexations
\index{Reynard-CaretteChristelle@Reynard-Carette, Christelle}
\index{VolteAdrien@Volte, Adrien}
\index{CaretteMichel@Carette, Michel}
\index{LyoussiAbdallah@Lyoussi, Abdallah}
\index{KohseGordon@Kohse, Gordon}
%
% Titre
\begin{flushleft}
\phantomsection\addtocounter{section}{1}
\addcontentsline{toc}{section}{Optimisation d'un calorimètre différentiel pour la mesure en ligne du débit de dose absorbée dans le réacteur nucléaire de recherche du MIT.}
{\Large \textbf{Optimisation d'un calorimètre différentiel pour la mesure en ligne du débit de dose absorbée dans le réacteur nucléaire de recherche du MIT.}}\label{ref:74}
\end{flushleft}
%
% Auteurs
Christelle Reynard-Carette$^{1,\star}$, Adrien Volte$^{1}$, Michel Carette$^{1}$, Abdallah Lyoussi$^{2}$, Gordon Kohse$^{3}$\\[2mm]
$^{\star}$ \Letter : \url{Christelle.carette@univ-amu.fr}\\[2mm]
{\footnotesize $^{1}$ Aix Marseille Univ, Université de Toulon, CNRS, IM2NP, Marseille, France}\\
{\footnotesize $^{2}$ CEA/DES/IRESNE/DER, Section of Experimental Physics, Safety Tests and Instrumentation, Cadarache, F-13108, Saint Paul-lez-Durance, France}\\
{\footnotesize $^{3}$ Massachusetts Institute of Technology, Nuclear Reactor Laboratory, Cambridge, Massachusetts, USA}\\
[4mm]
%
% Mots clés
\noindent \textbf{Mots clés : } Echauffement nucléaire, Calorimétrie, Réacteur nucléaire de recherche, Instrumentation\\[4mm]
%
% Résumé
\noindent \textbf{Résumé : } 

{\normalsize
Depuis fin 2009, Aix-Marseille Université et le CEA mènent des travaux de recherche sur la mesure en ligne du débit de dose absorbée pour des réacteurs nucléaires de recherche dans le cadre du laboratoire commun LIMMEX (Laboratoire d'Instrumentation et de Mesures en Milieux EXtrêmes). Le débit de dose absorbée, qui correspond à l'énergie déposée par les interactions rayonnements/matière par unité de temps et de masse, représente une grandeur clé en particulier pour le dimensionnement des expériences d'irradiation. Il peut atteindre plusieurs W/g (plusieurs kGy/s) dans de nombreux réacteurs (par exemple 20 W/g visés pour le Réacteur Jules Horowitz en cours de construction sur Cadarache). Il est aussi appelé échauffement nucléaire du fait de l'augmentation de température induite et quantifié grâce à des calorimètres à flux de chaleur. Le poster portera sur l'optimisation d'un calorimètre différentiel à éprouvettes calorimétriques compactes, nommé CALORRE, breveté en 2015 par AMU et le CEA, et validé en conditions réelles lors d'une campagne d'irradiations au sein du réacteur polonais MARIA. Cette étude d'optimisation est conduite dans le cadre du programme CALOR-I (Compact-CALORimeter Irradiations inside the MIT research reactor) financé par la fondation A*Midex en collaboration avec le CEA et le Nuclear Reactor Laboratory du MIT (2020-2023). Cette étude a pour objectif de proposer une nouvelle configuration du calorimètre CALORRE afin de mesurer pour la première fois l'échauffement nucléaire dans le MITR. Plus précisément, il s'agira de mener des expériences dans sa boucle fluide en cœur pour différentes conditions expérimentales (puissance du réacteur, température du fluide caloporteur, type et intensité de la convection dans la boucle) et de tester de nouvelles méthode de mesure avec et sans déplacement des deux éprouvettes superposées. Les optimisations ciblent l'utilisation de nouveaux matériaux et gaz, le changement du design de chaque éprouvette et de l'assemblage du calorimètre afin d'avoir un capteur avec un encombrement réduit (du même ordre que des calorimètres mono-éprouvette), une sensibilité adaptée aux faibles échauffements nucléaires (<2W/g), une réduction des températures atteintes et des non linéarités de la courbe de réponse (masse, effets du rayonnement thermique, résistance thermique de contact). Les résultats numériques de l'étude thermique paramétrique 3D sous COMSOL Multiphysics pour des conditions de laboratoire (étalonnage préalable) et des conditions d'irradiation seront présentés, discutés et comparés aux résultats expérimentaux obtenus avec des configurations précédentes de capteurs déjà testées.

 \vfill Work In Progress

}
 
