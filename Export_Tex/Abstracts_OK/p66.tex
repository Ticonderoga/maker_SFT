

    %=====================================
    %   WARNING
    %   FICHIER AUTOMATISE
    %   NE PAS MODIFIER
    %=====================================

    \newpage

\backgroundsetup{contents={Work In Progress},scale=7}
\BgThispage
%%%%%%%%%%%%%%%%%%%%%%%%%%%%%%%%%%%%%%%%%%%%
%% Papier 66
%%%%%%%%%%%%%%%%%%%%%%%%%%%%%%%%%%%%%%%%%%%%

% Indexations
\index{HadjAhmedAsmaa@Hadj Ahmed, Asmaa}
\index{DaurelleJean-Vincent@Daurelle, Jean-Vincent}
\index{FourmondVincent@Fourmond, Vincent}
\index{VicenteJerome@Vicente, Jerome}
%
% Titre
\begin{flushleft}
\phantomsection\addtocounter{section}{1}
\addcontentsline{toc}{section}{The design of new electrochemical cells for studying highly active metalloenzymes}
{\Large \textbf{The design of new electrochemical cells for studying highly active metalloenzymes}}\label{ref:66}
\end{flushleft}
%
% Auteurs
Asmaa Hadj Ahmed$^{1,\star}$, Jean-Vincent Daurelle$^{2}$, Vincent Fourmond$^{3}$, Jerome Vicente$^{2}$\\[2mm]
$^{\star}$ \Letter : \url{asmaa.hadj-ahmed@etu.univ-amu.fr}\\[2mm]
{\footnotesize $^{1}$ Laboratoire de Bioénergétique et Ingénierie des Protéines, UMR 7281 / Laboratoire IUSTI, UMR 7343 Aix Marseille Université/CNRS}\\
{\footnotesize $^{2}$ Laboratoire IUSTI, UMR 7343 Aix Marseille Université/CNRS, Polytech Marseille, Dpt Mécanique Energétique}\\
{\footnotesize $^{3}$ Laboratoire de Bioénergétique et Ingénierie des Protéines, Institut de Microbiologie de la Méditerranée, UMR 7281 Aix Marseille Université/CNRS}\\
[4mm]
%
% Mots clés
\noindent \textbf{Mots clés : } electrochemical cell, computational fluid dynamics, mass transport, optimization\\[4mm]
%
% Résumé
\noindent \textbf{Résumé : } 

{\normalsize
Protein film electrochemistry (PFE) is an electrochemical technique that is used for studying metalloenzymes. It consists in adsorbing a film of enzyme on a rotating disc electrode (RDE) in a configuration where the electron transfer is direct, and the enzymatic activity is monitored as an electrical current. This technique has proved extremely useful to study various aspects of the activity of different metalloenzymes. However, its application can be limited especially when it comes to highly active enzymes such as CODHs. These enzymes are so fast that the catalysis is mostly limited by the transport of the substrate (CO) towards the electrode, and not by the catalyzed chemical reaction even at the highest rotation rate of the RDE. This limitation can hide information about the catalysis in the electrochemical response.



So, to overcome this problem, in a previous study, by means of computational fluid dynamics, our team designed and built a new electrochemical cell (jet-flow cell) that provides better mass transport properties than RDE. As a result, mass transport was improved but it wasn't enough for our application. Thus, the design must be optimized.


Following this study, in order to validate the numerical model, we conducted cyclic voltammetry experiments on the newly built electrochemical cell, by using a simple redox couple [ $Fe(CN_6)^3$ / $Fe(CN_6)^{3-}$ ]. The results showed an excellent agreement between the simulation and the experiments which allowed us to validate our model. Furthermore, we succeeded for the first time to use the new cell for detecting the catalytic current given by an adsorbed enzyme, the nitrite reductase. Afterwards, we implemented a systematic study for the purpose of optimizing the geometry of the new design by simulation. In this work, we show the experimental results and the optimization study that we adopted to improve the transport in the new cell.

 \vfill Work In Progress

}
 
