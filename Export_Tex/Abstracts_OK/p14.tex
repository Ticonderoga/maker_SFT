

    %=====================================
    %   WARNING
    %   FICHIER AUTOMATISE
    %   NE PAS MODIFIER
    %=====================================

    \newpage

%%%%%%%%%%%%%%%%%%%%%%%%%%%%%%%%%%%%%%%%%%%%
%% Papier 14
%%%%%%%%%%%%%%%%%%%%%%%%%%%%%%%%%%%%%%%%%%%%

% Indexations
\index{TurquaisBenjamin@Turquais, Benjamin}
\index{SansJean-Louis@Sans, Jean-Louis}
\index{DavoustLaurent@Davoust, Laurent}
\index{DelacroixJules@Delacroix, Jules}
\index{JourneauChristophe@Journeau, Christophe}
\index{PilusoPascal@Piluso, Pascal}
\index{ChikhiNourdine@Chikhi, Nourdine}
%
% Titre
\begin{flushleft}
\phantomsection\addtocounter{section}{1}
\addcontentsline{toc}{section}{Métrologie à très haute température (1300-2500$^{\circ}$C) en pyroréflectométrie pour des applications nucléaires}
{\Large \textbf{Métrologie à très haute température (1300-2500$^{\circ}$C) en pyroréflectométrie pour des applications nucléaires}}\label{ref:14}
\end{flushleft}
%
% Auteurs
Benjamin Turquais$^{1,\star}$, Jean-Louis Sans$^{2}$, Laurent Davoust$^{3}$, Jules Delacroix$^{1}$, Christophe Journeau$^{1}$, Pascal Piluso$^{1}$, Nourdine Chikhi$^{4}$\\[2mm]
$^{\star}$ \Letter : \url{benjamin.turquais2@cea.fr}\\[2mm]
{\footnotesize $^{1}$ CEA, DES, IRESNE, DTN, Cadarache F-13108 Saint-Paul-Lez-Durance, France}\\
{\footnotesize $^{2}$ Laboratoire PROMES-CNRS, 7 rue du four solaire, 66120 Font-Romeu Odeillo (France)}\\
{\footnotesize $^{3}$ Grenoble-INP/Université Grenoble Alpes /CNRS, Laboratoire SIMaP, EPM Group, 38402 Saint Martin d'Hères, France}\\
{\footnotesize $^{4}$ CEA, DES, IRESNE, DEC, Cadarache F-13108 Saint-Paul-Lez-Durance, France}\\
[4mm]
%
% Mots clés
\noindent \textbf{Mots clés : } Métrologie, très haute température, incertitudes, pyroréflectométrie, nucléaire\\[4mm]
%
% Résumé
\noindent \textbf{Résumé : } 

{\normalsize
Afin d'améliorer la compréhension des accidents graves dans les réacteurs nucléaires et le comportement des matériaux constituant le cœur fondu (corium, acier), les propriétés thermophysiques (densité, tension de surface, viscosité) de ces matériaux doivent être déterminées à très haute température (T > 2000 $^{\circ}$C). Le niveau de précision de ces propriétés thermophysiques dépend directement de la qualité de la mesure de la température.



La qualité des mesures est rendue possible par la détermination des incertitudes associées. Or l'incertitude sur la mesure de température peut être significative (jusqu'à 10\%) à très haute température.



La pyrométrie est utilisée en particulier sur l'installation VITI de la plateforme accident grave PLINIUS du CEA pour estimer la température à partir du rayonnement émis par la surface ainsi que son émissivité pour chacune des longueurs d'onde de travail.



Afin de réduire les incertitudes sur la mesure de température, une technique innovante par rapport à la pyrométrie, appelée pyroréflectométrie, est utilisée pour corriger les températures monochromatiques avec les réflectivités spectrales des échantillons considérés. La détermination complète des incertitudes portant sur la mesure de températures est fonction de l'incertitude due à l'étalonnage et de l'incertitude due aux capteurs et à la chaîne de mesure, laquelle inclue la nécessaire correction des températures par les réflectivités. Dans le but de réduire l'impact de l'étalonnage sur le bilan final des incertitudes, l'étalonnage a été réalisé sur des cellules à points fixes eutectiques de qualité métrologique. Une instrumentation adaptée a également été mise en œuvre, incluant notamment le dimensionnement d'une tête optique spécifique au pyroréflectomètre. Enfin, un réglage minutieux du dispositif a permis d'utiliser les réflectivités pour corriger les températures monochromatiques et obtenir une température, proche de la température du système visé, avec une incertitude considérablement réduite.



La comparaison des incertitudes sur les températures obtenues par pyrométrie et par pyroréflectométrie montre qu'un bon étalonnage ainsi qu'une chaîne de mesure optimisée permet de réduire significativement les incertitudes de l'ordre de 10\% pour les pyromètres bichromatiques à environ 1-2\% pour le pyroréflectomètre sur la gamme de température allant de 1300 à 2500$^{\circ}$C.

 \vfill doi : \url{https://doi.org/10.25855/SFT2021-014}

}
 
