

    %=====================================
    %   WARNING
    %   FICHIER AUTOMATISE
    %   NE PAS MODIFIER
    %=====================================

    \newpage

\backgroundsetup{contents={Work In Progress},scale=7}
\BgThispage
%%%%%%%%%%%%%%%%%%%%%%%%%%%%%%%%%%%%%%%%%%%%
%% Papier 13
%%%%%%%%%%%%%%%%%%%%%%%%%%%%%%%%%%%%%%%%%%%%

% Indexations
\index{UllahZia@Ullah, Zia}
\index{MangiFareedHussain@Mangi, Fareed Hussain}
\index{AhmedAftab@Ahmed, Aftab}
%
% Titre
\begin{flushleft}
\phantomsection\addtocounter{section}{1}
\addcontentsline{toc}{section}{Influence of tracer injection location on mixing in a curved pipe}
{\Large \textbf{Influence of tracer injection location on mixing in a curved pipe}}\label{ref:13}
\end{flushleft}
%
% Auteurs
Zia Ullah$^{1}$, Fareed Hussain Mangi$^{1,\star}$, Aftab Ahmed$^{2}$\\[2mm]
$^{\star}$ \Letter : \url{fareed.mangi@iba-suk.edu.pk}\\[2mm]
{\footnotesize $^{1}$ Energy Systems Engineering Department, Sukkur IBA University}\\
{\footnotesize $^{2}$ Mechanical Technology Department, Indus University, Karachi}\\
[4mm]
%
% Mots clés
\noindent \textbf{Mots clés : } Mixing enhancement, Curved pipe, Numerical Simulations\\[4mm]
%
% Résumé
\noindent \textbf{Résumé : } 

{\normalsize
Mixing enhancement in laminar regime is of great importance because of its usage in many industrial applications such as in food and pharmaceutical industries as well as in micro Algae production. These are the areas where the quality of product can't be altered in turbulent regime due to the high shear forces between the fluid molecules which can result in an inappropriate outcome of the process. 

In this study, numerical simulations through commercial code Ansys Fluent 18.2 on 2D steady laminar flow in a 90$^{\circ}$ curved pipe are performed to evaluate effect of tracer injection location on mixing enhancement. A 90$^{\circ}$ curved geometry with a radius of curvature of 0.5 for the steady Reynolds number range from 300 to 1000 is designed in Solidworks 2016. Total length of pipe is 200mm. Geometry consist of two inlets and one outlet. The main inlet is large velocity inlet with diameter of 40mm through which hot water having temperature equal to 313K enter into elbow while the small inlet have diameter of 5mm through which cold water having temperature 293K. Fluid eject through pressure outlet of the elbow having diameter of 40mm. Boundary condition parameters i.e. velocity, pressure and temperature were set according to the changing Reynolds number. 


Overall solution and simulation consist four steps i.e. geometry design, meshing, setup for problem solution, and post processing which includes results and discussions. 



Interesting qualitative and quantitative results for pressure, velocity and temperature are discussed on the basis of tracer injection position to study mixing enhancement. Different vertical positions in a 90o curved pipe  from upper wall to the bottom wall with an increment of 1cm, 2cm and 3cm and the effect of Reynolds number are investigated.  







Many conclusions/ directions can be drawn form here but, In this study focus remained only on temperature, velocity, and pressure distribution in the curved mixing phenomenon.

 \vfill Work In Progress

}
 
