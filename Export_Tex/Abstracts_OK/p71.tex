

    %=====================================
    %   WARNING
    %   FICHIER AUTOMATISE
    %   NE PAS MODIFIER
    %=====================================

    \newpage

%%%%%%%%%%%%%%%%%%%%%%%%%%%%%%%%%%%%%%%%%%%%
%% Papier 71
%%%%%%%%%%%%%%%%%%%%%%%%%%%%%%%%%%%%%%%%%%%%

% Indexations
\index{BeaumaleMarion@Beaumale, Marion}
\index{LavieillePascal@Lavieille, Pascal}
\index{MiscevicMarc@Miscevic, Marc}
%
% Titre
\begin{flushleft}
\phantomsection\addtocounter{section}{1}
\addcontentsline{toc}{section}{Métrologie infrarouge haute précision pour la détermination des coefficients de transfert en condensation convective}
{\Large \textbf{Métrologie infrarouge haute précision pour la détermination des coefficients de transfert en condensation convective}}\label{ref:71}
\end{flushleft}
%
% Auteurs
Marion Beaumale$^{1,\star}$, Pascal Lavieille$^{1}$, Marc Miscevic$^{1}$\\[2mm]
$^{\star}$ \Letter : \url{marion.beaumale@laplace.univ-tlse.fr}\\[2mm]
{\footnotesize $^{1}$ LAPLACE}\\
[4mm]
%
% Mots clés
\noindent \textbf{Mots clés : } condensation convective, métrologie infrarouge, coefficient d'échange, expérimental\\[4mm]
%
% Résumé
\noindent \textbf{Résumé : } 

{\normalsize
Les systèmes diphasiques sont envisagés comme des solutions de refroidissement pour de multiples applications. La particularité de ce type de systèmes est le couplage fort entre performances thermiques et structuration des phases liquide et vapeur au sein de l'écoulement. La prédiction des transferts de chaleur demeure encore aujourd'hui l'une des problématiques principales dans le dimensionnement des systèmes mettant en œuvre la condensation en film à faible vitesse massique. L'originalité du dispositif expérimental proposé est sa capacité à mesurer localement et simultanément l'épaisseur du film liquide et le coefficient de transfert de chaleur. Pour la mesure des épaisseurs des films de condensats un capteur confocal chromatique et des mesures interférométriques en lumière blanche ont été utilisés. Les mesures de température de paroi quant à elles ont été obtenues par thermographie infrarouge. Le saphir a été choisi comme matériau du condenseur de par sa transparence dans le domaine visible permettant la visualisation de l'écoulement, et sa haute conductivité thermique améliorant les transferts de chaleur radiaux dans le tube. La section test, placée entre deux réservoirs à pression constante, est refroidie par de l'air conditionné. Ce choix est motivé par la transparence de l'air dans le domaine des longueurs d'ondes infrarouges détectées par la caméra thermique. En raison du mode de refroidissement utilisé (i.e. convection forcée d'air) la détermination du coefficient de transfert de chaleur nécessite une extrême précision de mesure de la température de paroi difficile à atteindre par les outils habituels. Le travail réalisé porte sur la mise en place d'une technique de mesure permettant la détermination avec une haute précision (à 0.05$^{\circ}$C) de la température de paroi d'un tube vertical en saphir par caméra infrarouge lors de la condensation convective du HFE 7000 en écoulement descendant. Le protocole développé, ainsi que sa calibration, seront présentés en détails dans la communication.

 \vfill doi : \url{https://doi.org/10.25855/SFT2021-071}

}
 
