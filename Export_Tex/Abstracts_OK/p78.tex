

    %=====================================
    %   WARNING
    %   FICHIER AUTOMATISE
    %   NE PAS MODIFIER
    %=====================================

    \newpage

\backgroundsetup{contents={Work In Progress},scale=7}
\BgThispage
%%%%%%%%%%%%%%%%%%%%%%%%%%%%%%%%%%%%%%%%%%%%
%% Papier 78
%%%%%%%%%%%%%%%%%%%%%%%%%%%%%%%%%%%%%%%%%%%%

% Indexations
\index{ShamsborhanHiva@Shamsborhan, Hiva}
\index{MenanteauSebastien@Menanteau, Sébastien}
\index{AbdelNourFadi@Abdel Nour, Fadi}
%
% Titre
\begin{flushleft}
\phantomsection\addtocounter{section}{1}
\addcontentsline{toc}{section}{Contribution à la modélisation numérique des phénomènes convectifs dans une enceinte chauffée.}
{\Large \textbf{Contribution à la modélisation numérique des phénomènes convectifs dans une enceinte chauffée.}}\label{ref:78}
\end{flushleft}
%
% Auteurs
Hiva Shamsborhan$^{1,\star}$, Sébastien Menanteau$^{1}$, Fadi Abdel Nour$^{2}$\\[2mm]
$^{\star}$ \Letter : \url{hiva.shamsborhan@icam.fr}\\[2mm]
{\footnotesize $^{1}$ Domaine Energétique, Environnement et Matériaux, Icam Lille, 6 rue Auber, 59016 Lille}\\
{\footnotesize $^{2}$ IUT de Béthune, LGCgE, 1230 rue de l'Université CS20819 62400 Béthune}\\
[4mm]
%
% Mots clés
\noindent \textbf{Mots clés : } Convection naturelle, confort thermique, simulation numérique.\\[4mm]
%
% Résumé
\noindent \textbf{Résumé : } 

{\normalsize
Le confort thermique des personnes dans une enceinte chauffée peut parfois être problématique en raison de la stratification de la température et des mouvements d'air causés par la formation de cellules de convection. La modélisation numérique de ces phénomènes peut ainsi s'avérer intéressante dans le but de mieux appréhender leurs mécanismes ou de diagnostiquer d'éventuelles zones d'inconfort pour l'utilisateur. Pour autant, ces études numériques peuvent être complexes à mener, notamment en raison des sources d'incertitudes liées aux conditions limites souvent peu maitrisées pour de tels espaces, les solutions obtenues pouvant en outre fortement dépendre des méthodes numériques sélectionnées. Dans le but d'investiguer les performances des modèles numériques en convection naturelle, une maquette expérimentale aux conditions thermiques parfaitement maîtrisées a ainsi été développée. Cette maquette positionnée dans une chambre climatique thermo-régulée, est chauffée au moyen d'un flux de chaleur constant et présente une ouverture afin d'engendrer un mouvement convectif d'air. Elle a été instrumentée de manière à obtenir une cartographie précise de la température intérieure en vue de comparer ces mesures à un ensemble de simulations numériques réalisées au moyen d'Ansys Fluent. L'aptitude de différentes approches de modélisation a ainsi pu être comparée en vue de déterminer les meilleurs usages possibles de l'outil numérique pour résoudre la thermique de l'écoulement. 



L'ensemble des études paramétriques nous a permis de prendre connaissance des bonnes pratiques pour la simulation numérique de la maquette. Le modèle instationnaire, en prenant en compte le rayonnement thermique couplé avec la convection naturelle, est le plus adapté au phénomène physique. Egalement, bien que les parois de la maquette soient fabriquées avec des matières isolantes, l'attribution d'un faible coefficient de convection aux parois présente une meilleure cohérence avec les résultats expérimentaux. La prise en compte des propriétés de l'air variant avec la température selon les fonctions polynômes présente mieux l'existence des cellules convectives près des parois.

 \vfill Work In Progress

}
 
