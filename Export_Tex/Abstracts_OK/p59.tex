

    %=====================================
    %   WARNING
    %   FICHIER AUTOMATISE
    %   NE PAS MODIFIER
    %=====================================

    \newpage

%%%%%%%%%%%%%%%%%%%%%%%%%%%%%%%%%%%%%%%%%%%%
%% Papier 59
%%%%%%%%%%%%%%%%%%%%%%%%%%%%%%%%%%%%%%%%%%%%

% Indexations
\index{GuilletGabriel@Guillet, Gabriel}
\index{GasparJonathan@Gaspar, Jonathan}
\index{FasquelleThomas@Fasquelle, Thomas}
\index{BarbosaSeverine@Barbosa, Séverine}
\index{KadochBenjamin@Kadoch, Benjamin}
\index{PizzoYannick@Pizzo, Yannick}
\index{RigolletFabrice@Rigollet, Fabrice}
\index{GardareinJean-Laurent@Gardarein, Jean-Laurent}
\index{LeNiliotChristophe@Le Niliot, Christophe}
%
% Titre
\begin{flushleft}
\phantomsection\addtocounter{section}{1}
\addcontentsline{toc}{section}{Estimation par méthode inverse du flux absorbé par une plaque en fonte destinée à la cuisson solaire}
{\Large \textbf{Estimation par méthode inverse du flux absorbé par une plaque en fonte destinée à la cuisson solaire}}\label{ref:59}
\end{flushleft}
%
% Auteurs
Gabriel Guillet$^{1,\star}$, Jonathan Gaspar$^{1}$, Thomas Fasquelle$^{1}$, Séverine Barbosa$^{1}$, Benjamin Kadoch$^{1}$, Yannick Pizzo$^{1}$, Fabrice Rigollet$^{1}$, Jean-Laurent Gardarein$^{1}$, Christophe Le Niliot$^{1}$\\[2mm]
$^{\star}$ \Letter : \url{gabriel.guillet@univ-amu.fr}\\[2mm]
{\footnotesize $^{1}$ Aix Marseille Univ, CNRS, IUSTI, Marseille, France}\\
[4mm]
%
% Mots clés
\noindent \textbf{Mots clés : } Techniques inverses, estimation de flux, cuiseur solaire\\[4mm]
%
% Résumé
\noindent \textbf{Résumé : } 

{\normalsize
Parmi les cuiseurs solaires, les cuiseurs de type Scheffler semblent prometteurs. Ils ont en effet un foyer fixe et sont capables de fournir une puissance suffisamment importante pour avoir des applications dans la restauration, l'agro-alimentaire et l'industrie. Dans le cadre de l'étude d'un cuiseur de ce type et de son intégration dans un restaurant à haute qualité environnementale, une méthode de caractérisation des performances énergétiques du système, faisant appel à la métrologie et aux techniques inverses, est en cours de développement.







Le cuiseur solaire évoqué est constitué d'un réflecteur de Scheffler, d'un réflecteur secondaire et d'une plaque en fonte. Celle-ci joue à la fois le rôle d'absorbeur et le rôle de plaque de cuisson.







Dans un premier temps, un modèle de l'optique du système et un modèle thermique de la plaque en fonte ont été élaborés. Ce dernier décrit notamment la distribution spatiale de la température dans la plaque au cours du temps et en fonction du flux absorbé.







Dans un second temps, on s'est intéressé aux techniques de mesure permettant de vérifier expérimentalement les performances énergétiques du système. Afin de déterminer l'efficacité de sa partie optique, il était nécessaire de mesurer le rayonnement solaire direct reçu par le réflecteur de Scheffler et le flux solaire concentré reçu par la plaque. Tandis que le premier se mesure facilement, la mesure du second est complexifiée par son inhomogénéité et sa valeur élevée. Afin de pallier cette difficulté, la plaque a été assimilée à un capteur en vue d'estimer le flux.







En effet, l'application des techniques inverses au transfert thermique par conduction dans la plaque a rendu l'estimation du flux de chaleur absorbé possible. La résolution du problème inverse s'est appuyée sur le modèle thermique 3D non-linéaire de la plaque et sur les mesures de l'expérience présentée ci-dessous.







Afin de recueillir les données expérimentales nécessaires, un banc d'expérimentation a été développé. Celui-ci est constitué d'un panneau rayonnant carré faisant face à ladite plaque en fonte disposée verticalement. Le flux de rayonnement reçu, constant mais non uniforme, a d'abord été mesuré en plusieurs points à l'aide d'un fluxmètre. Ensuite, la plaque a été disposée à la place du fluxmètre. Sa face arrière, non exposée au rayonnement, a été équipée de thermocouples et a été filmée par une caméra infrarouge alors qu'elle était soumise à un échelon de rayonnement. Sa réponse thermique a été mesurée et enregistrée.







Finalement, cette méthode permet d'estimer l'intensité du flux absorbé mais également de connaître sa distribution spatiale, tout en utilisant une instrumentation relativement simple et peu intrusive.







L'article résumé ci-dessus présentera brièvement le contexte et l'objectif de l'étude puis décrira le modèle thermique de la plaque en fonte, le banc d'expérimentation, son instrumentation et les mesures obtenues. La méthode de résolution du problème inverse sera détaillée et les résultats analysés.

 \vfill doi : \url{https://doi.org/10.25855/SFT2021-059}

}
 
