

    %=====================================
    %   WARNING
    %   FICHIER AUTOMATISE
    %   NE PAS MODIFIER
    %=====================================

    \newpage

%%%%%%%%%%%%%%%%%%%%%%%%%%%%%%%%%%%%%%%%%%%%
%% Papier 21
%%%%%%%%%%%%%%%%%%%%%%%%%%%%%%%%%%%%%%%%%%%%

% Indexations
\index{BouzidSihem@Bouzid, Sihem}
\index{BendadaLarbi@Bendada, Larbi}
\index{HebbirNacer@Hebbir, Nacer}
\index{HarnaneYamina@Harnane, Yamina}
%
% Titre
\begin{flushleft}
\phantomsection\addtocounter{section}{1}
\addcontentsline{toc}{section}{Optimisation des paramètres d'un échangeur de chaleur avec agitateur par la méthode taguchi et l'algorithme génétique}
{\Large \textbf{Optimisation des paramètres d'un échangeur de chaleur avec agitateur par la méthode taguchi et l'algorithme génétique}}\label{ref:21}
\end{flushleft}
%
% Auteurs
Sihem Bouzid$^{1,\star}$, Larbi Bendada$^{1}$, Nacer Hebbir$^{2}$, Yamina Harnane$^{3}$\\[2mm]
$^{\star}$ \Letter : \url{sihembouzid69@gmail.com}\\[2mm]
{\footnotesize $^{1}$ Université d'Oum El Bouaghi, Algérie. Laboratoire conception et modélisation avancée des systèmes mécaniques et thermo fluides. Oum El Bouaghi}\\
{\footnotesize $^{2}$ Université d'Oum El Bouaghi, Algérie. Laboratoire des Matériaux et Structure des Systèmes Électromécaniques et leurs Fiabilité (LMSEF)}\\
{\footnotesize $^{3}$ Université d'Oum El Bouaghi, Algérie. Laboratoire de Génie Mécanique (LGM), Biskra }\\
[4mm]
%
% Mots clés
\noindent \textbf{Mots clés : } échangeur de chaleur, optimisation, Taguchi, algorithmes génétiques\\[4mm]
%
% Résumé
\noindent \textbf{Résumé : } 

{\normalsize
Dans le domaine industriel, les ingénieurs énergéticiens ont besoin de savoir l'optimum de fonctionnement des échangeurs de chaleur dans un équipement donné. Cette information leur permet de décider quel échangeur est le meilleur pour leur construction, et si l'échangeur existe déjà quels sont les paramètres de fonctionnement optimums pour assurer les rendements attendus. La réaction chimique nécessitant une homogénéisation de la température requiert un échangeur de chaleur avec agitateur. Notre étude a été réalisée sur un banc d'essai de ce type (TD360d) en deux parties : Réalisation des tests de fonctionnement de l'échangeur en agissant sur trois paramètres : débit de fluide chaud, vitesse de rotation de l'agitateur et température d'entrée du fluide chaud ; puis l'utilisation des algorithmes de Taguchi et génétiques afin d'optimiser ces paramètres pour maximiser l'efficacité de l'échangeur. Deux méthodes ont été utilisées pour optimiser l'efficacité thermique de l'échangeur de chaleur. Une étude paramétrique préliminaire de 54 tests fonctionnels a été menée. Les paramètres de fonctionnement sont le débit du fluide chaud, la température d'entrée et la vitesse de rotation de l'agitateur. Deux modes de fonctionnement ont été considérés : le mode serpentin et le mode enveloppe. L'analyse de Taguchi a permis d'optimiser les niveaux de paramètres : A1B3C1 (A : débit, B : vitesse, C : Température) pour le mode serpentin et A3B3C1 pour le mode enveloppe. L'efficacité moyenne de l'échangeur de chaleur est de 0,2964 et 0,4100 respectivement dans les boîtiers de serpentin et d'enveloppe. L'évaluation des générations pour l'optimisation du rendement est d'environ 20 générations avec un optimum égal à 0,2952 et 0,3365 pour les modes serpentin et enveloppe, respectivement. Les meilleures valeurs des individus obtenues via l'Algorithme Génétique (AG) pour l'optimisation de l'efficacité sont A3B1C1 en mode serpentin et A1B1C1 en mode Enveloppe. On conclut que l'éfficacité Optimum de fonctionnement de l'échangeur de chaleur de ce type atteint  41\% en mode enveloppe pour un débit 3l/min, une vitesse de rotation 100, une température d'entrée chaude 40$^{\circ}$C.

 \vfill doi : \url{https://doi.org/10.25855/SFT2021-021}

}
 
