

    %=====================================
    %   WARNING
    %   FICHIER AUTOMATISE
    %   NE PAS MODIFIER
    %=====================================

    \newpage

%%%%%%%%%%%%%%%%%%%%%%%%%%%%%%%%%%%%%%%%%%%%
%% Papier 52
%%%%%%%%%%%%%%%%%%%%%%%%%%%%%%%%%%%%%%%%%%%%

% Indexations
\index{GuillouPaul@Guillou, Paul}
\index{MarcOlivier@Marc, Olivier}
\index{AdelardLaetitia@Adelard, Laetitia}
\index{MadyiraDaniel@Madyira, Daniel}
\index{AkinlabiEsther@Akinlabi, Esther}
\index{Castaing-LasvignottesJean@Castaing-Lasvignottes, Jean}
%
% Titre
\begin{flushleft}
\phantomsection\addtocounter{section}{1}
\addcontentsline{toc}{section}{Transferts couplés de masse et de chaleur en milieu poreux. Application au séchage de la bagasse de canne à sucre par comparaison numérique et expérimentale.}
{\Large \textbf{Transferts couplés de masse et de chaleur en milieu poreux. Application au séchage de la bagasse de canne à sucre par comparaison numérique et expérimentale.}}\label{ref:52}
\end{flushleft}
%
% Auteurs
Paul Guillou$^{1,\star}$, Olivier Marc$^{1}$, Laetitia Adelard$^{1}$, Daniel Madyira$^{2}$, Esther Akinlabi$^{2}$, Jean Castaing-Lasvignottes$^{1}$\\[2mm]
$^{\star}$ \Letter : \url{paul.guillou@univ-reunion.fr}\\[2mm]
{\footnotesize $^{1}$ Université de La Réunion - Laboratoire PIMENT}\\
{\footnotesize $^{2}$ Université de Johannesbourg}\\
[4mm]
%
% Mots clés
\noindent \textbf{Mots clés : } Séchage, Milieux poreux, Modélisation, Simulation, Expérimentation\\[4mm]
%
% Résumé
\noindent \textbf{Résumé : } 

{\normalsize
La biomasse représente une ressource renouvelable et abondante pouvant faire l'objet de différentes valorisation (énergétique, matériaux de construction, paillage agricole par exemple). Cependant ce type de matériau est caractérisé par une forte teneur en eau pouvant constituer un frein à son utilisation directe et qu'un séchage peut alors contribuer à lever. Un tel procédé est le siège de transferts de masse et de chaleur couplés en présence de différentes phases dans un milieu poreux. Pour étudier ce séchage, nous avons choisi deux approches complémentaires. La première est expérimentale et permet de rendre compte des spécificités de la biomasse tandis que la deuxième est numérique et permet de développer un modèle capable de rendre compte des comportements dynamiques expérimentaux observés. Cette démarche est appliquée à un matériau produit localement à l'île de La Réunion : la bagasse de canne à sucre.







L'approche expérimentale consiste à conditionner plusieurs kilogrammes de bagasse dans une géométrie cylindrique et disposée dans une enceinte thermorégulée. L'échantillon est instrumenté afin de permettre le suivi de la teneur en eau globale, de l'humidité relative locale et de la température à 9 positions différentes au cours du temps. Deux campagnes de séchages ont été réalisées pour deux températures de consignes constantes : 50$^{\circ}$C et 60$^{\circ}$C. Les résultats permettent notamment de mettre en avant la succession de plusieurs phases de séchage et d'obtenir la distribution spatiale de la température pour la comparaison avec les résultats numériques.







L'approche numérique a consisté à développer un modèle dynamique local en deux dimensions en considérant la biomasse comme un milieu continu équivalent et composé de trois phases (solide, gazeuse et liquide). Les équations de conservations, de transferts, d'états et d'équilibres sont établies sur l'ensemble du domaine et moyennées sur des volumes élémentaires représentatifs. Le système d'équations obtenu est résolu selon la méthode des volumes finis permettant d'obtenir les variables locales et globales des grandeurs précédemment suivies.







Une première comparaison des résultats expérimentaux et de simulation est réalisée sur la campagne de séchage de la bagasse à 50$^{\circ}$C et a permis l'identification des 4 coefficients de transferts du modèle (massique et thermiques, internes et aux limites). Les deux premiers concernent les transferts par conduction de chaleur et par diffusion moléculaire de la phase gazeuse tandis que les deux derniers concernent les échanges externes massiques et thermique entre l'échantillon et l'air. Une deuxième comparaison est réalisée sur la deuxième campagne de séchage pour la température de consigne de 60$^{\circ}$C afin de valider les coefficients précédemment identifiés. Les résultats obtenus permettent d'obtenir une bonne reproduction des dynamiques locales des températures aux différentes positions, de l'humidité relative au centre et de l'évolution globale de perte en eau de la bagasse pour les deux campagnes de séchages. Le modèle s'avère alors capable d'évaluer localement les flux thermiques et massiques mis en jeux dans un tel milieu.

 \vfill doi : \url{https://doi.org/10.25855/SFT2021-052}

}
 
