%!TEX root = ./abstracts/abstract.tex
\chapter{Le mot du comité d'organisation}

Pour la deuxième année consécutive, compte tenu de la situation exceptionnelle causée par l'épidémie de Coronavirus (covid 19), le 29\ieme{} congrès français de thermique a été annulé en présentiel. 

Cependant, le Conseil Scientifique de la Société Française de Thermique, en étroite collaboration avec le Comité d'Organisation et le Conseil Scientifique de la conférence, ont décidé d'organiser cette manifestation sous la forme d'une web-conférence qui se déroulera du 1\ier{} au 3 juin 2021 sur le thème \textbf{THERMIQUE et MIX ENERGETIQUE}.

Les pays du monde se sont engagés, par l'accord de Paris, à réduire fortement leurs émissions de gaz à effet de serre afin de limiter l'impact du changement climatique sur nos sociétés. Le  changement climatique trouve sa cause dans la production de gaz à effet de serre dont environ 70\% résulte de notre consommation d'énergies fossiles et l'utilisation du charbon, du pétrole et du gaz rend la croissance non soutenable. La Programmation Pluriannuelle de l'Energie (PPE) et la Stratégie Nationale Bas-Carbone (SNBC)  permettront d'inscrire la transition énergétique de la France dans un mouvement national mais également dans le cadre du développement du marché intérieur européen et de la transition énergétique européenne. La PPE décrit ainsi les mesures qui permettront à la France de décarboner l'énergie afin d'atteindre la neutralité carbone en 2050. En Europe, le secteur des transports est le premier consommateur d'énergie, devant ceux du bâtiment, de l'industrie et des services. La consommation énergétique de l'Europe s'est élevée à environ 1 352 millions de tonnes d'équivalent pétrole (Mtep) en 2019 pour environ 447 millions d'habitants, contre environ 2 264 Mtep aux États-Unis pour environ 333 millions d'habitants, selon l'Agence internationale de l'énergie.

Le thème \textbf{THERMIQUE et MIX ENERGETIQUE} sera l'occasion de mettre l'accent sur des sujets de recherche industriels et académiques à travers cinq conférences plénières :
\begin{itemize}[label=\textbullet]
    \item M. Pierre Montagne (General Electric, Gas Power) présentera les problématiques de combustion au sein des auxiliaires de turbines à gaz ;
    \item M. Charles Foulquié (SAFRAN) abordera la conception des turbofans dans le domaine de l'aviation ;
    \item Mme. Zlatina Dimitrova (Stellantis) exposera les activités de recherche opérées dans le domaine de  l'efficacité énergétique des véhicules automobiles ;
    \item M. Franck David (EDF R\&D) balaiera le panorama des activités du groupe industriel face aux enjeux énergétiques et à la nécessaire réduction des émissions de gaz à effets de serre ;
    \item Mme. Sylvie Lorente (Villanova university, USA) clôturera le congrès en présentant l'application de la loi constructale au stockage thermique.
\end{itemize}

Lors de ce congrès, quatre ateliers permettront des échanges sur des thématiques transversales :

\begin{itemize}[label=\textbullet]
    \item le groupe thématique « Transfert thermique atmosphérique et adaptation aux changements climatiques » créé en 2020 aura pour objectifs l'animation scientifique d'activités associant la thermique aux problématiques liées au climat. Cet atelier sera animé par Frédéric André, Cyril Caliot et Nicolas Ferlay ;
    \item les concours de l'enseignement supérieur et du CNRS feront l'objet d'une animation par Souad Harmand, pour le CNU 62, et Jean-Luc Battaglia, pour le CNRS section 10. Ces présentations, à destination notamment des jeunes chercheurs, permettront d'apporter quelques clés de compréhension et des conseils pour la préparation des concours d'enseignants-chercheurs et de chercheurs,
    \item l'atelier Hydrogène-énergie, à travers des exemples d'activités industrielles et de recherche, sera animé par Nadia Steiner et Olivier Joubert. Cet atelier exposera les enjeux scientifiques et technologiques de cette alternative aux énergies carbonées que représente l';
    \item Bernard Desmet présentera la banque de données thermophysiques.
\end{itemize}

Nous avons ainsi reçu 86 propositions de résumés. 50 communications ont été finalement acceptées pour publication dans les Actes du congrès annuel de la SFT 2021 et bénéficient dorénavant d'un DOI. 30 communications ont été retenues pour être présentées sous forme d'affiche uniquement. L'ensemble des travaux, représentant un total de 152 expertises, font ainsi l'objet d'une publication sur le site internet de la Société Française de Thermique.

Le Conseil Scientifique de la Société Française de Thermique a classé 6 communications pour le Prix Biot-Fourier et seront proposées pour publication dans la revue \textit{Entropie : thermodynamique – énergie – environnement – économie (ISTE)}. Le Comité local d'organisation du congrès attribuera également le Prix du meilleur poster.

La web-conférence sera l'occasion d'organiser, pour la première fois de son histoire, le 29\ieme{} congrès français de thermique de Belfort de manière totalement dématérialisée. Ainsi, toutes les 80 communications feront l'objet d'une présentation orale qui seront transmises en simultané sur des plateformes (Teams et/ou Youtube).

Le Comité d'Organisation du congrès remercie très sincèrement le Conseil d'Administration et le Conseil Scientifique de la Société Française de Thermique pour leur soutien renouvelé et leur confiance dans l'organisation du 29\ieme{} congrès.

A toutes et à tous, nous vous souhaitons un très bon congrès 2021 !


\begin{flushright}
{\bfseries François Lanzetta}\\
{Philippe Baucour, Sylvie Bégot et Valérie Lepiller}
\end{flushright}

